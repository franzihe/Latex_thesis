% !TeX spellcheck = en_GB
% ************************** Thesis Abstract *****************************
% Use `abstract' as an option in the document class to print only the titlepage and the abstract.
\begin{abstract}
\noindent
\textcolor{red}{This abstract is from the GEO4012 Literature study course. And has to be updated!}
Previous studies show that retrievals can lead to non-unique snowfall accumulation. Regional models are still under development to represent mesoscale features precisely.
The Haukeliseter site is used to measure snow accumulation since winter 2010\,/\,2011. This new approach makes it unique since it is exposed to high wind speeds and snow amounts of 2\,-\,3\,m are observed during a regular winter season.
During Christmas 2016 an extreme weather event occurred at Haukeliseter. High wind speeds combined with high precipitation of snow and rain were monitored.
Optimal estimation retrievals are under construction to measure the amount of snowfall accumulation. Retrievals such as the one for CloudSat try to avoid the non-uniqueness of different snowfall habits and uses a priori information such as temperature. Another approach is to use microphysical information from MASC and PIP to reduce uncertainties in the radar snowfall retrieval.
The MetCoOP ensemble prediction system is operational at the Norwegian Meteorological institute since 2016. Preliminary comparison between MEPS and double fence gauge measurements show an overestimation of snowfall accumulation with longer lead time.
\end{abstract}