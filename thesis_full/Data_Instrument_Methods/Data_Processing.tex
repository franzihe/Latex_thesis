% !TeX spellcheck = en_GB
\section{Computing Snow Quantities} \label{sec:data_proc}
%The previous \Cref{sec:retrieval,sec:MEPS} represented the details on retrieving snowfall amounts from the optimal estimation retrieval and the forecast model outputs. 
The following section describes how the different model variables where processed to achieve a comparison between the retrieved observed values and the forecast model output. 

%%%%%%%%%%%%%%%%%%%%%%%%%%%%%%%%%%%%%%%%%%%%%%%%%%%%%%%%%%%%%%%%%%%%%%%%%%
%%%%%%%%% THICKNESS IN MEPS %%%%%%%%%%%%%%
\subsection{MEPS}\label{sec:layer_thickness}
Vertical hybrid coordinates are terrain-following and are mass-based, \citep{muller_arome-metcoop:_2017}. %How the vertical hybrid coordinates are transformed into layer thickness or height is described in \Cref{sec:layer_thickness}. Furthermore, 
MEPS underlies non-hydrostatic dynamics, \cite{metcoop_wiki_description_2017}.
\\
MEPS has a vertical resolution in hybrid sigma pressure coordinates, which starts at the surface and decreases with height. To calculate the actual vertical pressure in [\SI{}{\hPa}], AROME uses the vertical discretisation based on \citet{simmons_energy_1981}. %, a formula is provided in the OPeNDAP Dataset of \texttt{meps\_full\_2\_5km\_*.nc} by the \citet{norwegian_meteorological_institute_met_2016}.  
\begin{align}
p(n,k,j,i) = a_p(k) + b_p(k) \cdot p_s(n,j,i) \qquad [\SI{}{\Pa}].
\label{eq:hybrid_sigma_pressure}
\end{align}
$p_s$ is the surface air pressure in \SI{}{\Pa}, $a_p$, $b_p$ are the vertical coordinate parameters. Prognostic variables are at model level $k$. $n$ is the forecast time, $j$ and $i$ are the latitudal and longitudal grid points, respectively.
\\
The next step was to convert pressure-levels into actual heights with the help of the hypsometric equation. Here, the air temperature in model levels is used to calculate the mean temperature of each thickness layer. 
\begin{align}
\overline{T} = \dfrac{\int\limits_{p2}^{p1} T \partial ln p}{\int\limits_{p2}^{p1}\partial ln p} \qquad [\SI{}{\kelvin}]
\label{eq:T_avg}
\end{align}
%For the numerical integration, the Simpson rule was used, which is a build-in function in Python. \\
\citet{martin_mid-latitude_2006} presents steps of differentiating the hypsometric equation by using the virtual air temperature. But when the atmospheric mixing ratio is large, the virtual temperature will only be \SI{1}{\percent} larger than the actual air temperature. Since the error is small, calculations are done with the provided air temperature in model levels.
\\
The thickness, $\Delta z$, of each layer is then calculated by using the hypsometric equation from \citet{martin_mid-latitude_2006} and the previously calculated mean temperature (\Cref{eq:T_avg}):
\begin{equation}
\begin{split}
\Delta z  = z_2 - z_1 
& = \frac{R_d \overline{T}}{g} ln\left(\frac{p_1}{p_2} \right) \qquad [\SI{}{\metre}]
\end{split}
\label{eq:hypsometric}
\end{equation}
where $R_d$ is gas constant for dry air with a value of \SI{287}{\joule\per\kilogram\per\kelvin},  standard gravity $g\,=\,$\SI{9.81}{\metre\per\square\second}. $p_1$ and $p_2$ are the pressure levels at lower and higher levels %, respectively 
($p_2 < p_1$).
%To gain the respective height of each pressure layer, $\Delta z$ is summed.
\newpage
%%%%%%%%%%%%%%%%%%%%%%%%%%%%%%%%%%%%%%%%%%%%%%%%%%%%%%%%%%%%%%%%%%%%%%%%%%
%%%%%%%%% SWC %%%%%%%%%%%%%%
\subsection{Snow Water Content}
To get a valid comparison between the SWC from the optimal estimation retrieval and the results from MEPS, the SWC is averaged hourly respectively three hourly. Taking the model initialisation of MEPS at \SI{0}{\UTC} the instantaneous model forecast values are saved every hour for the deterministic and first ensemble member. Three hourly data is available for the second to tenth perturbed member. To approach hourly mean values from the retrieved SWC an average over \SI{30}{\minute} prior and \SI{30}{\minute} after each full hour is performed. %This leads to a match of the average value at the same time as from MEPS. 
Hence, in \Cref{sec:res:large_scale_vert} and \ref{sec:res:oro_infl} hourly (three-hourly) retrieved averages are compared with instantaneous forecast values.
\\
Since MEPS has a higher vertical resolution than the optimal estimation snowfall retrieval, each vertical profile of SWC is averaged in \SI{200}{\metre} intervals. To accomplish the same vertical resolution only values above \SI{100}{\metre} altitude are used to start at the same range height as given from the MRR (\Cref{sec:MRR}).
\\
Within the output from MEPS snow water content does not exist for each model layer. Hence the calculation of the SWC is performed by using the three solid precipitation categories given in MEPS. Namely the instantaneous mixing ratio of snow ($r_\mathbf{s}$), graupel  ($r_\mathbf{g}$) and the cloud ice ($r_\mathbf{i}$). The mixing ratios are represented in \SI{}{\kg\per\kg} and a transformation to \SI{}{\g\per\cubic\meter} is performed. Densities of each model level ($\rho$) are calculated with \Cref{eq:hybrid_sigma_pressure} and the model level air temperature. These are then multiplied with the sum of the solid precipitation mixing ratio. 
\begin{align}
\rho & = \frac{p}{R_d T} & \quad [\SI{}{\kg\per\cubic\meter}]  \\
SWC & = \rho \cdot (r_\mathbf{s} + r_\mathbf{g} + r_\mathbf{i}) \cdot \num{e6} & \quad [\SI{}{\g\per\cubic\meter}].
\label{eq:SWC_ml}
\end{align}
The same is done for liquid water content in the atmosphere, just cloud condensed water and rain amount are summed.

%%%%%%%%%%%%%%%%%%%%%%%%%%%%%%%%%%%%%%%%%%%%%%%%%%%%%%%%%%%%%%%%%%%%%%%%%%
%%%%%%%%% SWP %%%%%%%%%%%%%%
\subsection{Snow Water Path}\label{sec:SWP}
The snow water path (SWP) is the vertically integrated and averaged SWC (\Cref{eq:SWC,eq:SWC_ml}): %, where the numerical Simpson's integration is applied.  
\begin{equation}
\begin{split}
SWP = \int_{h_0}^{h_1=\SI{3000}{\metre}} \text{SWC}(h) \, dh \approx 
\frac{h_1 - h_0}{6}  & \left[ \text{SWC}(h_0)    + \text{SWC}(h_1)   \vphantom{\frac{h_0 + h_1}{2}} \right. \\ 
& \left. + 4 \text{SWC}\left(\frac{h_0 + h1}{2}\right)  
\right] \qquad [\SI{}{\g\per\square\meter}]
\end{split}
\label{eq:SWP}
\end{equation}
The snow water path is a measure of the weight of frozen particles per unit area. It indicates the total amount of ice in the atmosphere.
%%%%%%%%%%%%%%%%%%%%%%%%%%%%%%%%%%%%%%%%%%%%%%%%%%%%%%%%%%%%%%%%%%%%%%%%%%
\newpage
\section{Statistics}
For this thesis I calculated ensemble mean, coefficient of variation, mean error, mean absolute error, and percent difference. 
%%%%%%%%% Ensemble spread %%%%%%%%%%%%%%
%\subsection{Ensemble Mean and Coefficient of Variation}
\label{sec:ens_mean_spread}
The model ensemble mean ($\bar{f}$) is the average of all ten ensemble members ($f_i$) of MEPS.
\begin{align}
\bar{f} & = \frac{\sum_{i=1}^n f_i}{n} \label{eq:meanMEPS}
\end{align}
%The standard deviation, $\sigma$ of the ten ensemble members is divided by the mean of all ensemble members. 
The coefficient of variation (CV) is represented by the standard deviation, $\sigma$  with respect to the ensemble mean.
% \\
%This coefficient gives the possibility to compare the SWC results for different days with different values. 
%It follows for even a low ensemble spread of SWC (standard deviation of all ensemble members) then the different members do not need to be less variable.
The standard deviation is defined as:
\begin{align}
\sigma & = \sqrt{\frac{\sum_{i=1}^n (f_i - \bar{f})^2}{n-1}} .\label{eq:stdMEPS}
\end{align}
The CV gives the variability between the individual ensemble member forecast for SWC (\Cref{tab:verification}). Furthermore, the variation around the deterministic forecast shows. Since the CV is normed over the ensemble mean, daily variations can be compared to each other for the Christmas 2016 storm period (\Cref{sec:res:large_scale_vert}).
%which follows for the coefficient of variation:
\begin{align}
CV & = \frac{\sigma}{\bar{f}}
\end{align}
% \\
% To verify how well MEPS performed during the Christmas storm a variation of the SWC is calculated. For this, the standard deviation (\Cref{eq:stdMEPS}) is divided by the ensemble mean (\Cref{eq:meanMEPS}).
%
%%%%%%%%%%%%%%%%%%%%%%%%%%%%%%%%%%%%%%%%%%%%%%%%%%%%%%%%%%%%%%%%%%%%%%%%%%
%%%%%%%%% MAE & ME %%%%%%%%%%%%%%
%\subsection{Mean Error and Mean Absolute Error}
In \Cref{sec:res:large_scale_sfc} mean error ($ME$) and mean absolute error ($MAE$) is used. The $ME$ for each model ensemble member and from the observations ($o_i$) is calculated with: 
\begin{align}
ME = \frac{\sum_{i=1}^n f_i - o_i}{n}
\end{align}
%where $MEPS_{ens}$ represents the value of each ensemble member and $DoFe_{obs}$ defines the observation from the double fence. 
For the mean absolute error ($MAE$) follows then:
\begin{align}
MAE = \frac{\sum_{i=1}^n \left| f_i - o_i\right|}{n} \label{eq:MAE}
\end{align}
%
%%%%%%%%%%%%%%%%%%%%%%%%%%%%%%%%%%%%%%%%%%%%%%%%%%%%%%%%%%%%%%%%%%%%%%%%%%
%%%%%%%%% %Diff, %Avg Diff %%%%%%%%%%%%%%
%\subsection{Percent Difference and Average Difference}
The percentage difference of double fence gauge and retrieved surface accumulation (\Cref{tab:res:ret_sens} and \ref{tab:res:ret_error}) and MEPS forecasts (\Cref{tab:res:MEPS_err} and \ref{tab:res:MEPS_err_10}) are calculated by:
\begin{align}
\SI{}{\percent} \text{Difference} & = \frac{f_i - o_i}{o_i} \times 100
\end{align}
$f_i$ presents the snowfall from the retrieval or the MEPS ensemble forecast. The mean is then taken from all \SI{}{\percent} Difference values to see the average deviation during the Christmas storm 2016.

