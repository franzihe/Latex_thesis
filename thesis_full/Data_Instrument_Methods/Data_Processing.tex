% !TeX spellcheck = en_GB
\section{Data processing} \label{sec:data_proc}
%The previous \Cref{sec:retrieval,sec:MEPS} represented the details on retrieving snowfall amounts from the optimal estimation retrieval and the forecast model outputs. 
The following section will describe how the different variables where processed to achieve a comparison between the retrieved values and the forecast model output. 

%%%%%%%%%%%%%%%%%%%%%%%%%%%%%%%%%%%%%%%%%%%%%%%%%%%%%%%%%%%%%%%%%%%%%%%%%%
%%%%%%%%% THICKNESS IN MEPS %%%%%%%%%%%%%%
\subsection{Layer thickness in MEPS}\label{sec:layer_thickness}
To compare the measurements from the surface with the MEPS data, the closest grid point to Haukeliseter, is used.
\\
MEPS has a vertical resolution in hybrid sigma pressure coordinates, were one is at the surface and decreases with height. To calculate the actual vertical pressure in \SI{}{\Pa}, a formula is provided in the OPeNDAP Dataset of \texttt{meps\_full\_2\_5km\_*.nc} by the \cite{norwegian_meteorological_institute_met_2016}.  
\begin{align}
	p(n,k,j,i) = a_p(k) + b(k) \cdot p_s(n,j,i) \qquad [\SI{}{\Pa}].
	\label{eq:hybrid_sigma_pressure}
\end{align}
$p_s$ is the surface air pressure in \SI{}{\Pa}, and information about the variables $a_p$, $b$ are not given from the access form. \textcolor{red}{Find reference for sigma-hybrid coordinate transformation equation.}  
\\
The next step was to convert pressure-levels into actual heights by the use of the hypsometric equation. Here, the air temperature in model levels is used to calculate the mean temperature of each layer. 
\begin{align}
	\overline{T} = \dfrac{\int\limits_{p2}^{p1} T \partial ln p}{\int\limits_{p2}^{p1}\partial ln p} \qquad [\SI{}{\kelvin}]
	\label{eq:T_avg}
\end{align}
For the numerical integration, the Simpson rule was used, which is a build-in function in Python. \\
\cite{martin_mid-latitude_2006} presents steps of differentiating the hypsometric equation by using the virtual air temperature. But when the atmospheric mixing ratio is large, will the virtual temperature only be \SI{1}{\percent} larger than the actual air temperature. Since the error is little calculations are done with the provided air temperature in model levels.
\\
The thickness, $\Delta z$, of each layer is then be found by using the hypsometric equation from \cite{martin_mid-latitude_2006} and the previously calculated mean temperature (\Cref{eq:T_avg}):
\begin{equation}
\begin{split}
\Delta z  = z_2 - z_1 
& = \frac{R_d \overline{T}}{g} ln\left(\frac{p_1}{p_2} \right) \qquad [\SI{}{\metre}]
\end{split}
\label{eq:hypsometric}
\end{equation}
where $R_d$ is gas constant for dry air with a value of \SI{287}{\joule\per\kilogram\per\kelvin},  standard gravity $g\,=\,$\SI{9.81}{\metre\per\square\second}. $p_1$ and $p_2$ are the pressure levels at lower and higher levels, respectively ($p_2 < p_1$).
To gain the respective height of each pressure layer, $\Delta z$ is summed.

%%%%%%%%%%%%%%%%%%%%%%%%%%%%%%%%%%%%%%%%%%%%%%%%%%%%%%%%%%%%%%%%%%%%%%%%%%
%%%%%%%%% SWC %%%%%%%%%%%%%%
\subsection{Snow water content}
To get a valid comparison between the SWC from the optimal estimation retrieval and the results from MEPS, the SWC is averaged over each hour. Taking the model initialisation of MEPS at \SI{0}{\UTC} the model produces forecast values at \SI{0}{}, \SI{1}{}, \SI{2}{}, $\ldots$, \SI{22}{}, \SI{23}{}, $\ldots$, \SI{66}{\UTC}. To approach hourly mean values from the retrieval SWC an average over \SI{30}{\minute} prior and \SI{29}{\minute} after each full hour is performed. This leads to a match of the average value at the same time as from MEPS. \\
Since MEPS has a higher vertical resolution than the optimal estimation snowfall retrieval each vertical profile of SWC is averaged every \SI{200}{\metre}. To accomplish the same vertical resolution only values above \SI{100}{\metre} are used to start at the same range height as given from the MRR (\Cref{sec:MRR}).
\\
Within the output from MEPS snow water content does not exist for each model layer. Hence the calculation of the SWC is performed by using the three solid precipitation categories given in MEPS. Namely the mixing ratio of snowfall ($r_s$), graupelfall ($r_g$) and the atmosphere cloud ice content ($r_i$). The mixing ratios are represented in \SI{}{\kg\per\kg} and a transformation to \SI{}{\g\per\cubic\meter} is performed. Densities in each model level ($\rho_{ml}$) are calculated and then multiplied with the sum of the solid precipitation mixing ratio. 
\begin{align}
	\rho_{ml} & = \frac{p_{ml}}{R_d T} & \quad [\SI{}{\kg\per\cubic\meter}]  \\
	SWC_{ml} & = \rho_{ml} \cdot (r_s + r_g + r_i)_{ml} \cdot \num{e6} & \quad [\SI{}{\g\per\cubic\meter}].
	\label{eq:SWC_ml}
\end{align}

%%%%%%%%%%%%%%%%%%%%%%%%%%%%%%%%%%%%%%%%%%%%%%%%%%%%%%%%%%%%%%%%%%%%%%%%%%
%%%%%%%%% SWP %%%%%%%%%%%%%%
\subsection{Snow water path}
The snow water path (SWP) is the vertically integrated value of the averaged SWC (\Cref{eq:SWC,eq:SWC_ml}), where the numerical Simpson's integration is applied.  
\begin{equation}
\begin{split}
\int_{h_0}^{h_1=\SI{3000}{\metre}} \text{SWC}(h) \, dh \approx 
\frac{h_1 - h_0}{6}  & \left[ \text{SWC}(h_0)    + \text{SWC}(h_1)   \vphantom{\frac{h_0 + h1}{2}} \right. \\ 
& \left. + 4 \text{SWC}\left(\frac{h_0 + h1}{2}\right)  
\right] \qquad [\SI{}{\g\per\square\meter}]
\end{split}
\label{eq:SWP}
\end{equation}
The snow water path is a measure of the weight of ice particles per unit area. It indicates the total amount of ice in the atmosphere.

%%%%%%%%%%%%%%%%%%%%%%%%%%%%%%%%%%%%%%%%%%%%%%%%%%%%%%%%%%%%%%%%%%%%%%%%%%


\subsection{Dew point temperature for Skew-T log-P Diagram}
The Python module \texttt{pyMeteo} is used to calculate the dew point temperature of each ensemble member to study the stability of the atmosphere (\url{https://pythonhosted.org/pymeteo/}, last visited: 25.01.2018). The additional package \texttt{thermo.py} is able to calculate the dew point temperature if the pressure and the specific humidity in each level are known. 
\begin{align}
	e_l & = ln\left( \frac{\frac{q_v}{\epsilon} \cdot \frac{p}{100}}{1 + \frac{q_v}{\epsilon}} \right) \\
	T_d & = 273.15 + \frac{243.5 \cdot e_l -440.8}{19.48 -e_l} \qquad [\SI{}{\kelvin}]
\end{align}
where, $q_v$ is the specific humidity, $p$ pressure in [\SI{}{\Pa}], $\epsilon = R_d / R_v = 0.622$ with $R_v$ gas constant for water vapour.