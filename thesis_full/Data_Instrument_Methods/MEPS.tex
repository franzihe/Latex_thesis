% !TeX spellcheck = en_GB
\chapter{Numerical forecast model} \label{ch:MEPS}
MEPS was newly operational at Met-Norway when the extreme weather occurred in Norway. Comparing model data with actual observations helps to verify the agreement between model prediction and ground based measurements. 
\\
AROME-MetCoOp was operational from March 2014 until November 2016, when it was replaced with an ensemble prediction system (EPS) based on AROME-MetCoOp.
MEPS is used as weather forecast at the Norwegian Meteorological Institute, the Swedish Meteorological and Hydrological Institute (SMHI) and the Finnish Meteorological Institute (FMI), \citep{muller_arome-metcoop:_2017, koltzow_metcoop_2017}.
%%%%%%%%%%%%%%%%%%%%%%%%%%%%%%%%%%%%%%%%%%%%%%%%%%%%%%%%%%%%%%%%%%%%%%%%%%
%%%%%%%%%%%%%%%%%%%%%%%%%%%%%%%%%%%%%%%%%%%%%%%%%%%%%%%%%%%%%%%%%%%%%%%%%%
%%%%%%%%% MEPS %%%%%%%%%%%%%%
\section{AROME - MetCoOP}
\label{sec:MEPS}
In principle, MEPS is a short-term weather forecast of \SI{66}{\hour} with 10 ensemble member and a horizontal resolution of \SI{2.5}{\km} and 65 vertical levels. In the following section, the model configuration will be briefly explained. % as well as the motivation for the study case presented.
\\
The orange frame in \Cref{fig:site} shows the MEPS model domain as it was operational for December 2016. It covers the Nordic Countries including open water such as the Atlantic Ocean, the North and the Baltic Sea.  
\\
The centre of the model is approximately at \ang{63.5}\,N, \ang{15}\,E. 
The horizontal grid points are projected on a Lambert projection to receive the same area size of each grid cell. 
The outer, parent grid is the ECMWF-IFS model (European Centre for Medium-Range Weather Forecasts Integrated Forecasting System) with a horizontal resolution of \SI{9}{\km} \citep{homleid_verification_2016}. 
\\
Vertical hybrid coordinates are terrain-following and are mass-based, \citep{muller_arome-metcoop:_2017}. How the vertical hybrid coordinates where transformed into layer thickness or height is described in \Cref{sec:layer_thickness}. Furthermore, MEPS underlies non-hydrostatic dynamics, \cite{metcoop_wiki_description_2017}.
\\
The representation of snow is covered by a modification of the three-class ice parametrization (ICE3) scheme. Where liquid-phase processes are separated from slow ice-phase processes. To model the snow cover an one-layer atmosphere model scheme is implemented. This includes three variables such as: snow water equivalent (SWE), snow density, and snow albedo \citep{muller_arome-metcoop:_2017}.
\\
The MEPS forecasting system consist of 1+9 members where each of the perturbed members perform an initialization of \SI{66}{\hour} at \SIlist{00;06;12;18}{\UTC} \citep{metcoop_wiki_description_2017}. The ECMWF-IFS forecasts are used \SI{6}{\hour} prior to the actual cycle in MEPS. As synoptic observations are included in the model the snow-depth predictions underlay a special performance. Observations of snow-depth are only available at \SIlist{06;18}{\UTC}, therefore is the snow analysis only performed twice daily \citep{muller_arome-metcoop:_2017, homleid_verification_2016}. 
%\\
%For more detailed information to the model the reader is referred to \cite{muller_arome-metcoop:_2017} and \cite{wiki_description_2017} and its references therein.

\section{MèsoNH}
\cite{meteo_france_meso-nh_2009}; 
\cite{seity_arome-france_2010}
\section{SURFEX}
\cite{masson_surfexv7.2_2013}
\section{ICE3 scheme}
\cite{pinty_mixed-phased_1998}
