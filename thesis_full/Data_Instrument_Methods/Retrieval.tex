% !TeX spellcheck = en_GB
\chapter{Optimal Estimation Retrieval Algorithm} %Section - 1.2
\label{ch:retrieval}
Since 2006, with the launch of CloudSats Cloud Profiling Radar (CPR) a global estimation of snowfall can be done. Several studies, such as \cite{kulie_utilizing_2009} have shown that estimated snowfall values depend heavily upon assumed snowflake microphysical properties.
%\textcolor{red}{Steve had some insertions but no comments.}
%depending on the retrieval assumption, snowfall estimation can give the same values for different a priori guess, e.g. snowflake microphysical properties. 
%Introducing information from snow microphysics will reduce the non-uniqueness in optimal estimation radar retrieval schemes. 
%This can be done by using the temperature at a lower level as a priori as the CloudSat retrieval does.  
%\\
\cite{wood_microphysical_2015} showed that a refinement of the CloudSat snowfall retrieval algorithm can be done by using snowflake models. 
%\textcolor{red}{That was Steves comment which is exactly what I wrote? "
This study was based on data from the Canadian CloudSat-CALIPSO Validation Project \citep[C3VP,][]{hudak_canadian_2006}, where they concentrated on cold season clouds and precipitation.
%"} 
\\
\noindent In an attempt to reduce the non-uniquness of the problem, \cite{wood_microphysical_2015} used the a priori knowledge of snowfall microphysics and temperature (from ground-based observations) to refine the forward-model assumptions for the CloudSat snowfall retrieval scheme. 
%\textcolor{red}{Steve: "They also used a temperature to introduce a priori information on particle size into the retrieval scheme. " Me: They did? So by introducing a temp they estimate the particle size?}
Results from this scheme showed a good agreement with reported values observed at meteorological measurement sites. \\
Model estimates have proven, how useful the estimation retrieval can be to verify ground-based radar snowfall measurements \citep{norin_intercomparison_2015}.
Although the retrieval has obviously been improved the estimation algorithm, can still lead to uncertainties in the retrievals of up to \SIrange{140}{200}{\percent} \citep{wood_estimation_2011}. 
\\
% \noindent The snowfall retrieval assumes an exponential particle size distribution (PSD)
% \begin{align}
% 	N(D) = N_{0} exp(-\lambda D).
% 	\label{eq:PSD}
% \end{align} 
% $\lambda$ represents here the PSD slope parameter and $N_{0}$ the number density. 
% %$D$ is the particle maximum dimension evaluated from the MASC. 
% \\
% The optimal estimation technique is based on Gaussian statistics. Minimizing the scalar cost function, $\Phi$ for the snowfall properties, $x$ by; 
% \begin{equation}
% \begin{split}
% \Phi(x,y,a) = &(y- F(x))^T \mathbf{S}_y^{-1} (\mathbf{y}-F(\mathbf{x})) \\
% &+(x-a)^T \mathbf{S}_{a}^{-1} (x-a)
% \end{split} \label{eq:scalar_cost_fct}
% \end{equation}
% where, $x$, vector of retrieved snowfall properties (slope parameter and number density); $y$, vector of observation (MRR reflectivity) ; $a$, vector of the a priori guess (temperature dependent); $F$, forward model; $\mathbf{S}_a$, a priori error covariance matrix; $\mathbf{S}_y$, measurement error covariance matrix.
% \\
\cite{cooper_variational_2017} developed a technique to combine MRR, MASC, and PiP information into a common retrieval framework. Specifically, estimates of snowflake microphysical properties from the MRR are used as the a priori term in the optimal-estimation retrieval scheme. The usage of either MASC\,/\,PiP or MRR fall-speed can show which a priori guess in the retrieval gives the more accurate retrieved snowfall rate at the ground. \\
The difference between the retrieval and the snow gauge observations was \SI{-18}{\percent} when applied to data from Barrow, Alaska.\\
\cite{cooper_variational_2017} also showed that the retrieval is sensitive to habit and fall speed. The installation of a MRR, MASC, and PIP should help to adjust the particle models for graupels and rimed particles which are often observed at Haukeliseter. 

\section{Present snow}\label{sec:pre_snow}
%\item surface temperature \SI{< 2}{\celsius}
%\item follows \Cref{fig:MRR_sfcT}
% %%% image surface temperature and MRR %%%%%%%%%%%%%%%%%%%%%%%%%%%%%%%%%%%%%
% !TeX spellcheck = en_GB
\begin{figure}[b!]
	\centering
	%    \begin{subfigure}[b]{\textwidth}
	\includegraphics[width=0.9\textwidth]{./fig_MRR/MRR_sfcT_20161225}
	%	\end{subfigure}
	\caption{A priori temperature dependence within the optimal estimation retrieval for an all day precipitation event on \SI{25}{\dec}. The upper panel shows the surface a priori guess, $T_{ap}$, measured at the Haukeliseter site. The lower panel presents the reflectivity measure by the MRR. Additionally, the purple frame indicates the time, where the MRR reflectivity was larger than \SI{-10}{\dB Z} and surface temperatures less than \SI{2}{\celsius}. }\label{fig:MRR_sfcT}
\end{figure}
%%%%%%%%%%%%%%%%%%%%%%%%%%%%%%%%%%%%%%%%%%%%%%%%%%%%%%%%%%%%%%%%%%%%%%%%%%
To achieve vertical profiles of snowfall from MRR different steps and assumptions are done in the here presented snowfall retrieval. From one of the lower levels, the snowfall rate at the surface can be estimated. The retrieval is only performed for profiles, which are likely to have observed snow, where most retrievals use rain. In previous studies relationships between reflectivity and snowfall have been developed. Even if the distribution is known, different crystal shapes led to different results. Snow densities vary significantly from storm to storm, where small particles are still Rayleigh scattered, and larger particles non-Rayleigh scattered \citep{gunn_microwave_1954}. 
\\
To obtain the likelihood of present snow a reflectivity threshold of \SI{- 15}{\decibel Z} is used. This threshold is similar to the one used in \cite{wood_level_2013}, where it states that light liquid precipitation is related to \SI{- 10}{\decibel Z} \textcolor{red}{citation here}. \cite{wood_estimation_2011} compared the reflectivities in the lowest bin and adjacent bin and found, that the reflectivity above \SI{-15}{\dB Z} are not influenced by ground clutter.
\newline
The Haukeliseter measurement site is equipped with a weather mast, measuring the air temperature every minute at two meter height (compare \Cref{fig:MRR_sfcT}, upper panel). 
%
Since the MRR measures above \SI{300}{\metre} and only temperature measurements at the surface exists, a priori temperature ($T_{a}$) is assumed to be similar to the observed near-surface air temperature. Using a moist adiabatic lapse rate of $dT/dz = \SI{5}{\kelvin\per\km}$ gives $T_{a}$ in each layer. Temperature measurements above \SI{2}{\celsius} \textcolor{red}{read up reference Liu (2008), \cite{wood_level_2013}} are neglected and assumed to be present liquid precipitation. \\
The purple line in \Cref{fig:MRR_sfcT} represents the time frame during \SI{25}{\dec}, where the MRR measures reflectivity is less than \SI{- 15}{\decibel Z}, and a priori temperature passes the \SI{2}{\celsius} limit at the surface.  


\section{Size distribution} \label{sec:size_dist}
To determine the snowfall rate at the surface an exponential particle size distribution (PSD) is used. 
\begin{align}
N(D) & = N_{0} \exp\left(-\lambda D\right) \qquad [ \SI{}{\per\cubic\metre\per\mm} ] \label{eq:num_dens}
\end{align}
where $\lambda$ represents the PSD slope parameter and $N_{0}$ the number density. $D$ is the particle maximum dimension evaluated from the 2D-scattering model for branched 6-arm spatial particles with porosities for reflectivity measurements at \SI{24}{\giga\Hz} (see \Cref{app:scat_scheme}).
%exponential particle size distribution (PSD) \\
%	$$ \qquad \text{$D$ from B6 scattering model}$$
%	$$\lambda = 10^{perturbuation}$$ %\\
%   %$$N_0 = 10^{perturbation}$$
%The slope parameter and the number density in \Cref{eq:num_dens} varies with height with the help of the following logarithmic assumptions \textcolor{red}{In vsnow_hmrr.pro: DEFINE A PRIORI PSD INFO (LOG FORM) THROUGH WOOD TEMPERATURE PARAMETERIZATION}.
\\
\cite{wood_estimation_2011} showed a linear fit between $\log(\lambda)$ and the a priori temperature, respectively $\log(N_0)$ and the a priori temperature.
Since $T_{a}$ varies with a moist adiabatic lapse in each layer bin the slope parameter and the number density in  \Cref{eq:num_dens} are changing too. Using $T_{a}$ in \SI{}{\celsius} for each layer bin the following logarithmic assumption is used,
\begin{align}
\log(\lambda) & = -0.03053 \cdot T_{ap} - 0.08258  \label{eq:lambda} \qquad [ \log(\SI{}{\per\mm}) ]\\
\log(N_0) & = -0.07193 \cdot T_{ap} +2.665  \qquad [ \log(\SI{}{\per\cubic\metre\per\mm})]
\label{eq:N0}
\end{align}
% \item \citep{wood_level_2013}: state is described by the exponential size distribution parameters $N_0$ and $\lambda$. Values for $N_0$ may range over several order of magnitude, so $log(N_0$) is retrieved instead. $log(\lambda$) is retreived since they are less skewed and $\lambda$ was strongly non-Gaussian. \textcolor{red}{where is this equations from?}\\
Since the results from C3VP were similar to other observations and $N_0$ ranges over several orders of magnitude as well as $\lambda$ was for the snow events non-Gaussian, for both the log-transformed values are created to achieve the state vector $\mathbf{x}$ of unknown micropyhsical properties.
\begin{align}
\mathbf{x} & = \begin{bmatrix}
log(\lambda)_0 	\\
\vdots 			\\
log(\lambda)_{\text{nlayer}} 	\\
log(N_0)_0		\\
\vdots			\\
log(N_0)_{\text{nlayer}}		
\end{bmatrix} \qquad \text{nlayer} = 14
\label{eq:snow_prop}
\end{align}

\section{Snowfall retrieval scheme}\label{sec:ret_scheme}
The optimal estimation method is based on Gaussian statistics. Minimizing the scalar cost function, $\Phi$ for the snowfall properties, $\mathbf{x}$. The cost function weights the difference between the observed reflectivity and the simulated measurements as well as the difference between the estimated and a priori guess. A forward model $F(\mathbf{x})$ relates unknown snowfall parameters $\mathbf{x}$ to radar observations $y$ and approximates the true physical state between them \citep{wood_estimating_2014,cooper_variational_2017}. \\
Scalar cost function:
\begin{equation}
\begin{split}
\Phi(\mathbf{x},y,a) = & (y- F(\mathbf{x}))^T \mathbf{S}_y^{-1} 			(y-F(\mathbf{x})) \\
&+(\mathbf{x}-a)^T \mathbf{S}_{a}^{-1} (\mathbf{x}-a)
\end{split} \label{eq:scalar_cost_fct}
\end{equation}
where, $\mathbf{x}$, vector of retrieved snowfall properties (\Cref{eq:snow_prop}); $y$, vector of observation (MRR reflectivity) ; $a$, vector of the a priori guess (temperature dependent); $F$, forward model (\Cref{eq:forward_model}; $\mathbf{S}_a$, a priori error covariance matrix; $\mathbf{S}_y$, measurement error covariance matrix.
%
% a priori: $$
% 		\mathbf{S}_a = \begin{bmatrix}
%        0.133 	& 0 	& \dots & 0 		& 0         \\%[0.3em]
%        0		& 0.133 & \dots	& 0 		& 0		 \\
%        \vdots 	& \vdots& \vdots& \ddots 	& \vdots \\
%        0		& 0		& \dots	& 0.95 		& 0 \\
%        0 		& 0		& \dots & 0 		& 0.95
%      \end{bmatrix} $$
\\
$\mathbf{S}_a$ links the uncertainties of the PSD information and the surface temperature differences. The diagonal matrix elements in $\mathbf{S}_a$ are equal to \numlist{0.133;0.95} for the particle slope parameter and the number density, respectively as in Eq. 7.35 and 7.36 in \cite{wood_estimation_2011}. \\
% $$\mathbf{S}_y = \begin{bmatrix}
%        6.25 	& 0 	& \dots & 0 		& 0         \\%[0.3em]
%        0		& 6.25 & \dots	& 0 		& 0		 \\
%        \vdots 	& \vdots& \vdots& \ddots 	& \vdots \\
%        0		& 0		& \dots	& 6.25		& 0 \\
%        0 		& 0		& \dots & 0 		& 6.25
%      \end{bmatrix} $$ 
$\mathbf{S}_y$ characterises the the uncertainties associated with the measurements and the error in the forward model. This study uses for the diagonal matrix elements $2.5^2$ \textcolor{red}{UNIT!? based on the study from CITATION. BECAUSE.}
\\
% sensitivity matrix: K \\
% 	$$\pm perturbation = \hat{x} \cdot \left(1 \pm \frac{0.2}{100}\right)$$ \\
% 	perform Forward model \\
% 	matrix of Kernel functions, Jacobian of forward model with respect to the state vector 
% 	$$fn_{i,j} = \frac{y_{max_j} - y_{min_j}}{+ perturbation_i - (- perturbation_i)}$$
% 	$$\mathbf{K}(i,j) = \begin{bmatrix}
% 		fn_{0,0} 	& fn_{1,0} 		& \ldots & fn_{m-1,0} \\
%         fn_{0,1} 	& fn_{1,1}	 	& \ldots & fn_{m-1,1} \\
%         \vdots	 	& \vdots		& \ddots & \vdots 	\\
%         fn_{0,n-1} 	& fn_{1,n-1}	& \ldots & fn_{m-1,n-1}
% 	\end{bmatrix}$$
%
% covariance of solution $\hat{x}$ at convergence
%    $$$$
\textcolor{red}{I don't understand the next steps and if it is still the same $\mathbf{x}$?!} 
\\
At convergence is the error covariance of the retrieved state vector $\mathbf{S}_x$
\begin{align}
\mathbf{S}_x & = \left( \mathbf{S}_a^{-1} + \mathbf{K}^T \mathbf{S}_y^{-1} \mathbf{K} \right)^{-1}
\end{align}
which follows for $\mathbf{x}$
\begin{align}
\mathbf{x} & = \underbrace{\left( \mathbf{S}_a^{-1} + \mathbf{K}^T \mathbf{S}_y^{-1} \mathbf{K} \right)^{-1} }_\text{$\mathbf{S}_x$} \left( \mathbf{S}_a^{-1} a + \mathbf{K}^T \mathbf{S}_y^{-1} \left(y - F(\mathbf{x}) + \mathbf{K} \mathbf{x} \right)  \right)
\end{align}
%\textcolor{red}{\cite{wood_level_2013}: 'As a diagnostic test of the results, a $\chi^2$ statistic is calculated using the retrieved state vector.}
The Jacobian matrix, $\mathbf{K}$, represents the sensitivity matrix of the perturbed result of the forward model. The true state $\mathbf{x}$ is perturbed by \SI{\pm 0.2}{\percent} and thus $\mathbf{K}$ represents the relation between simulated values to the true state.  \textcolor{red}{Why are we perturbing it?}
\begin{align}
\mathbf{K}(\mathbf{x}) & = \frac{\partial y}{\partial \mathbf{x}} =
\begin{bmatrix}
\frac{\partial y_0}{\partial \mathbf{x}_0} & 
\frac{\partial y_0}{\partial \mathbf{x}_1}  & 
\ldots & 
\frac{\partial y_0}{\partial \mathbf{x}_{2\,\text{x\,nlayer}}} \\[0.3em]
\vdots & \vdots & \ddots & \vdots \\[0.3em]
\frac{\partial y_{\text{nlayer}}}{\partial \mathbf{x}_0} &
\frac{\partial y_{\text{nlayer}}}{\partial \mathbf{x}_1} &
\ldots &
\frac{\partial y_{\text{nlayer}}}{\partial \mathbf{x}_{2\,\text{x\,nlayer}}}
\end{bmatrix} % Eq. 3.6 in \cite{wood_estimation_2011}
\label{eq:Kmatrix}
\end{align}
Usually, is $\mathbf{K}$ not diagonal \citep{wood_estimation_2011}, hence it is a mix of the true state and the a priori guess and the influence from them can be estimated. 
The closer $\mathbf{K}$ is diagonal, the more is $\mathbf{x}$ determined by the real observed and a priori values. If the limit of the partial derivative is close to unity, the retrieved value $\mathbf{x}$ is its true state \citep{wood_estimation_2011}. \\
\\
\textcolor{red}{mmmh? What exactly are we doing here?}
Test the if convergent:
\begin{align}
\hat{x} & = \left( \mathbf{x} - F(\mathbf{x}) \right)^T \mathbf{S}_x^{-1} \left(\mathbf{x} - F(\mathbf{x}) \right)
\end{align}
only if $\hat{x}$ is smaller than \num{2} it is a 'good' retrieval. 
\\
To test the result of $\mathbf{x}$ a $\chi^2$ test is performed at the convergence of $\mathbf{S}_x$.
%
\begin{equation}
\begin{split}
\chi^2 = & \left( F(\mathbf{x}) - y  \right) \mathbf{S}_y^{-1} \left( F(\mathbf{x}) - y \right)^T \\
& + \left( \mathbf{x} - a \right)^T \mathbf{S}_a^{-1} \left( \mathbf{x} - a\right). 
\end{split}
\label{eq:chi}
\end{equation}
%
The first term in \Cref{eq:chi} measures the part of $\chi^2$ related to the noise of the forward model, and the second part the relation to the state vector. Thus the second term describes the accuracy of the quantities within the reflectivity and temperature measurements \citep{rodgers_inverse_2000}.
% error contribution 
Furthermore, are the error contribution from the reflectivity measurement uncertainty, $\mathbf{S}_{y_\epsilon}$, and the uncertainty of the a priori values, $\mathbf{S}_{a_\epsilon}$ estimated. \textcolor{red}{Where comes this formulas from? Is the $\mathbf{D}_y$ the $\mathbf{K}_b$ in Eq. 15 in \cite{wood_characterization_2013}? How to interpret?}
%
\begin{align}
\mathbf{D}_y  & = \mathbf{S}_x \mathbf{K}^T \mathbf{S}_y^{-1} %\hspace{4cm} 
& \mathbf{S}_{y_\epsilon} & = \mathbf{D}_y \mathbf{S}_y \mathbf{D}_y^T \\
\mathbf{D}_a & = \mathbf{S}_x \mathbf{S}_a^{-1} %\hspace{4cm} 
& \mathbf{S}_{a_\epsilon} & = \mathbf{D}_a \mathbf{S}_a
\end{align}
The diagonal of $\mathbf{S}_{y_\epsilon}$ or $\mathbf{S}_{a_\epsilon}$ is a first estimate of the retrieval noise related to the observations \citep{rodgers_inverse_2000}. Uncertainties in $\mathbf{S}_{a_\epsilon}$ occur because of the variability in $T_a$ and similar for $\mathbf{S}_{y_\epsilon}$ due to measurement errors from the MRR.
\\
\newline
%
% fall speed assumption:
To achieve snowfall rates at the surface, the snow water content (SWC) has to be transformed into snowfall rates. The use of an assumed particle fall speed of $V = \SI{0.85}{\mPs}$ and the retrieved SWC (\Cref{eq:SWC}) gives the snow mass flux $J_{snow} = \text{SWC} \cdot V$ in [\SI{}{\kilogram\per\square\metre\per\second}]. \textcolor{red}{Why did we use this fall speed? Where does this assumption come from? Similar as \cite{cooper_variational_2017} Eq. 4?} 
% snow mass flux
To compare retrieved snow fall rates to double fence measurements and the forecast model output, the precipitation amount at the surface is calculated. 
\begin{align}
P = J_{snow} \times \num{e-3} \cdot \left(\SI{3600}{\second} \cdot24 \right) \qquad [\SI{}{\mm\per\day}]
\end{align}
The precipitation amount at the surface, presented in \Cref{ch:Res}, are taken to be equal to the values at the snow layer in \SI{800}{\metre}. The use of the values at \SI{800}{\metre} is due to the small increase of reflectivity (ground clutter) in the bottom layers and would follow more observed snow. 
%
\begin{itemize}
	\item \textcolor{red}{Not sure what to do with this!}
	\item The error from the retrieved state vector $\mathbf{x}$ is calculated
	$$\pm \mathbf{S}_{x_{err}} = \mathbf{x} \pm \mathbf{S}_x$$ 
	$$equiv_{err} = \frac{1}{2} \left( \frac{\abs{\text{SWC}(- \mathbf{S}_{x_{err}} ) - \text{SWC} (+ \mathbf{S}_{x_{err}}) }}{ \text{SWC}(- \mathbf{S}_{x_{err}})} + \frac{\abs{ \text{SWC}(- \mathbf{S}_{x_{err}}) - \text{SWC}(- \mathbf{S}_{x_{err}}) }}{ \text{SWC}(- \mathbf{S}_{x_{err}})} \right)$$
\end{itemize}


\section{Forward model}
Forward model defines a relationship between the radar observations and the retrieved state vector $\mathbf{x}$. The knowledge about the a priori parameters and related covariances, as well as $\mathbf{x}$, are used to minimize \Cref{eq:scalar_cost_fct}. The values of $\mathbf{x}$ are found by Newtonian iteration \cite[Eq. 5]{wood_estimating_2014}.
\newline
The snow water content in each layer is estimated from the knowledge of the snow particle mass-dimension relationship in \Cref{app:scat_scheme}, and a given slope parameter and number density (\Cref{eq:snow_prop}). % Snow water content  in each layer 
\begin{align}
\text{SWC} & = \int_{D_{min}}^{D_{max}} m(D) N(D) dD \qquad [\SI{}{\gram\per\cubic\metre}] \label{eq:SWC}
\end{align}
To achieve a relationship between the reflectivity and the snowfall amount one needs to account for attenuation in the atmosphere. Using the previously calculated PSD (\Cref{eq:num_dens}) the singly-scattered non-attenuated reflectivity for Rayleigh approximation ($2\pi D/\lambda \ll 1$, $\lambda$: wavelength of incident radiation) is found \citep{lecuyer_estimation-based_2002,kulie_utilizing_2009,wood_microphysical_2015}. 
% singly-scattered non-attenuated reflectivity
\begin{align}
\eta_{bk} & = \int_{D_{min}}^{D_{max}} N(D) \sigma_{bk} dD \qquad [\SI{}{\per\metre}] \nonumber \\
Ze^{ss,na} & = \frac{\Lambda^4}{\left\| K_w \right\|^2 \pi^5} \eta_{bk} \qquad \textcolor{red}{[\SI{}{\mm^6\metre^{-3}}]} \label{eq:singleZ}
\end{align}
% $\ldots$ snow backscatter coefficient
where, $\left\| K_w \right\|^2$ is the complex refractive index of water and varies between \numlist{0.91;0.93} for wavelength between \SIlist{0.01;0.10}{\metre} and is independent of temperature. It also exists a  complex refractive index for ice $\left\| K_i \right\|^2$, which is \SI{0.18}{}. This is valid for a density of \SI{0.917}{\gram\per\cubic\cm} and is independent of temperature and of wavelength in the microwave region \citep{doviak_doppler_1993}. In this work  $\left\| K_w \right\|^2 = 0.93$ is chosen, \textcolor{red}{BECAUSE???}. $\sigma_{bk}$ represents the backscattering cross-section and $\Lambda$ is the wavelength OF the radar. 
\\
The singly-scattered reflectivity has to be corrected for attenuation in the layers above the actual layer. According to Beer's law is in a homogeneous medium one way transmission assumed. 
% account for attenuation
% homogeneous medium $\Rightarrow$ one way transmission, Beer's law
\begin{align}
\frac{I_{\lambda}}{I_{\lambda_0}} & = \exp \left[ - \int_0^s \beta_{ext} ds'\right] \label{eq:Beer}
\end{align}
where $s$ is the path length through the medium. The transmissivity $I_{\lambda}/I_{\lambda_0}$ is the relation of survived radiation through extinction in the atmosphere with the snow extinction coefficient $\beta_{ext}$
\begin{align}
\beta_{ext} & = \int_{D_{min}}^{D_{max}} N(D) \sigma_{ext} dD \qquad [\SI{}{\per\metre}] \label{eq:bext}
\end{align}
The extinction coefficient is the sum of absorption and scattering in the atmosphere followed from the extinction cross-section $\sigma_{ext} = \sigma_{abs} + \sigma_{scat}$, \citep{lohmann_introduction_2016,lamb_physics_2011}. \textcolor{red}{\cite[Eq. 12.1 and more][]{lohmann_introduction_2016}} \\
%singly-scattered attenuated reflectivity 
Following \Cref{eq:singleZ,eq:Beer,eq:bext} the singly-scattered attenuated reflectivity $Ze^{ss, a}$ is
\begin{align}
Ze^{ss, a} & = Ze^{ss, na} \cdot \frac{I_{\lambda}}{I_{\lambda_0}} \qquad \textcolor{red}{[\SI{}{\mm^6\metre^{-3}}]} \label{eq:sing_scatt_att_Z}
\end{align}
% multiply-scattered attenuated reflectivity 
% \cite{matrosov_s._y._influence_2009} found that in heavy snow conditions the multiply-scattered attenuated reflectivity lies between the singly-scattered attenuated and non-attenuated reflectivities \citep[compare Fig. 3.3 in][]{matrosov_s._y._influence_2009}. Therefore, is as in \cite{wood_level_2013} the multiply-scattered attenuated reflectivity, $Ze^ms$, approximated as geometric mean.
% \begin{align}
% 	Ze^ms & = \sqrt{Ze^{ss,na} \cdot Ze^{ss,a}} \qquad [\SI{}{\dB Z}]
% \end{align}
% $\Rightarrow$ simulated reflectivities from forward model 
That follows for the simulated reflectivity from the forward model, $F(\mathbf{x})$ in \Cref{eq:scalar_cost_fct}, 
\begin{align}
F(\mathbf{x}) & = \begin{bmatrix} 
Ze^{ss,a}_1 \\
\vdots \\
Ze^{ss,a}_{nlayer}
\end{bmatrix} \qquad [\SI{}{\decibel Z}]. \label{eq:forward_model}
\end{align}


\begin{itemize}
	\item \textcolor{red}{not really using the following - What to do with it?}
	\item \textcolor{red}{total number concentration} \\
	$$n_{tot} = \int_{D_{min}}^{D_{max}} N(D) dD \qquad [\SI{}{\per\cubic\metre}]$$ 
	\item $\ldots$ \textcolor{red}{??? }
	
\end{itemize}






