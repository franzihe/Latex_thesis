% !TeX spellcheck = en_GB
% %%% ***************** CHAPTER RESULTS ***************** %%%
\chapter{Results and Discussion} \label{ch:Res}
% \textcolor{red}{\textbf{The overleaf file from the Results is here:} \\ \url{https://www.overleaf.com/15070250fhqztvnygzmn}} 
% \\V
%
In this chapter the results of the surface observation, the optimal estimation retrieval and the regional mesoscale forecast model are presented. On the basis of the methodology described in \Cref{ch:Methods} it should be evaluated if a regional mesoscale forecast model prognoses the same synoptical patterns as observed at the measurement site. Also, vertical SWC forecasted by MEPS is being verified with the retrieved SWC at Haukeliseter. 



%%%%%%%%%%%%%%%%%%%%%%%%%%%%%%%%%%%%%%%%%%%%%%%%%%%%%%%%%%%%%%%%%%%%%%%%%
%%%%%%%% surface %%%%%%%%%%%%%%
% !TeX spellcheck = en_GB
% %%% ***************** CHAPTER RESULTS ***************** %%%
\chapter{Results and Discussion} \label{ch:Res}
% \textcolor{red}{\textbf{The overleaf file from the Results is here:} \\ \url{https://www.overleaf.com/15070250fhqztvnygzmn}} 
% \\V
%
In this chapter the results of the surface observation, the optimal estimation retrieval and the regional mesoscale forecast model are presented. On the basis of the methodology described in \Cref{ch:Methods} it should be evaluated if a regional mesoscale forecast model predicts the same synoptic patterns as observed at the measurement site. Also, vertical SWC forecasted by MEPS is being verified with the retrieved vertical SWC at Haukeliseter. Attention should be paid to the fact, that this study study is very unique of its kind. As far as the author has knowledge was no approach done to verify a vertical regional forecast model with the help of vertical observation measurements. 

%%%%%%%%%%%%%%%%%%%%%%%%%%%%%%%%%%%%%%%%%%%%%%%%%%%%%%%%%%%%%%%%%%%%%%%%%%
%%%%%%%%% Observation agreement %%%%%%%%%%%%%%
\section{Comparison of surface observations} 
%%%%%%% image scatter obs ret %%%%%%%%%%%%%%%%
\begin{figure}[t]
	\centering
	\begin{subfigure}[b]{0.38\textwidth}
		\includegraphics[trim={0.cm 0.cm 13cm 0cm},clip,
		width=\textwidth]{./fig_obs_ret/obs_ret_20161220_26_00}
		\caption{}\label{fig:res:obs_ret_scatter}
	\end{subfigure}
	%
	\begin{subfigure}[b]{0.59\textwidth}
		\includegraphics[trim={0.cm 0.cm 0cm 0cm},clip,
		width=\textwidth]{./fig_obs_ret/diff_20161220_26_00}
		\caption{}\label{fig:res:diff_ret_scatter}
	\end{subfigure}
	% label
	%    \begin{subfigure}[t]{0.18\textwidth}
	%   	\includegraphics[trim={25.cm 13.cm 0cm 1.3cm},clip,
	%  width=\textwidth]{./fig_obs_ret/obs_ret_20161220_26_00}
	% \end{subfigure}
	\caption{\protect\subref{fig:res:obs_ret_scatter}: Surface precipitation amount comparison between the double fence observations and the retrieved surface accumulation of precipitation for \SI{48}{\hour}. In black the linear correlation between the double fence observations and retrieved surface snow. 
    \protect\subref{fig:res:diff_ret_scatter}: Difference between the retrieved and the observed accumulation by the double fence. The colours represent the different starting days at \SI{0}{\UTC} for the \SI{48}{\hour} accumulation.}\label{fig:res:obs_ret}
\end{figure}
%%%%%%%%%%%%%%%%%%%%%%%%%%%%%%%%%%%%%%%%%%%%%%
To be able to compare the vertical predicted snow water content with the retrieved snow water content a verification of the surface accumulation is made. If the retrieved surface accumulation is confident in comparison to the double fence measurement, then the vertical measurements can be trusted.
\\
The correlation in \Cref{fig:res:obs_ret_scatter} demonstrates a good agreement between the \SI{48}{\hour} accumulation measured by the double fence and the retrieved surface accumulation.
The black line in \Cref{fig:res:obs_ret_scatter} presents a linear correlation with a regression coefficient of R = \num{0.97}. 
In general, the retrieved surface snowfall accumulation is underestimated when compared to the double fence measurements, but not to a large degree. 
\\
\Cref{fig:res:diff_ret_scatter} shows the difference between retrieved accumulation and observed accumulation by the double fence. For the time period \num{20} to \SI{24}{\dec}, \Cref{fig:res:diff_ret_scatter} indicates an underestimation of retrieved snow accumulation of less than \SI{-5}{\mm} for the first \SI{24}{\hour}. 
Snow accumulation calculated on \SI{23}{\dec} at \SI{0}{\UTC} show after \SI{24}{\hour} an underestimation by the retrieval of up to \SI{-6.5}{\mm}. Larger underestimation after \SI{43}{\hour} is related to the observation of liquid precipitation on \SI{25}{\dec} between \SIrange{12}{21}{\UTC} for accumulations on \SI{24}{\dec}. On \SI{25}{\dec} no fair comparison to the double fence measurement can be performed after \SI{12}{\UTC} because of the neglection of liquid precipitation when temperatures exceed \SI{2}{\celsius}.
\\
%The mean absolute difference of all days is \SI{2.06}{\mm} (excluding values on \SI{25}{\dec} after \SI{12}{\UTC} and on \SI{26}{\dec} after \SI{17}{\UTC} because of attenuation at the MRR). 
For a \SI{12}{\hour} accumulation follows for the Christmas storm (\num{20} to \SI{26}{\dec}) an average error of \SI{85.5}{\percent} (\Cref{tab:res:ret_error}). For longer, \SI{24}{\hour} accumulation decreases the average error to be \SI{- 4.7}{\percent} (excluding values on \SI{25}{\dec} after \SI{12}{\UTC} and on \SI{26}{\dec} after \SI{17}{\UTC} because of attenuation at the MRR). The daily surface snowfall accumulation difference between retrieval and observation in \Cref{tab:res:ret_error} show almost always a well agreement to the boundary condition of the double fence. The only well pronounced mismatch is seen non \SI{21}{\dec}, where it measures much more than the double fence gauge (+\SI{435.8}{\percent}). 
\\
Similar to this study, \citet{cooper_variational_2017} used a CloudSat snow particle model, PSD and fall speed from MASC observations for five snow events at Barrow, Alaska. The comparison to the weather station revealed an difference between National Weather Service observations and retrieved accumulations of \SI{- 18}{\percent} for all five snow events.
%%%%%%% table error sfc acc ret obs %%%%%%%%%%%%%%%%
\begin{table}[t!]
	\begin{center}
		\caption{Comparison of observed (obs.) and retrieved (ret.) snowfall amounts for the Christmas storm 2016. Difference refers to the difference of the retrieved and observed snow accumulation after \SI{12}{\hour} and \SI{24}{\hour}. The average difference is the value over all six/four days. Excluding values after \SI{12}{\UTC} on \SI{25}{\dec} and after \SI{17}{\UTC} on \SI{26}{\dec}.}\label{tab:res:ret_error}
		\begin{tabular}{c||r|r|c|c||c|c|c|c}
			\hline \hline
			& \multicolumn{4}{c||}{\textbf{\SI{12}{\hour} accumulation}} & \multicolumn{4}{c}{\textbf{\SI{24}{\hour} accumulation}}    \\ \cline{2-9}
            \textbf{Day} & \multicolumn{2}{c|}{\textbf{Snowfall}} & \textbf{Difference} & \textbf{Average} &  \multicolumn{2}{c|}{\textbf{Snowfall}} & \textbf{Difference} & \textbf{Average}  \\\cline{2-3} \cline{6-7}
            \textbf{in 2016} & \textbf{obs.} & \textbf{ret.} & & \textbf{difference} & \textbf{obs.} & \textbf{ret.} & & \textbf{difference} \\\cline{2-9}
			& \multicolumn{2}{c|}{[\SI{}{\mm}]} & [\SI{}{\percent}] & [\SI{}{\percent}] & \multicolumn{2}{c|}{[\SI{}{\mm}]} & [\SI{}{\percent}] & [\SI{}{\percent}] \\ \hline\hline
			\num{20} Dec & \num{0.1} &\num{0.0} & \num{-97.8} & & \num{0.1} & \num{0.0} & \num{- 97.8} &  \\\cline{1-9}
			\num{21} Dec & \num{0.7} & \num{3.8} & +\num{435.8} & \multirow{6}{*}{+\num{85.5}} & \num{17.1} & \num{16.6} & \num{-2.7} & \multirow{4}{*}{\num{-4.7}}   \\\cline{1-4}\cline{6-8}
			\num{22} Dec & \num{13.6}& \num{13.2} & \num{- 3.0} & & \num{25.6} &\num{25.1} & \num{-2.1} &   \\\cline{1-4}\cline{6-8}
			\num{23} Dec & \num{6.3} &\num{5.2} & \num{- 16.8} & & \num{23.3}& \num{19.8} & \num{-14.9} &   \\\cline{1-4}\cline{6-8}
			\num{24} Dec & \num{14.7} & \num{13.4} & \num{- 8.6} && \num{24.8} & \num{25.0} & +\num{0.8} &   \\\cline{1-4}\cline{6-9}
			\num{25} Dec &  \num{4.3} & -- & -- & & +\num{15.4} & -- & -- & \\\cline{1-4}\cline{6-9}
			\num{26} Dec & \num{8.8} & \num{10.6} & +\num{20.1} &  &  \num{25.1} &-- & -- &  \\\hline\hline
		\end{tabular}
	\end{center}
\end{table}
%%%%%%%%%%%%%%%%%%%%%%%%%%%%%%%%%%%%%%%%%%%%%%
\\
\Cref{tab:res:ret_error} shows the difference for each individual day and the average difference for six and 4 days, depending on the accumulation of \SI{12}{\hour} or \SI{24}{\hour}.
The choice of the correct PSD model, slope parameters and fall speed in the optimal estimation snowfall retrieval, shows a good agreement with the observations at Haukeliseter for the 2016 Christmas storm in contrast to the \SI{200}{\percent} difference when only using the CloudSat snowfall algorithm \Cref{sec:retrieval}. It indicates also that the non-uniqueness of snow accumulation is reduced, when using a combination of ground-based observations instead of only Ze-S relationships. 
%Of course, more storms should be investigated to find the exact correlation between the surface observations and the estimated accumulation to see if the deviation keeps as small for different snow patterns at Haukeliseter. 
During the 2016 Christmas storm the average error for \SI{24}{\hour} accumulation is almost similar to Barrow, Alaska. It turns out that there is no relation between high and low precipitation events since the differences vary. \citet{cooper_variational_2017} also showed different combinations of PSD assumptions and snow fall speed. For Barrow, best agreements between observations and retrieved snowfall were found by using the CloudSat particle model, slope parameters and snowfall speeds from the MASC. The here presented work (\Cref{sec:retrieval}) uses a particle model based on observed particle sizes during the entire winter 2016/2017, like in \Cref{fig:MASC}. \textcolor{red}{Add Discussion on different combination/parameters as discussed with Steve. }
\\
\\
On \num{20} and \SI{21}{\dec}, the difference error is large (\SI{-97.8}{\percent} and \SI{435.8}{\percent}, respectively). This is probably related to an observation of precipitation at the double fence, even though no precipitation was observed. The double fence observation might be related to some particles stirred up by wind into the orifice of the gauge. Since no manual observations are done at the Haukeliseter site, is it difficult to say if blowing snow occurred and might introduce additional errors. But from the vertical MRR reflectivity it can be seen, that precipitation was not observed on before \SI{21}{\dec} \SI{9}{\UTC}.
\\
Even though it is assumed that the double fence is the absolute correct measurement it still underlies some uncertainties. 
A better way to asses the accuracy of the retrieved surface snowfall accumulation could be to compare the results to measurements inside a bush gage. A bush gauge is a precipitation gauge surrounded by a large bush to create artificial calm winds to increase the catch ratio of frozen precipitation and is considered as the best available measurement for solid precipitation \citep{wolff_wmo_2018}. Unfortunately there are only two bush gauges in the world, and because of local limitations a double fence construction is developed as reference for the Solid Precipitation Measurement intercomparison study during \numrange{1986}{1993} \citep{goodison_wmo_1998}. Comparisons between bush gauge and double fence precipitation measurements have shown, that for wind speeds up to \SI{9}{\mPs} outside the fence, the double fence will measure up to \SI{10}{\percent} less precipitation. 
While wind speeds outside the double fence might reach \SI{20}{\mPs} show measurements inside a decrease to \SI{5}{\mPs}. \citet{wolff_wmo_2018} believes the underestimation of the double fence will not be more than \SI{20}{\percent} during frozen precipitation events with high wind speeds. 
\\
The low average difference value for \SI{24}{\hour} accumulation, in \Cref{tab:res:ret_error} during the Christmas 2016 event (\SI{-4.7}{\percent}) follows a much lower average difference between retrieved and observed surface accumulation than at Barrow (\SI{36}{\percent}) and therefore a very good agreement between observed and retrieved snow accumulation during \num{21} to \SI{24}{\dec}. In \Cref{sec:res:verticalSWC}, the vertical SWC will be compared to the forecasted MEPS values for the 2016 Christmas storm. Despite the condition that the double fence measurement is influenced by wind will the small average difference for \num{21} to \SI{24}{\dec} give confidence in the retrieved profiles of snow water content when comparing to the forecast, but it should be kept in mind that retrieved snow accumulation is underestimated and therefore may the vertical SWC be too low.
%%%%%%%%%%%%%%%%%%%%%%%%%%%%%%%%%%%%%%%%%%%%%%%%%%%%%%%%%%%%%%%%%%%%%%%%%%

%%%%%%%%%%%%%%%%%%%%%%%%%%%%%%%%%%%%%%%%%%%%%%%%%%%%%%%%%%%%%%%%%%%%%%%%%%
%%%%%%%%% Synoptic Phenomena observed? %%%%%%%%%%%%%%
\section{Observation and predictions of large scale weather phenomena at the surface}\label{sec:res:large_scale_sfc}
One of the main factors, that made the Christmas 2016 storm so interesting is the fact, that fronts passed over Norway during the six-day period. One aim of this work is to determine if large scale features were observed at the measurement site during the extreme event. 
\\
A comparison between the surface observations at Haukeliseter and the ECMWF analysis of the dynamic tropopause and geopotential thickness maps show that frontal passages occurred on three days during the Christmas storm (\Cref{fig:GeopJet} and \ref{fig:DynTropo}).  
%These frontal boundaries show in the surface observations and ensemble forecasts on \SIlist{23;25;26}{\dec}. 
A typical cyclone has a prevailing warm front and a faster moving cold front. As the storm gets more intense and the cold front rotates around the low-pressure centre and catches the warm front the cyclone will begin to occlude. Changes in pressures, temperature, wind direction and wind speed can occur. In some cases, an intensification of the precipitation can be observed as well.
\\
\Cref{fig:res:sfc_obs_meps} shows the different parameters forecasts initialised at \SI{00}{\UTC} for \SIlist{23;25;26}{\dec}, as well as the observations at the Haukeliseter measurement site in dash-red.
Typical pressure decreases and increases, as well as temperature increases, and wind changes are present on \SIlist{23;26}{\dec}, since these changes show in the surface observations in \Cref{fig:res:sfc_obs_meps}, it is assumed that frontal boundaries passed. The \SI{25}{\dec} shows only an increase of temperature leading to the assumption of a warm air passage in \Cref{fig:res:sfc_temp25}.
%%%%%%% image sfc obs  %%%%%%%%%%%%%%%%
\begin{figure}[H]
	\centering
	% sfc pressure
	\begin{subfigure}[b]{0.49\textwidth}
		\includegraphics[trim={0.cm 5.cm 0cm 0cm},clip,
		width=\textwidth]{./fig_sfc_pressure/20161223_00}
		\caption{}\label{fig:res:sfc_pres23}
	\end{subfigure}
	%
	\begin{subfigure}[b]{0.49\textwidth}
		\includegraphics[trim={0.cm 5.cm 0cm 0cm},clip,
		width=\textwidth]{./fig_sfc_pressure/20161225_00}
		\caption{}\label{fig:res:sfc_pres25}
	\end{subfigure}
	% sfc temp
	\begin{subfigure}[b]{0.49\textwidth}
		\includegraphics[trim={0.cm 5.cm 0cm 0cm},clip,
		width=\textwidth]{./fig_sfc_temp/20161223_00}
		\caption{}\label{fig:res:sfc_temp23}
	\end{subfigure}
	%
	\begin{subfigure}[b]{0.49\textwidth}
		\includegraphics[trim={0.cm 5.cm 0cm 0cm},clip,
		width=\textwidth]{./fig_sfc_temp/20161225_00}
		\caption{}\label{fig:res:sfc_temp25}
	\end{subfigure}
	% sfc wd
	\begin{subfigure}[b]{0.49\textwidth}
		\includegraphics[trim={0.cm 5.cm 0cm 0cm},clip,
		width=\textwidth]{./fig_sfc_wd/20161223_00}
		\caption{}\label{fig:res:sfc_wd23}
	\end{subfigure}
	%
	\begin{subfigure}[b]{0.49\textwidth}
		\includegraphics[trim={0.cm 5.cm 0cm 0cm},clip,
		width=\textwidth]{./fig_sfc_wd/20161225_00}
		\caption{}\label{fig:res:sfc_wd25}
	\end{subfigure}
	% sfc ws
	\begin{subfigure}[b]{0.49\textwidth}
		\includegraphics[trim={0.cm 5.cm 0cm 0cm},clip,
		width=\textwidth]{./fig_sfc_ws/20161223_00}
		\caption{}\label{fig:res:sfc_ws23}
	\end{subfigure}
	%
	\begin{subfigure}[b]{0.49\textwidth}
		\includegraphics[trim={0.cm 5.cm 0cm 0cm},clip,
		width=\textwidth]{./fig_sfc_ws/20161225_00}
		\caption{}\label{fig:res:sfc_ws25}
	\end{subfigure}
	% sfc precip
	\begin{subfigure}[b]{0.49\textwidth}
		\includegraphics[trim={0.cm 3.6cm 0cm 0cm},clip,
		width=\textwidth]{./fig_sfc_precip/20161223_00}
		\caption{}\label{fig:res:sfc_precip23}
	\end{subfigure}
	%
	\begin{subfigure}[b]{0.49\textwidth}
		\includegraphics[trim={0.cm 3.6cm 0cm 0cm},clip,
		width=\textwidth]{./fig_sfc_precip/20161225_00}
		\caption{}\label{fig:res:sfc_precip25}
	\end{subfigure}
	
	% label
	\begin{subfigure}[b]{\textwidth}
		\centering
		\includegraphics[trim={5.5cm 0cm 5.cm 17.2cm},clip,
		width=0.8\textwidth]{./fig_sfc_ws/20161225_00}
	\end{subfigure}
    \caption{\SI{48}{\hour} surface observations and ensemble forecasts initialised on the \SI{23}{\dec} at \SI{0}{\UTC} (left column, \protect\subref{fig:res:sfc_pres23}, \protect\subref{fig:res:sfc_temp23}, \protect\subref{fig:res:sfc_wd23}, \protect\subref{fig:res:sfc_ws23}, \protect\subref{fig:res:sfc_precip23}), and on \SI{25}{\dec} at \SI{0}{\UTC} (right column, \protect\subref{fig:res:sfc_pres25}, \protect\subref{fig:res:sfc_temp25}, \protect\subref{fig:res:sfc_wd25}, \protect\subref{fig:res:sfc_ws25}, \protect\subref{fig:res:sfc_precip25}) as well as \SI{26}{\dec} (\protect\subref{fig:res:sfc_pres26}, \protect\subref{fig:res:sfc_temp26}, \protect\subref{fig:res:sfc_wd26}, \protect\subref{fig:res:sfc_ws26}, \protect\subref{fig:res:sfc_precip26}). Line representation according to the label. Upper panel sea level pressure, second \SI{2}{\metre} air temperature, third and fourth \SI{10}{\metre} wind direction and speed, respectively, and lowest panel precipitation amount. }\label{fig:res:sfc_obs_meps}
	%
\end{figure}
\begin{figure}[H]\ContinuedFloat
	\centering
	% sfc pressure
	\begin{subfigure}[b]{0.49\textwidth}
		\includegraphics[trim={0.cm 5.cm 0cm 0cm},clip,
		width=\textwidth]{./fig_sfc_pressure/20161226_00}
		\caption{}\label{fig:res:sfc_pres26}
	\end{subfigure}
	
	% sfc temp
	\begin{subfigure}[b]{0.49\textwidth}
		\includegraphics[trim={0.cm 5.cm 0cm 0cm},clip,
		width=\textwidth]{./fig_sfc_temp/20161226_00}
		\caption{}\label{fig:res:sfc_temp26}
	\end{subfigure}
	
	% sfc wd
	\begin{subfigure}[b]{0.49\textwidth}
		\includegraphics[trim={0.cm 5.cm 0cm 0cm},clip,
		width=\textwidth]{./fig_sfc_wd/20161226_00}
		\caption{}\label{fig:res:sfc_wd26}
	\end{subfigure}
	
	% sfc ws
	\begin{subfigure}[b]{0.49\textwidth}
		\includegraphics[trim={0.cm 5.cm 0cm 0cm},clip,
		width=\textwidth]{./fig_sfc_ws/20161226_00}
		\caption{}\label{fig:res:sfc_ws26}
	\end{subfigure}
	
	% sfc precip
	\begin{subfigure}[b]{0.49\textwidth}
		\includegraphics[trim={0.cm 3.6cm 0cm 0cm},clip,
		width=\textwidth]{./fig_sfc_precip/20161226_00}
		\caption{}\label{fig:res:sfc_precip26}
	\end{subfigure}
	
	% label
	\begin{subfigure}[b]{\textwidth}
		\centering
		\includegraphics[trim={5.5cm 0cm 5.cm 17.2cm},clip,
		width=0.8\textwidth]{./fig_sfc_ws/20161225_00}
	\end{subfigure}
    \caption{\textit{(Continued from previous page.)} Initialisation \SI{26}{\dec} at \SI{0}{\UTC}.}
\end{figure}
%%%%%%%%%%%%%%%%%%%%%%%%%%%%%%%%%%%%%%%%%%%%%%
\pagebreak
\noindent
As described in \Cref{sec:largeScale} shows the ECMWF dynamic tropopause analysis map (\Cref{fig:DT23}) more ridging and therefore warmer air over Southern Norway on \SI{23}{\dec}. The low-pressure system approaches in the course of the day south-east of Iceland and hence stronger west to south west wind are associated with the cyclone (\Cref{fig:GP23}). The MEPS forecast, initialised on \SI{23}{\dec} at \SI{0}{\UTC} in \Cref{fig:res:sfc_pres23} follows the observations and shows the decrease in pressure after \SI{12}{\UTC} due to the passage of the occluded front with a constant pressure after the transition. Since warmer air is more advected to the north and the DT in \Cref{fig:DT23} shows a warm low-pressure core, an increase in temperature was observed and predicted at the measurement site (\Cref{fig:res:sfc_temp23}). 
\\
As the cyclone is advected to the north-east, closer into the Norwegian Sea, a wind change can be seen in the analysis map from ECMWF (\Cref{fig:GP23}). First west wind and later south-west wind was associated with the low-pressure system. The MEPS forecast and observations in \Cref{fig:res:sfc_wd23} and \subref{fig:res:sfc_ws23} indicate a wind change from west to south with a slight decrease in wind speed.
\\
On \SI{23}{\dec} was the passage of the occlusion also observed by an increase in precipitation. Before \SI{18}{\UTC} shows the surface accumulation light precipitation. During the passage of the occluded front increases the observed surface accumulation and is associated to continuous, heavy precipitation.
\\
\\
Similar patterns were seen for the passage of the occluded front on \SI{26}{\dec} in the ECMWF analysis \Cref{fig:DT26} and \ref{fig:GP26}. In this case the low-pressure system was located north of Morø og Romsdal in the Norwegian Sea. In the morning was the cyclone located east of Iceland and in the course of the day it got closer to the coast of Norway. Before landfall at \SI{16}{\UTC} indicates \Cref{fig:res:sfc_pres26} a pressure decrease. During the passage the sea level pressure reaches its lowest point of \SI{985}{\hPa} and increased afterwards during the dissipation of the Christmas storm. Pressure, temperature, and wind changes were already forecasted for initialisations on \SI{25}{\dec} (\Cref{fig:res:sfc_pres25}, \subref{fig:res:sfc_temp25}, \subref{fig:res:sfc_wd25}, \subref{fig:res:sfc_ws25}), only wind speed and precipitation seem not to agree with the observations at Haukeliseter.
\\
Since the cyclone was surrounded by colder air (south of the low-pressure system) first a drop and then an increase of temperature were observed and forecasted by MEPS. An indication of the passage is also seen in the \SI{10}{\metre} wind observations and forecasts. As the cyclone is east of Iceland with a westward large-scale surface wind (\Cref{fig:GP26_00} and \Cref{fig:GP26_12}), shows \Cref{fig:res:sfc_wd26} and \subref{fig:res:sfc_ws26} west wind with observed strength up to \SI{17.5}{\mPs}. During the passage changes the wind direction to north-west with higher wind speed which can be associated to the location of the low-pressure system and the closer surface isobars (\Cref{fig:DT26}). 
\\
The precipitation was continuing throughout the day, with light to moderate precipitation before the passage and heavy precipitation around \SI{16}{\UTC} followed by moderate to light precipitation. 
\\
%%%%%%% image liquid obs particle %%%%%%%%%%%%%%%%
\begin{figure}[t]
	\centering
	\includegraphics[%trim={45.cm 30.cm 13cm 35cm},clip,
	width=0.8\textwidth]{./MASC_obs/Masc_obs_liquid_2512}
	\caption{MASC images of falling water drops observed on \SI{25}{\dec} at \SI{17}{\UTC} from three different angles. Not all parts of the liquid sphere are equally illuminated.}\label{fig:res:obs_masc}
\end{figure}
%%%%%%%%%%%%%%%%%%%%%%%%%%%%%%%%%%%%%%%%%%%%%%
\\
While on \SIlist{23;25}{\dec} the precipitation was associated with a passage/landfall of an occluded front, was the \SI{25}{\dec} marked by the transition of a warm sector. The ECMWF analysis showed a ridging at the DT surface. The surface cyclone core is south east of Iceland in \Cref{fig:DT25} with two associated frontal boundaries. While the warm front is approaching the west coast, is the cold front north-west of Great Britain. The cold fronts tail moved into lower latitudes, following the slowdown of the cold front, leading to a stationary frontal boundary. Furthermore, the mid-latitude jet is aligned along the surface frontal boundaries (\Cref{fig:DT25_00}), while the Haukeliseter site is located below the left jet exit region. This leads to rising motion at the surface.
\\
Neither pressure nor wind observations and forecasts indicate the passage of any frontal boundary. The only indication of the transition is seen in the increase of temperature at \SI{11}{\UTC} until \SI{21}{\UTC} (\Cref{fig:res:sfc_temp25}). In \Cref{fig:res:sfc_wd25} a small wind change was observed by the wind mast at \SI{10}{\UTC}. This development from west to north-west was not forecasted by MEPS, it rather estimated strong west winds.
\\
In \Cref{fig:res:obs_masc} are the surface observations from the MASC during the passage of the warm sector. Without the images taken around \SI{17}{\UTC} it would not be possible to verify that liquid precipitation occurred at the Haukeliseter site. Together with the increase in surface temperature in \Cref{fig:res:sfc_temp25} it is qualified that at least the warm sector of the low-pressure system appeared at the measurement site.
\textcolor{red}{Should I include the DIANA analysis maps? But I dont know what the meteorologists are using to produce them? ECMWF?}
%
%%%%%%% image scatter obs ret %%%%%%%%%%%%%%%%
\begin{figure}[t!]
	\centering
	% sfc pressure
	\begin{subfigure}[b]{0.49\textwidth}
		\includegraphics[trim={0.cm 0cm 12.5cm 0cm},clip,
		width=\textwidth]{./fig_sfc_pressure/obs_model_20161221_23_00}
		\caption{}\label{fig:scat:pres2123}
	\end{subfigure}
	%
	\begin{subfigure}[b]{0.49\textwidth}
		\includegraphics[trim={0.cm 0cm 12.5cm 0cm},clip,
		width=\textwidth]{./fig_sfc_pressure/obs_model_20161224_26_00}
		\caption{}\label{fig:scat:pres2426}
	\end{subfigure}
    \caption{\SI{48}{\hour} scatter plots for surface observations and ensemble forecasts initialised for \SIrange{21}{23}{\dec} (left column, \protect\subref{fig:scat:pres2123}, \protect\subref{fig:scat:temp2123}, \protect\subref{fig:scat:wd2123}, \protect\subref{fig:scat:ws2123}, \protect\subref{fig:scat:precip2123}) and  for \SIrange{24}{26}{\dec} (right column, \protect\subref{fig:scat:pres2426}, \protect\subref{fig:scat:temp2426}, \protect\subref{fig:scat:wd2426}, \protect\subref{fig:scat:ws2426}, \protect\subref{fig:scat:precip2426}). Sea level pressure.}\label{fig:scat:obs_meps}
\end{figure}
\begin{figure}\ContinuedFloat
	\centering
    %     % sfc temp
	\begin{subfigure}[b]{0.49\textwidth}
		\includegraphics[trim={0.cm 0cm 12.5cm 0cm},clip,
		width=\textwidth]{./fig_sfc_temp/obs_model_20161221_23_00}
		\caption{}\label{fig:scat:temp2123}
	\end{subfigure}
	%
	\begin{subfigure}[b]{0.49\textwidth}
		\includegraphics[trim={0.cm 0cm 12.5cm 0cm},clip,
		width=\textwidth]{./fig_sfc_temp/obs_model_20161224_26_00}
		\caption{}\label{fig:scat:temp2426}
	\end{subfigure}
% 	
% 	% label
% 	\begin{subfigure}[b]{0.49\textwidth}
% 		\centering
% 		\includegraphics[trim={25.cm 15.5cm 0cm 3.6cm},clip,
% 		width=0.8\textwidth]{./fig_sfc_temp/obs_model_20161221_23_00}
% 	\end{subfigure}
% 	\begin{subfigure}[b]{0.49\textwidth}
% 		\centering
% 		\includegraphics[trim={25.cm 15.5cm 0cm 3.6cm},clip,
% 		width=0.8\textwidth]{./fig_sfc_temp/obs_model_20161224_26_00}
% 	\end{subfigure}
% \end{figure}
% \begin{figure}\ContinuedFloat
	% sfc wd
	\begin{subfigure}[b]{0.49\textwidth}
		\includegraphics[trim={0.cm 0cm 12.5cm 0cm},clip,
		width=\textwidth]{./fig_sfc_wd/obs_model_20161221_23_00}
		\caption{}\label{fig:scat:wd2123}
	\end{subfigure}
	%
	\begin{subfigure}[b]{0.49\textwidth}
		\includegraphics[trim={0.cm 0cm 12.5cm 0cm},clip,
		width=\textwidth]{./fig_sfc_wd/obs_model_20161224_26_00}
		\caption{}\label{fig:scat:wd2426}
	\end{subfigure}
    	
	% label
	\begin{subfigure}[b]{0.49\textwidth}
		\centering
		\includegraphics[trim={25.cm 15.5cm 0cm 3.6cm},clip,
		width=0.8\textwidth]{./fig_sfc_temp/obs_model_20161221_23_00}
	\end{subfigure}
	\begin{subfigure}[b]{0.49\textwidth}
		\centering
		\includegraphics[trim={25.cm 15.5cm 0cm 3.6cm},clip,
		width=0.8\textwidth]{./fig_sfc_temp/obs_model_20161224_26_00}
	\end{subfigure}
    \caption{\textit{(Continued from previous page.)} Upper panel \SI{2}{\metre} air temperature, second panel \SI{10}{\metre} wind direction.}
\end{figure}
\begin{figure}\ContinuedFloat
	% sfc ws
	\begin{subfigure}[b]{0.49\textwidth}
		\includegraphics[trim={0.cm 0cm 12.5cm 0cm},clip,
		width=\textwidth]{./fig_sfc_ws/obs_model_20161221_23_00}
		\caption{}\label{fig:scat:ws2123}
	\end{subfigure}
	%
	\begin{subfigure}[b]{0.49\textwidth}
		\includegraphics[trim={0.cm 0cm 12.5cm 0cm},clip,
		width=\textwidth]{./fig_sfc_ws/obs_model_20161224_26_00}
		\caption{}\label{fig:scat:ws2426}
	\end{subfigure}
% 	% label
% 	\begin{subfigure}[b]{0.49\textwidth}
% 		\centering
% 		\includegraphics[trim={25.cm 15.5cm 0cm 3.6cm},clip,
% 		width=0.8\textwidth]{./fig_sfc_temp/obs_model_20161221_23_00}
% 	\end{subfigure}
% 	\begin{subfigure}[b]{0.49\textwidth}
% 		\centering
% 		\includegraphics[trim={25.cm 15.5cm 0cm 3.6cm},clip,
% 		width=0.8\textwidth]{./fig_sfc_temp/obs_model_20161224_26_00}
% 	\end{subfigure}
%     \caption{\textit{(Continued from previous page.)} }
% \end{figure}
% \begin{figure}\ContinuedFloat
	% sfc precip
	\begin{subfigure}[b]{0.49\textwidth}
		\includegraphics[trim={0.cm 0cm 12.5cm 0cm},clip,
		width=\textwidth]{./fig_sfc_precip/obs_model_20161221_23_00}
		\caption{}\label{fig:scat:precip2123}
	\end{subfigure}
	%
	\begin{subfigure}[b]{0.49\textwidth}
		\includegraphics[trim={0.cm 0cm 12.5cm 0cm},clip,
		width=\textwidth]{./fig_sfc_precip/obs_model_20161224_26_00}
		\caption{}\label{fig:scat:precip2426}
	\end{subfigure}
	% label
	\begin{subfigure}[b]{0.49\textwidth}
		\centering
		\includegraphics[trim={25.cm 15.5cm 0cm 3.6cm},clip,
		width=0.8\textwidth]{./fig_sfc_temp/obs_model_20161221_23_00}
	\end{subfigure}
	\begin{subfigure}[b]{0.49\textwidth}
		\centering
		\includegraphics[trim={25.cm 15.5cm 0cm 3.6cm},clip,
		width=0.8\textwidth]{./fig_sfc_temp/obs_model_20161224_26_00}
	\end{subfigure}
	\caption{\textit{(Continued from previous page.)} Upper panel \SI{10}{\metre} wind speed, lower panel surface precipitation amount. }
\end{figure}
%%%%%%%%%%%%%%%%%%%%%%%%%%%%%%%%%%%%%%%%%%%%%%
\\
\\
The comparison between the ECMWF analysis displays, that the ensemble member forecast system MEPS covers the prediction of large scale phenomena like frontal boundaries and liquid precipitation at the surface. \Cref{fig:scat:obs_meps} presents the correlation between the observations and the \SI{48}{\hour} ensemble forecast. The relation between Haukeliseter observations and the forecast members is indicated by the regression line for each day.
\\
Sea level pressure has the best correlation under all variables. The best agreement shows on \SI{26}{\dec} when the Christmas storm made landfall and dissipates after the passage of the occluded front at \SI{16}{\UTC}. \citet{dahlgren_comparison_2013} showed that by mixing in large scale information from the boundary condition (ECMWF) into the regional model, the forecast for sea level pressure will be improved. The model-observation comparison by \citet{dahlgren_comparison_2013} showed a declination of forecasts with pressure mixing after \SI{24}{\hour} 
Since the pressure values are in good agreement with the observations is assumed that the warm front did not pass through at Haukeliseter on \SI{25}{\dec} and only the warm sector is observed. This shows a quite detailed forecast ability of MEPS, as from the ECMWF analysis it is not quite clear where the warm front could have passed through. 
\\
\Cref{fig:scat:temp2426} indicates a moderate correlation between observation and the \SI{48}{\hour} ensemble member forecast system. In general, underestimated MEPS the observed temperature, but it estimated it at the correct timing on \SI{25}{\dec}. The previous operational model AROME-MetCoOp showed a negative winter bias after winter 2012 \citep{muller_arome-metcoop:_2017}. \Cref{fig:bias:temp} shows for \SIlist{23;26}{\dec} positive as well as negative mean error for the individual members. On \SI{25}{\dec}, during the warm sector a negative bias was observed, underestimating the temperature when compared to the observation. The mean error for the Norwegian model domain of AROME-MetCoOP estimated by \citet{muller_arome-metcoop:_2017} is smaller than \SI{1.8}{\kelvin} for the surface temperature in December 2014. The forecasts for \SIlist{23;25;26}{\dec} show mean absolute error values of up to \SIlist{0.61;0.77;1.44}{\kelvin}, respectively. It shows by using an ensemble forecast system a reduction of mean errors and an increase in forecast accuracy can be done. 
\\
During the Christmas storm 2016 high wind speeds were observed at the Haukeliseter site (\Cref{fig:scat:ws2123} and \subref{fig:scat:ws2426}).
According to \citet{muller_arome-metcoop:_2017} are large wind speeds significantly better simulated for AROME-MetCoOp compared to ECMWF's forecast. The wind speeds are still overestimated, which is an already known difficulty in the deterministic version of MEPS. In AROME-MetCoOp wind speed prediction agreed better with observations for wind speeds between \SIrange{3}{13}{\mPs} than ECMWF forecasts did, showing the advantage of a high-resolution weather model. With increasing wind speed the forecast accuracy decreases with a mean absolute error below \SI{2}{\mPs} for December 2014 in AROME-MetCoOp.
The mean absolute error for wind speed during the Christmas storm is higher at all days ranging from \SIrange{3}{7}{\mPs}.
During the three days with frontal passages shows the \SI{23}{\dec} the highest mean absolute error of \SI{6.5}{\mPs}, more than three times as high as the monthly averaged value from \citet{muller_arome-metcoop:_2017}. Their study case in February 2015 showed a slight overestimation of ECMWF \SI{10}{\metre} wind compared to the Norwegian AROME-MetCoOp domain, but still overestimates MEPS the wind.
\textcolor{red}{What could be still a weakness that the model overestimates the wind speed? In \citet{muller_arome-metcoop:_2017}: change from ECOCLIMAP1 because the surface roughness was too low and followed high wind speeds? Is this still the case for MEPS? High wind speeds followed also from wrongly adressed 'permanent snow'. Do not use 'orographi drag' in AROME-MetCoOp, could that lead to the too high estimated wind? When 'canopy drag' was changed saw increase in SBL drag which followed a decrease in wind speed. But AROME-MetCoOp is able to forecast high wind speeds, while ECMWF is not.}
\\
Haukeliseter is a measurement site exposed to high wind speeds \citep{wolff_measurements_2013,wolff_derivation_2015} the ensemble prediction system MEPS seems to still have issues forecasting the wind speed correctly in mountainous terrain.
\Cref{fig:res:sfc_obs_meps} indicates that MEPS is able to estimate larger scale features which is probably related to the outer boundary conditions of ECMWF described by \citet{dahlgren_comparison_2013}.
In general, were surface parameters well predicted, only wind speed and precipitation accumulation showed overestimation in MEPS. Wind speeds forecasted higher than observations, is probably related to the weakness of wind speed prediction already known from the previous operational model AROME-MetCoOp. On the \SIlist{25;26}{\dec} MEPS also overestimated the precipitation amount at the surface, this will be further discussed in \Cref{sec:sfc_acc}.
%%%%%%%%%%%%%%%%%%%%%%%%%%%%%%%%%%%%%%%%%%%%%%%%%%%%%%%%%%%%%%%%%%%%%%%%%%

%%%%%%%%%%%%%%%%%%%%%%%%%%%%%%%%%%%%%%%%%%%%%%%%%%%%%%%%%%%%%%%%%%%%%%%%%
%%%%%%%% Overestimation of surface snowfall %%%%%%%%%%%%%%
\section{Surface snowfall accumulation}\label{sec:sfc_acc}
%%% image surface accumulation %%%%%%%%%%%%%%%%%%%%%%%%%%%%%%%%%%%%%
\begin{figure}[t!]
	\centering
	% 21/12
	\begin{subfigure}[t]{0.49\textwidth}		\includegraphics[trim={0.cm 5.2cm 0.cm 0cm},clip,width=\textwidth]{./fig_sfc_acc/acc_wind_20161221_00}
		\caption{}\label{fig:sfc_acc21}
	\end{subfigure}
	% 22/12
	\begin{subfigure}[t]{0.49\textwidth}		\includegraphics[trim={0.cm 5.2cm 0.cm 0cm},clip,width=\textwidth]{./fig_sfc_acc/acc_wind_20161222_00}
		\caption{}\label{fig:sfc_acc22}
	\end{subfigure}
	%	\end{figure}
	%   \begin{figure}\ContinuedFloat
	% 23/12
	\begin{subfigure}[t]{0.49\textwidth}	\includegraphics[trim={0.cm 5.2cm 0.cm 0cm},clip,width=\textwidth]{./fig_sfc_acc/acc_wind_20161223_00}
		\caption{}\label{fig:sfc_acc23}
	\end{subfigure}
	% 24/12
	\begin{subfigure}[t]{0.49\textwidth}			\includegraphics[trim={0.cm 5.2cm 0.cm 0cm},clip,width=\textwidth]{./fig_sfc_acc/acc_wind_20161224_00}
		\caption{}\label{fig:sfc_acc24}
	\end{subfigure}
	% 25/12
	\begin{subfigure}[t]{0.49\textwidth}
		\includegraphics[trim={0.cm 3.6cm 0.cm 0cm},clip,width=\textwidth]{./fig_sfc_acc/acc_wind_20161225_00}
		\caption{}\label{fig:sfc_acc25}
	\end{subfigure}
	% 26/12
	\begin{subfigure}[t]{0.49\textwidth}	\includegraphics[trim={0.cm 3.6cm 0.cm 0cm},clip,width=\textwidth]{./fig_sfc_acc/acc_wind_20161226_00}
		\caption{}\label{fig:sfc_acc26}
	\end{subfigure}
	
	% label
	\begin{subfigure}[t]{\textwidth}
		\centering
		\includegraphics[trim={1.2cm 0cm 1.1cm 21.4cm},clip,width=0.8\textwidth]{./fig_sfc_acc/acc_wind_20161226_00}
	\end{subfigure}
	\caption{\SI{48}{\hour} surface snowfall accumulation for \SIrange{21}{26}{\dec} (\protect\subref{fig:sfc_acc21}  \protect\subref{fig:sfc_acc26}). Representing the values from the double fence in red, hexagons; optimal estimation retrieval output at snow layer height \SI{400}{\metre} in dash-dotted green; and ensemble member deterministic forecast, initialised at \SI{00}{\UTC} in black and its nine perturbed ensemble members in grey. The ensemble mean of all ten members is shown in dashed blue. Underneath are the associated last hour \SI{10}{\minute} average wind from the weather mast at \SI{10}{\metre} height. }\label{fig:sfc_acc}
\end{figure}
%%%%%%%%%%%%%%%%%%%%%%%%%%%%%%%%%%%%%%%%%%%%%%%%%%%%%%%%%%%%%%%%%%%%%%%%%%
One approach of this study is to see if observed surface accumulation was correctly predicted by the regional weather model MEPS. Precipitation amount at the surface are shown in \Cref{fig:sfc_acc}. The figures are representing the observed and forecasted surface precipitation accumulation in \SI{}{\mm} over \SI{48}{\hour}. Accumulation, measured by the double fence are presented as red hexagons. Minutely retrieved surface snowfall amount in dash-dotted green. The ten \SI{48}{\hour} forecast ensemble members are lines in black and grey, the deterministic and its perturbed ensemble members, respectively. The blue dashed line shows the ensemble mean of all ten members. Since the deterministic and the first ensemble member are having values every hour and the other perturbed members only every three hours, shows the ensemble mean the precipitation amount at \SI{0}{\hour}, \SI{3}{\hour}, $\ldots$, \SI{21}{\hour}, \SI{24}{\hour}, $\ldots$, \SI{48}{\hour} forecast time. When too few ensemble members were present, like on \SI{23}{\dec}, no ensemble mean is calculated (\Cref{fig:sfc_acc23}). 
Underneath is the associated \SI{10}{\minute} average wind of the last hour from the \SI{10}{\metre} weather mast at Haukeliseter, to see if surface accumulation observations may be influenced by wind. 
\\
\Cref{fig:sfc_acc21,fig:sfc_acc22,fig:sfc_acc23} show in general a better agreement between observations and forecast for \SI{48}{\hour} forecasts initialised on \SIrange{21}{23}{\dec} at \SI{0}{\UTC}. The spread of the ensemble members around the control run fit better to the observations as well than initialisations on \SIrange{24}{26}{\dec}. 
During these days intensifies the low-pressure system and gets closer advected to the Norwegian coast and influencing the local weather in Norway (\Cref{ch:weather_ana}). \Cref{fig:sfc_acc24,fig:sfc_acc25,fig:sfc_acc26} indicates a larger estimated surface precipitation amount for all ten ensemble members than observed at the measurement site between \SIrange{24}{26}{\dec}. 
\\ 
\\
The correlation between double fence observation and ensemble forecast is presented in \Cref{fig:scat:pres2123} and \subref{fig:scat:pres2426}. Showing a better agreement between \SIrange{21}{23}{\dec} than initialisation on \SIrange{24}{26}{\dec}. On \SIrange{21}{23}{\dec} is the slope of the regression relatively close to unity, indicating a good agreement between the ensemble forecast and the observations by the double fence.
The largest disagreement between surface observations and forecasts is seen on \SI{25}{\dec} with a positive bias up to \SI{17}{\mm} (\Cref{fig:bias:precip}). The mean absolute error is not larger than \SI{13}{\mm} for the first three days and increases with intensification of the storm up to \SI{19}{\mm} on \SI{24}{\dec}.
\\
Initialisations on \SI{24}{\dec} indicate an overestimation of the deterministic surface snowfall prediction already after \SI{13}{\hour} forecast time. The deterministic forecast in solid black is much higher and increases faster than the observations. In \Cref{fig:sfc_acc24} at \SI{16}{\UTC} a higher value of approximately \SI{15}{\mm} can be seen when compared to the surface measurements. This difference remains almost constantly over the forecast time. Furthermore, all ensemble members seem to overestimate the surface accumulation after \SI{24}{\hour} prediction time. 
\\
Since the MEPS performance was better on the previous days one might assume that the double fence measurement is influenced by surface winds. It shows in \Cref{fig:sfc_acc24} that the \SI{10}{\minute} average wind at \SI{13}{\UTC} increases from \SI{5}{\mPs} to \SI{10}{\mPs} (see also \Cref{fig:res:sfc_ws24}). \citet{wolff_wmo_2018} states that the double fence gauge is influenced by wind, but accumulation measurement errors occur rather at higher wind speeds larger than \SI{20}{\mPs}. It is therefore assumed that the measurements from the double fence are correct and MEPS had rather a forecasting issue.
\\
While the cyclone gets more advected to Norway increases the forecast inaccuracy of the surface precipitation. 
On \SI{25}{\dec} the miscalculation of the precipitation amount is associated with the warm sector passage at Haukeliseter (\Cref{sec:res:large_scale_sfc}). Afterwards follows the model the same path as the double fence observations, but higher. The \SI{25}{\dec} indicates a good spread between the ensemble members and the deterministic forecast, while on \SI{24}{\dec} the ensemble members were not spread symmetrically around the deterministic forecast. 
\\
An overestimation of the surface accumulation is also observed on \SI{26}{\dec}. While the large-scale analysis indicates the passage of an occlusion after \SI{15}{\UTC} (\Cref{fig:DT26}, \ref{fig:GP26}) seems the overestimation to occur after \SI{12}{\hour} forecast time in \Cref{fig:sfc_acc26}. Again, all ensemble members seem to follow the course of the double fence accumulation, but larger. 
\\
Whereas the spread between the ensemble members is large in the beginning of \SI{24}{\dec} is the variability between the members narrow for \SIlist{25;26}{\dec}. The variability between the ensemble member can be compared with a box-whisker plot. A box-whisker-plot shows the time evolution of the distribution of the precipitation amount made of ten ensemble members up to \SI{48}{\hour}. Since some ensemble member do not have forecast values every hour provides the box-whisker-plot in \Cref{fig:boxplot} information every \SI{3}{\hour}. The red line shows the ensemble mean of all ten members and shows if the distribution is skewed. The short light green horizontal line is showing the median, wide vertical box represents the 25th and 75th percentiles, and minimum and maximum values are indicated by the vertical lines, whiskers.
\\
%%% image surface accumulation %%%%%%%%%%%%%%%%%%%%%%%%%%%%%%%%%%%%%
\begin{figure}[h!]
	\centering
	\begin{subfigure}[b]{\textwidth}
		\centering
		\includegraphics[trim ={0cm 2.2cm 0cm 0cm},clip,width=\textwidth]{./fig_boxplot_sfc/20161224_0}
		\caption{}\label{fig:boxplot:24}
	\end{subfigure}
	%
	\begin{subfigure}[b]{\textwidth}
		\centering
		\includegraphics[trim ={0cm 2.2cm 0cm 0cm},clip,width=\textwidth]{./fig_boxplot_sfc/20161225_0}
		\caption{}\label{fig:boxplot:25}
	\end{subfigure}
	%
	\begin{subfigure}[b]{\textwidth}
		\centering
		\includegraphics[trim ={0cm 1.cm 0cm 0cm},clip,width=\textwidth]{./fig_boxplot_sfc/20161226_0}
		\caption{}\label{fig:boxplot:26}
	\end{subfigure}
	\caption{Box-whisker-plot of the ten ensemble members of MEPS. Red line indicating the ensemble mean, lower and upper whisker the 25th and 75th percentile, respectively. Light green shows the median of all members and the box represents the middle \SI{50}{\percent} of scores of the precipitation.}\label{fig:boxplot}
\end{figure}
%%%%%%%%%%%%%%%%%%%%%%%%%%%%%%%%%%%%%%%%%%%%%%%%%%%%%%%%%%%%%%%%%%%%%%%%%%
\\
The box-whisker-plot in \Cref{fig:boxplot} shows the distribution of the ten ensemble members for the respective days. All three days with overestimation seem to be different in their variability. As expected increases the forecast uncertainty with longer forecast time for precipitation amount.  
\\
\Cref{fig:boxplot:25} shows for \SI{25}{\dec} the least variability between the ten ensemble members of up to almost \SI{24}{\hour}. The \SI{25}{\dec} is also the forecast with the smallest positive bias of these three days. As \Cref{fig:scat:precip2426} suggests is the overestimation not as high as for \SIlist{24;25}{\dec}. On \SIlist{24;26}{\dec} is the mean error for the surface accumulation largest with values up to \SI{19}{\mm}. 
\\
Larger variability is already present after \SI{3}{\hour} prediction time in \Cref{fig:boxplot:24} on \SI{24}{\dec}. The spread between the ensemble members (shown by the minimum and maximum whiskers) seems to be wide indicating a larger uncertainty about the amount of surface accumulation. The ensemble mean (red line) is always higher than the median and already after \SI{12}{\hour} forecast time is the median closer to the lower 25th percentile. Also, all upper whiskers in \Cref{fig:boxplot:24} are taller than the lower ones, which would follow that the ensemble members vary amongst the most positive quartile and that it is very similar for the least positive quartile group. Since the deterministic forecast, black line in \Cref{fig:sfc_acc24}, is in the upper percentile compared to its perturbed members it follows that for this forecast the deterministic forecast was not the best guess for the surface accumulation and by using the 'wrong' initial state it can have led to larger miscalculations. 
\\
I believe that the uncertainty appearing already after \SI{3}{\hour} could be associated with a too long spin-up time of MEPS. MEPS usually has a spin-up time of about three hours, on \SI{24}{\dec} this might have been longer as a result of poorer initial conditions \textcolor{red}{Need a reference here, not stated in \citet{muller_arome-metcoop:_2017}}. The regional model MEPS receives initial and boundary conditions from ECMWF before it can produce forecasts \citep{muller_arome-metcoop:_2017}. Since initial conditions such as observations have uncertainties as well as the model has mistrust, and  the own climatology needs to be approached, a model has to stabilize before the simulations can be trusted. The spin-up time varies depending on the quality of the initial and boundary conditions. Apparently, it seems, that the initial and boundary conditions for MEPS were not perfect for initialisations on \SI{24}{\dec} at \SI{0}{\UTC}. The deterministic and perturbed members seem not to have stabilised yet and show larger variability in \Cref{fig:boxplot24} from early on.
\\
The uncertainty might have also been related to the fact, that the large-scale situation got more complex. The precipitation amount associated with the passage of an occluded front on \SI{23}{\dec} was higher than on the previous days (\Cref{fig:res:sfc_precip23} and \Cref{fig:TPU}). On previous days was the hourly precipitation around \SI{0}{\UTC} less intense than on \SI{23}{\dec}. This led to a higher accumulation amount over shorter time and could have followed a larger variability in the forecast model. Another possibility is perhaps that MEPS might have accounted for additional precipitation around \SI{12}{\UTC} on \SI{24}{\dec} and this showed a stronger increase in accretion in \Cref{fig:sfc_acc24} at \SI{13}{\UTC}. I believe it could be associated to a local resolution effect of MEPS. \Cref{fig:meps:site} shows the MEPS resolution and its \SI{2.5}{\km} grid cells around the Haukeliseter site. The complex terrain represented in the model could have followed a local misplacement of a precipitation cell by a few kilometres and followed an estimation of more accumulation at the site after noon.
\\
\\
\Cref{fig:boxplot:25} and \subref{fig:boxplot:26} show a smaller ensemble member variability on \SIlist{25;26}{\dec} than on \SI{24}{\dec}. The box-whiskers are narrower for the first \SI{30}{\hour} in \Cref{fig:boxplot:25}, but slightly larger after \SI{6}{\hour} forecast time for initialisations on \SI{26}{\dec}. \Cref{sec:res:large_scale_sfc} presented a good agreement between observations and forecast of large scale features in terms of pressure, temperature and wind direction. While the occlusion on \SI{26}{\dec} was more intuitive (\Cref{fig:res:sfc_pres26,fig:res:sfc_temp26,fig:res:sfc_wd26,fig:res:sfc_ws26}) than the warm front passage on \SI{25}{\dec} (\Cref{fig:res:sfc_pres25}, \subref{fig:res:sfc_temp25}, \subref{fig:res:sfc_wd25}, \subref{fig:res:sfc_ws25}) shows the mean error for each variable to be best for \SI{25}{\dec} (\Cref{fig:MAE:pres},\subref{fig:MAE:temp},\subref{fig:MAE:wd}, \subref{fig:MAE:ws}, and \subref{fig:MAE:precip}).
\\
On \SI{25}{\dec}the overestimation started to occur around \SI{13}{\UTC} in \Cref{fig:sfc_acc25}, related to the delayed forecasted temperature increase in \Cref{fig:res:sfc_temp25}.  As \Cref{fig:boxplot25} shows, increases the variability in the forecast after \SI{15}{\hour} prediction time. In general, agree median and mean well for the entire period of a \SI{48}{\hour} forecast. After \SI{39}{\hour} prediction time is the mean much higher than the median and closer to the lower 25th percentile in \Cref{fig:boxplot25}. It seems, that all ten ensemble members agree well on the prediction and nevertheless overestimates MEPS the surface accumulation. I consider that MEPS misinterpreted the amount of precipitation related to the passage of the warm sector.  
\\
During \SI{26}{\dec} the core of the low-pressure system goes through between \SIlist{15;18}{\UTC} at Haukeliseter. The box-whiskers in \Cref{fig:boxplot:26} indicates a larger variability after \SI{6}{\hour} prediction while the precipitation amount forecast is miscalculated at \SI{12}{\UTC} and following the structure of the double fence observation. Variability of all ensemble members show to increase at \SI{6}{\hour} forecast time, but then decreases again in \Cref{fig:boxplot:26}.  
\\
Since the box-whisker-plot in \Cref{fig:boxplot:25} and \subref{fig:boxplot:26} show less variability in the beginning it is assumed that spin-up time issues are less likely. It could be related to an error in the initialisation state, even though it does not show in the variability in the beginning. An error associated with the spin-up time of MEPS is not totally excluded for these days. In \Cref{fig:sfc_acc25} and \subref{fig:sfc_acc26} agrees the ensemble mean well with the deterministic forecast, which is an indication of a symmetrical spread around the deterministic run. 
\\ 
The overestimation during \SIlist{24;26}{\dec} might be related to the high forecasted wind speeds, as well as to the complex development of the low-pressure system north-west of Norway on \SI{24}{\dec}.
\\
%%% table surface accumulation %%%%%%%%%%%%%%%%%%%%%%%%%%%%%%%%%%%%%
\begin{table}[h]
	\begin{center}
		\caption{Surface snowfall accumulation measured by the double fence gauge. Presenting \SI{12}{\hour} accumulation before noon and after noon, as well as the total \SI{24}{\hour} surface accretion. }\label{tab:sfc_acc}
		\begin{tabular}{c|c|c|c}
			\hline \hline
			\textbf{Day} & \multicolumn{3}{c}{\textbf{Accumulation}} \\ 
			& \multicolumn{3}{c}{[\SI{}{\mm}]} \\ \hline
			& \SI{12}{\hour} (\footnotesize{\num{0} to \SI{12}{\UTC}}) & \SI{12}{\hour} (\footnotesize{\num{12} to \SI{23}{\UTC}}) & \SI{24}{\hour} \\ \hline \hline
			\SI{21}{\dec} & \num{0.7} &  \num{16.4} & \num{17.1} \\ \hline
			\SI{22}{\dec} & \num{13.6} &  \num{12.0} & \num{25.6} \\ \hline
			\SI{23}{\dec} & \num{6.3} &  \num{17.0} & \num{23.3} \\ \hline
			\SI{24}{\dec} & \num{14.7} &  \num{10.1} & \num{24.8} \\ \hline
			\SI{25}{\dec} & \num{4.3} &  \num{11.1} & \num{15.4} \\ \hline
			\SI{26}{\dec} & \num{8.8} &  \num{16.3} & \num{25.1} \\ 
			\hline \hline
		\end{tabular}
	\end{center}
\end{table}
%%%%%%%%%%%%%%%%%%%%%%%%%%%%%%%%%%%%%%%%%%%%%%%%%%%%%%%%%%%%%%%%%%%%%%%%%%
\\
According to \citet{muller_arome-metcoop:_2017} are strong precipitation events better predicted with AROME-MetCoOp than with ECMWF (European Centre for Medium-Range Weather Forecasts). In \Cref{sec:dim:dec_obs} it was described, that during \SIrange{21}{27}{\dec} \SI{56.9}{\percent} of the total December 2016 accumulation were observed. \citet{muller_arome-metcoop:_2017} states also, that an overestimation appears, where the precipitation event (\SI{12}{\hour} accumulation) is less than \SI{10}{\mm} this seems not to be true for all days but could be possible for \SIlist{25;26}{\dec} (observed accumulation in \Cref{tab:sfc_acc}). In December 2014 was the \SI{12}{\hour} precipitation mean absolute error in AROME-MetCoOp with \SI{1.5}{\mm}. For the Christmas storm is the mean absolute error not larger than \SI{5}{\mm} for the first \SI{12}{\hour} accumulation on \SIlist{24;26}{\dec} (\Cref{fig:MAE:precip12}). Therefore, the assumption follows that on \SI{26}{\dec} the overestimation might be correlated to the <\SI{10}{\mm} problem described by \citet{muller_arome-metcoop:_2017}. The \SI{12}{\hour} accumulation is presented in \Cref{tab:sfc_acc} for Haukeliseter and shows that \SI{12}{\hour} accrection was less than \SI{10}{\mm} for \SIlist{25;26}{\dec}.
On \SI{25}{\dec} the mean absolute error was \SI{1.1}{\mm} for the first \SI{12}{\hour} accumulation and shows that this could be an influence but does not necessarily mean to be the case, since the overestimation started to occur after \SI{11}{\hour} prediction time. 
\\
\\
% keep as summary in the end
It will be interesting to re-run the ensemble prediction system again with all available observations to see, if this has an influence on the overestimation indicated in \Cref{fig:sfc_acc24,fig:sfc_acc25,fig:sfc_acc26}. ECMWF as boundary condition might not have reached its stabilised state itself when MEPS was initiated and could also have led to a misinterpretation of surface accumulation. A re-run with analysis data from ECMWF could possibly improve the original forecast \textcolor{red}{find reference for this}. 
Another approach could be to perturb the initial state (deterministic forecast) in other way, to see if different perturbations might lead to a better correlation between observation and forecast at the ground than presented in \Cref{fig:scat:precip2124}. Also, the deterministic forecast (best guess) might have been chosen incorrectly and followed a miscalculation of surface accumulation, since the misinterpreted best guess was perturbed.
It is very important to have correct measurements such as the double fence or MRR observations, to produce better initial condition for weather forecast models, so that initialisations can start at a realistic state.
\\
Also, more study cases should be considered to get a better estimate about the performance of MEPS during extreme winter events. The mean absolute error for \SI{12}{\hour} accumulation has shown a great variability, depending on the initialisation time and the intensification of the low-pressure system.  
%%%%%%%%%%%%%%%%%%%%%%%%%%%%%%%%%%%%%%%%%%%%%%%%%%%%%%%%%%%%%%%%%%%%%%%%%%





%%%%%%%%%%%%%%%%%%%%%%%%%%%%%%%%%%%%%%%%%%%%%%%%%%%%%%%%%%%%%%%%%%%%%%%%%

%%%%%%%%%%%%%%%%%%%%%%%%%%%%%%%%%%%%%%%%%%%%%%%%%%%%%%%%%%%%%%%%%%%%%%%%%%
%%%%%%%%% SFC ACCUMULATION %%%%%%%%%%%%%%
% !TeX spellcheck = en_GB
\section{Surface snowfall accumulation}\label{sec:sfc_acc}
The surface accumulation is a good point to start to see, if the retrieved snowfall amounts at the surface catch the boundary condition as of the double fence. Precipitation amount at the surface are shown in \Cref{fig:sfc_acc}. The figures are representing the observed surface precipitation accumulation in \SI{}{\mm} over \SI{48}{\hour}. Accumulation, measured by the double fence are presented as purple hexagons. Minutely retrieved surface snowfall amount in dash-dotted orange. The ten \SI{48}{\hour} forecast ensemble members are lines in black and grey, the deterministic and its perturbed ensemble members, respectively. The blue dashed line shows the ensemble mean of all ten members. Since the deterministic and the first ensemble member are having values every hour and the other perturbed members only every three hours, shows the ensemble mean the precipitation amount at 0\SI{0}{\hour}, \SI{3}{\hour}, $\ldots$, \SI{21}{\hour}, \SI{24}{\hour}, $\ldots$, \SI{48}{\hour} forecast time. 
Underneath is the associated \SI{10}{\minute} average wind of the last hour from the \SI{10}{\metre} weather mast at Haukeliseter, to see if surface accumulation observations are influenced by wind. 
%%% image surface accumulation %%%%%%%%%%%%%%%%%%%%%%%%%%%%%%%%%%%%%
% !TeX spellcheck = en_GB
%\begin{landscape}
\begin{figure}[t!]
	\centering
	% 21/12
	\begin{subfigure}[t]{0.49\textwidth}		\includegraphics[trim={3.cm 2.6cm 2.cm 1.9cm},clip,width=\textwidth]{./fig_sfc_acc/acc_wind_20161221_00}
		\caption{}\label{fig:sfc_acc21}
	\end{subfigure}
	% 22/12
	\begin{subfigure}[t]{0.49\textwidth}		\includegraphics[trim={3.cm 2.6cm 2.cm 1.9cm},clip,width=\textwidth]{./fig_sfc_acc/acc_wind_20161222_00}
		\caption{}\label{fig:sfc_acc22}
	\end{subfigure}
	%	\end{figure}
	%   \begin{figure}\ContinuedFloat
	% 23/12
	\begin{subfigure}[t]{0.49\textwidth}	\includegraphics[trim={3.cm 2.6cm 2.cm 1.9cm},clip,width=\textwidth]{./fig_sfc_acc/acc_wind_20161223_00}
		\caption{}\label{fig:sfc_acc23}
	\end{subfigure}
	% 24/12
	\begin{subfigure}[t]{0.49\textwidth}			\includegraphics[trim={3.cm 2.6cm 2.cm 1.9cm},clip,width=\textwidth]{./fig_sfc_acc/acc_wind_20161224_00}
		\caption{}\label{fig:sfc_acc24}
	\end{subfigure}
	% 25/12
	\begin{subfigure}[t]{0.49\textwidth}
		\includegraphics[trim={3.cm 2.6cm 2.cm 1.9cm},clip,width=\textwidth]{./fig_sfc_acc/acc_wind_20161225_00}
		\caption{}\label{fig:sfc_acc25}
	\end{subfigure}
	% 26/12
	\begin{subfigure}[t]{0.49\textwidth}	\includegraphics[trim={3.cm 2.6cm 2.cm 1.9cm},clip,width=\textwidth]{./fig_sfc_acc/acc_wind_20161226_00}
		\caption{}\label{fig:sfc_acc26}
	\end{subfigure}
	\caption{Surface snowfall accumulation. Representing the values from the double fence in purple, hexagons; optimal estimation retrieval output at snow layer height \SI{800}{\metre} in dash-dotted orange; and ensemble member deterministic forecast, initialised at 0\SI{00}{\UTC} in black and its nine perturbed ensemble members in grey. The ensemble mean of all ten members is shown in blue dashed.  Underneath are the associated last \SI{10}{\minute} average wind from the weather mast at \SI{10}{\metre} height. }\label{fig:sfc_acc}
\end{figure}



%%%%%%%%%%%%%%%%%%%%%%%%%%%%%%%%%%%%%%%%%%%%%%%%%%%%%%%%%%%%%%%%%%%%%%%%%%
\noindent
In general show the \SI{48}{\hour} surface accumulation in \Cref{fig:sfc_acc21,fig:sfc_acc22,fig:sfc_acc23} a good agreement between the foretasted values and the retrieved snowfall amount when comparing to the double fence. \SI{24}{\dec} and \SI{25}{\dec} show a disagreement between the surface observations and the model forecast. During this days is the precipitation amount predicted by MEPS for all ten ensemble members higher than for the measured accumulation. The possible reason for the overestimation at the ground is later discussed in \Cref{sec:2412:surface} and \ref{sec:2512:surface}. \\
Retrieved accumulation almost always reached the boundary condition of the double fence observations. The only well pronounced mismatch is seen on \SI{25}{\dec}, where it measures much less than the double fence gauge.  
\\ \\
The surface accumulation initialised on the \SI{21}{\dec} at 0\SI{0}{\UTC} has one ensemble member overestimating the precipitation amount after \SI{33}{\hour} forecast time. Otherwise, agree all three systems well with each other and the perturbed ensemble members are equally spread around the deterministic forecast.
\\
On \SI{22}{\dec} (\Cref{fig:sfc_acc22}) fit the ensemble mean relatively well to the observed surface accumulation, were the double fence estimates the least amount. Clearly, the ensemble members in grey are not equally distributed around the deterministic forecast. The deterministic is predicting more surface accumulation with a large jump after \SI{11}{\hour}, always being higher than most of the ensemble members and observations. 
\\
When too few ensemble members were present, like on \SI{23}{\dec}, no ensemble mean is calculated. \Cref{fig:sfc_acc23} shows a good agreement between the double fence observations and the deterministic forecast. Here, for the first time measures the retrieved surface snowfall accumulation less than for the double fence, but the difference is almost negligible and starts to be too little after \SI{20}{\UTC} on \SI{23}{\dec}. This underestimation might be related to the wind change from weaker south to stronger west wind.
\\
\Cref{fig:sfc_acc24} indicates an overestimation of the deterministic surface snowfall prediction already after \SI{16}{\hour} forecast time, when initialised on \SI{24}{\dec}. The deterministic forecast in solid black is much higher and increases faster than the observations. A higher value of approximately \SI{15}{\mm} can be seen when compared to the surface measurements at \SI{16}{\UTC} on \SI{24}{\dec}. This difference remains almost constantly over the forecast time. Furthermore, all ensemble members seem to overestimate the surface accumulation after \SI{24}{\hour} prediction time. Since MEPS performed on the previous days one might assume, that the double fence gauge measurements are influenced by the surface winds. By comparing the \SI{10}{\minute} average wind at \SI{13}{\UTC} it shows an increase of wind speed from \SI{5}{\mPs} to \SI{10}{\mPs}. In \cite{wolff_wmo_2018} it is stated, that the gauge protected by a double fence is influenced by wind but the error is not too big compared to strong wind higher than \SI{20}{\mPs}. Therefore, it is assumed that the measurements from the double fence are correct and MEPS had rather a forecast error, since the retrieved surface snow accumulation would assume the same precipitation amount. The total accumulated precipitation amount provided in MEPS includes liquid and solid precipitation. The ensemble mean shows also an inaccuracy of forecasted precipitation at the surface. One reason for the overestimation of the accumulation on the ground could be that MEPS has expected a large amount of liquid precipitation, which actually did not occur. A discussion, including a whisker-box-plot from the ensemble members is provided in \Cref{sec:2412:surface}.
\\
On the \SI{26}{\dec} the MRR did not work after approximately \SI{17}{\UTC} and therefore only values before \SI{17}{\UTC} are compared. 
The surface precipitation amount on \SI{25}{\dec} shows again a miscalculation from MEPS in \Cref{fig:sfc_acc25}. After \SI{12}{\hour} forecast time the ensemble members overestimate the surface accumulation, which gets more pronounced at \SI{18}{\UTC}. But still, the model forecast members seem to follow the same structure as the double fence, just too high. Compared to the \SI{24}{\dec} where the ensemble members were not spread equally around the deterministic forecast, shows the \SI{25}{\dec} a good distribution since the ensemble mean is almost the same as the deterministic forecast. 
The retrieved snowfall accumulation seems to be too little over the entire period, when it starts to precipitate more around \SI{18}{\UTC} on the \SI{25}{\dec} in \Cref{fig:sfc_acc25}. This might be, because the optimal estimation retrieval does not account for liquid precipitation, which was observed during this time period. While the double fence gauge measures liquid and solid precipitation could the pure neglection of liquid precipitation follow the disagreement between double fence and retrieved surface accumulation, which will be further discussed in \Cref{sec:2512:surface}.
\\
Because of an instrumentation error after \SI{17}{\UTC} on \SI{26}{\dec} is this day not really representable. From the double fence precipitation measurement in \Cref{fig:TPU26} and \ref{fig:TPU27} it is known that precipitation was continuous present until \SI{27}{\dec} \SI{10}{\UTC}. Nevertheless, \Cref{fig:sfc_acc26} shows an overestimation by MEPS after \SI{12}{\hour} prediction. The spread around the deterministic forecast is relatively narrow with a good agreement between ensemble mean and deterministic. 
\\ \noindent
%%%% DISCUSSION of sfc accumulation %%%%%%%%
% \textcolor{red}{DISCUSS! Why is the surface accumulation predicted better for the first days and not too well for the \SIlist{24;25}{\dec}? From Introduction, since excluded: \\
%
\begin{table}[h]
	\begin{center}
		\caption{Surface snowfall accumulation measured by the double fence gauge. Presenting \SI{12}{\hour} accumulation before noon and after noon, as well as the total \SI{24}{\hour} surface accretion. }\label{tab:sfc_acc}
		\begin{tabular}{c|c|c|c}
			\hline \hline
			\textbf{Day} & \multicolumn{3}{c}{\textbf{Accumulation}} \\ 
			& \multicolumn{3}{c}{[\SI{}{\mm}]} \\ \hline
			& \SI{12}{\hour} (\footnotesize{\num{0} to \SI{12}{\UTC}}) & \SI{12}{\hour} (\footnotesize{\num{12} to \SI{23}{\UTC}}) & \SI{24}{\hour} \\ \hline \hline
			\SI{21}{\dec} & \num{0.7} &  \num{16.4} & \num{17.1} \\ \hline
			\SI{22}{\dec} & \num{13.6} &  \num{12.0} & \num{25.6} \\ \hline
			\SI{23}{\dec} & \num{6.3} &  \num{17.0} & \num{23.3} \\ \hline
			\SI{24}{\dec} & \num{14.7} &  \num{10.1} & \num{24.8} \\ \hline
			\SI{25}{\dec} & \num{4.3} &  \num{11.1} & \num{15.4} \\ \hline
			\SI{26}{\dec} & \num{8.8} &  \num{16.3} & \num{25.1} \\ 
			\hline \hline
		\end{tabular}
	\end{center}
\end{table}
%
\\
According to \cite{muller_arome-metcoop:_2017} are strong precipitation events better predicted with MEPS than ECMWF (European Centre for Medium-Range Weather Forecasts), which are used as boundary conditions to initialise MEPS. In \Cref{sec:int:dec_obs} it was described, that during \SIrange{21}{27}{\dec} \SI{56.9}{\percent} of the total December 2016 accumulation was observed. Also, the Christmas storm was just above being called an extreme event with strong precipitation over seven days. 
\\
During the first few days the ensemble outputs cover the surface snow amount good in comparison to the double fence observations.
The spread of the ensemble members around the control run fits as well to the observations for this time period. The \SI{21}{\dec} had the highest snow accumulation within \SI{24}{\hour} at the surface (compare \Cref{tab:sfc_acc}). % 
\\ \noindent
For an initialisation on the \SI{24}{\dec}, \SI{00}{\UTC} one can see that  MEPS over estimates the amount of snow accumulation. It is even more pronounced with the initialisation on the \SI{25}{\dec}, \SI{00}{\UTC} (compare \Cref{fig:sfc_acc25}). Even though \cite{muller_arome-metcoop:_2017} states, that an overestimation appears, where the precipitation event (\SI{12}{\hour} accumulation) is less than \SI{10}{\mm} this seems not to be true for all days. On the \SI{24}{\dec} the miscalculation appears to happen after \SI{13}{\hour}. The accumulation before \SI{12}{\hour} was \SI{14.7}{\mm} and after that it was around \SI{10}{\mm}. Also on the \SI{25}{\dec} this seems not to be the case even though after noon \SI{12}{\hour} accumulation is less than \SI{10}{\mm}. While this was also the case on \SIlist{21;23}{\dec} before noon one can not see an inaccuracy between the observations and the forecast. Whereas on \SI{26}{\dec} the overestimation might be correlated to the \SI{10}{\mm} problem described by \cite{muller_arome-metcoop:_2017}, since until noon a small miscalculation can be seen and the double fence \SI{12}{\hour} accumulation measured \SI{8.8}{\mm}. 
%
%%%%%%%%%%%%%%%%


%%%%%%%%%%%%%%%%%%%%%%%%%%%%%%%%%%%%%%%%%%%%%%%%%%%%%%%%%%%%%%%%%%%%%%%%%%
%%%%%%%%% surface obs %%%%%%%%%%%%%%
\subsection{Wednesday, \SI{21}{\dec}}\label{sec:2112:surface}
The surface accumulation at the ground in \Cref{fig:sfc_acc21} showed a good agreement between retrieved snowfall amount, MEPS precipitation amount, and the reference frame of the double fence gauge. Since MEPS had an outlier ensemble member a box-whisker-plot is been provided. 
%%% image surface MEPS boxplot %%%%%%%%%%%%%%%%%%%%%%%%%%%%%%%%%%%%%
\begin{figure}[t]
	\includegraphics[width=\textwidth]{./fig_boxplot_sfc/20161221_0}
	\caption{Box-whisker-plot of the ten ensemble members of MEPS. Red line indicating the ensemble mean, lower and upper whisker the 25th and 75th percentile, respectively. Light green shows the median of all members and the box represents the middle \SI{50}{\percent} of scores of the precipitation.}\label{fig:boxplt21}
\end{figure}
%%%%%%%%%%%%%%%%%%%%%%%%%%%%%%%%%%%%%%%%%%%%%%%%%%%%%%%%%%%%%%%%%%%%%%%%%%
A box-whisker-plot shows the time evolution of the distribution of the precipitation amount made of ten ensemble members up to \SI{48}{\hour}. Since some ensemble member do not have forecast values every hour provides the box-whisker-plot in \Cref{fig:boxplt21} information every \SI{3}{\hour}. The red line shows the ensemble mean of all ten members. The short light green horizontal line is showing the median, wide vertical box represents the 25th and 75th percentiles, and minimum and maximum values are indicated by the vertical lines.
\\
The box-whisker-plot in \Cref{fig:boxplt21} shows the distribution of the ten ensemble members. In the first \SI{15}{\hour} of the forecast time agree all members well, since the box and whiskers are narrow. With increasing forecast time, increases the uncertainty. After \SI{33}{\hour} is the ensemble mean slightly higher than the median of the data. This shift is associated with the one ensemble member being an outlier.
In general can the surface forecast be trusted, especially up to \SI{24}{\hour} since the values of the ensemble members are well distributed around the mean. Maximum and minimum are not having a too large difference which also shows the small spread between the members.
%%%%%%%%%%%%%%%%%%%%%%%%%%%%%%%%%%%%%%%%%%%%%%%%%%%%%%%%%%%%%%%%%%%%%%%%%%

%%%%%%%%%%%%%%%%%%%%%%%%%%%%%%%%%%%%%%%%%%%%%%%%%%%%%%%%%%%%%%%%%%%%%%%%%%
%%%%%%%%% surface obs %%%%%%%%%%%%%%
\subsection{Saturday, \SI{24}{\dec}}\label{sec:2412:surface}
As discussed early seems the surface precipitation amount on \SI{24}{\dec} not to be influenced by too little precipitation which \cite{muller_arome-metcoop:_2017} showed to be the case for precipitation amount up to \SI{10}{\mm}. 
%%% image surface MEPS boxplot %%%%%%%%%%%%%%%%%%%%%%%%%%%%%%%%%%%%%
\begin{figure}[t]
	\includegraphics[width=\textwidth]{./fig_boxplot_sfc/20161224_0}
	\caption{Box-whisker-plot of the ten ensemble members of MEPS. Red line indicating the ensemble mean, lower and upper whisker the 25th and 75th percentile, respectively. Light green shows the median of all members and the box represents the middle \SI{50}{\percent} of scores of the precipitation.}\label{fig:boxplt24}
\end{figure}
%%%%%%%%%%%%%%%%%%%%%%%%%%%%%%%%%%%%%%%%%%%%%%%%%%%%%%%%%%%%%%%%%%%%%%%%%%
To understand what might have led to the overestimation of surface precipitation on the \SI{24}{\dec} in \Cref{fig:sfc_acc24}, a box-whisker-plot is presented. Compared to \SI{21}{\dec} shows the box-whisker-plot in \Cref{fig:boxplt24} an uncertainty between the ten ensemble members already after \SI{3}{\hour} forecast time. The spread between the ensemble members (shown by the minimum and maximum whiskers) seems to be wide. Not all ten members agree on the same precipitation amount as they did for example on \SI{21}{\dec}.
\\
The ensemble mean (red line) is always higher than the median and already after \SI{12}{\hour} forecast time is the median closer to the lower 25th percentile. Also, all upper whiskers are taller than the lower ones, which would follow that the ensemble members vary amongst the most positive quartile and that it is very similar for the least positive quartile group.
A comparison with \Cref{fig:sfc_acc24} shows that most of the member lie beneath the ensemble mean (dashed, blue line). On \SI{24}{\dec} the ensemble mean is much lower than the deterministic forecast, which lies closer to the 50th percentile. This is not for all days the case, on most of the days is the ensemble mean either similar or a little less to the deterministic forecast. Since the deterministic forecast, black line in \Cref{fig:sfc_acc24}, is in the upper percentile compared to its perturbed members it follows that for this forecast the deterministic forecast was not the best guess for the surface accumulation and by using the 'wrong' initial state it can have led to larger miscalculations. Therefore, it would be interesting to perform a new deterministic forecast and its associated perturbations to see if a change in choosing another initial state results in a similar measured precipitation amount at the ground.
\\
\\
The uncertainty appearing already after \SI{3}{\hour} can be associated with a too long spin-up time of MEPS. MEPS usually has a spin-up time of about three hours on \SI{25}{\dec} this might have been longer and followed by poorer initial conditions. To represent the surface accumulation well, the model systems needs to be spin-up. The regional model MEPS needs initial and boundary conditions from ECMWF before it can produce forecasts. Since initial conditions such as observations have uncertainties as well as the model has mistrust and needs to approach its own climatology, a model has to stabilize before the simulations can be trusted. The spin-up time varies depending on the quality of the initial and boundary conditions. Apparently, it seems, that the initial and boundary conditions for MEPS were not perfect on \SI{24}{\dec} at \SI{0}{\UTC} since the deterministic and perturbed members seem not to have stabilised yet and show uncertainties in \Cref{fig:boxplt24} from early on.  At this point it might be interesting to re-run the initialisation again with all available observations to see, if that might have an influence on the overestimation observed in \Cref{fig:sfc_acc24}. It might not necessarily be the observations. Since, ECMWF is the boundary condition of MEPS it could also be that the ECMWF forecast did not have reached its stabilised state when MEPS was initiated.
\\
The uncertainty might also have resulted from the fact, that the precipitation around \SI{0}{\UTC} on \SI{24}{\dec} was higher than on the previous days (see, \Cref{fig:TPU}). Where on the previous days the hourly precipitation around \SI{0}{\UTC} was less intense might a big accretion have followed an uncertainty already after \SI{3}{\hour}. MEPS initialised on \SI{24}{\dec} at \SI{0}{\UTC} might have accounted for an additional precipitation at \SI{12}{\UTC} on \SI{24}{\dec} and that led to the strong increase at \SI{13}{\UTC}. This might be a local effect, that a precipitation cell in the model was spatially misplaced or a by a few kilometres or a higher precipitation amount was expected by the model and actually did not occur at Haukeliseter rather at another site close to Haukeliseter, and followed that strong increase after noon. 
\\
It is therefore important as the double fence construction or measurements from the MRR to give models a good initial condition from observations, so that spin-up time can be reduced and model initialisation start at a realistic state.

%%%%%%%%%%%%%%%%%%%%%%%%%%%%%%%%%%%%%%%%%%%%%%%%%%%%%%%%%%%%%%%%%%%%%%%%%%
%%%%%%%%% surface obs %%%%%%%%%%%%%%
\subsection{Sunday, \SI{25}{\dec}}\label{sec:2512:surface}
On \SI{25}{\dec} the surface accumulation for the first \SI{12}{\hour} is \SI{4.3}{\mm} (see \Cref{tab:sfc_acc}). \cite{muller_arome-metcoop:_2017} stated that the deterministic forecast is showing some overestimation if the \SI{12}{\hour} accumulation is less than \SI{10}{\mm}. Even though the surface accretion is smaller than \SI{10}{\mm} might that not correlate with miscalculation on \SI{25}{\dec}. The overestimation started to be pronounced \SI{13}{\hour} after the initialisation in \Cref{fig:sfc_acc25}.
%
%%% image surface MEPS boxplot %%%%%%%%%%%%%%%%%%%%%%%%%%%%%%%%%%%%%
\begin{figure}[t]
	\includegraphics[width=\textwidth]{./fig_boxplot_sfc/20161225_0}
	\caption{Box-whisker-plot of the ten ensemble members of MEPS initialised on \SI{25}{\dec} at \SI{0}{\UTC}. Red line indicating the ensemble mean, lower and upper whisker the 25th and 75th percentile, respectively. Light green shows the median of all members and the box represents the middle \SI{50}{\percent} of scores of the precipitation.}\label{fig:boxplt25}
\end{figure}
%%%%%%%%%%%%%%%%%%%%%%%%%%%%%%%%%%%%%%%%%%%%%%%%%%%%%%%%%%%%%%%%%%%%%%%%%%
%
Compared to \SI{24}{\dec} are the box-whiskers narrower for the first \SI{30}{\hour} on \SI{25}{\dec} in \Cref{fig:boxplt25}. The overestimation started to occur around \SI{13}{\UTC} in \Cref{fig:sfc_acc25}. As \Cref{fig:boxplt25} shows, increases the uncertainty in the forecast after \SI{15}{\UTC}. In general agree median and mean well for the entire period of a \SI{48}{\hour} forecast. After \SI{39}{\hour} prediction time is the mean much higher than the median and closer to the lower 25th percentile in \Cref{fig:boxplt25}. It seems, that all ten ensemble members agree well on the prediction and nevertheless overestimates MEPS the surface accumulation. It shows that the MEPS estimation follows the double fence amount, just not as high. 
\\
In this case it might have been a miscalculation of the occurrence and amount of the precipitation. From the box-whisker-plot (\Cref{fig:boxplt25}) it seems not to be an initialisation problem, since all members agree and the fact that the ensemble mean agrees with the deterministic run. On \SI{25}{\dec} it was expected from the weather maps that a warm front, the warm sector, and a cold front are going to pass. MEPS might have misinterpreted this passages and expected more, probably liquid precipitation associated with the warm front. 
An error associated with the spin-up time of MEPS is not totally excluded. Since the box-whisker-plot shows a good agreement between all members it is not very likely that this was the problem on \SI{25}{\dec}
\\
That the retrieval underestimates the surface precipitation in the afternoon on \SI{25}{\dec} is due to the total negligence  of liquid precipitation if the surface temperature exceeds \SI{2}{\celsius}. Since the optimal estimation retrieval only uses the moist adiabatic lapse rate of \SI{5}{\kelvin\per\km} it might not represent the true state of the atmosphere. Therefore, a use of radiosonde can provide a real structure of the vertical temperature profile which then can help to give real estimations of solid precipitation in the vertical.
After the optimal estimation retrieval is fully developed it will be interesting to study the combination of liquid and snowfall precipitation. 

%%%%%%%%%%%%%%%%%%%%%%%%%%%%%%%%%%%%%%%%%%%%%%%%%%%%%%%%%%%%%%%%%%%%%%%%%%

%%%%%%%%%%%%%%%%%%%%%%%%%%%%%%%%%%%%%%%%%%%%%%%%%%%%%%%%%%%%%%%%%%%%%%%%%
%%%%%%%% surface %%%%%%%%%%%%%%
% !TeX spellcheck = en_GB
%%%%%%%%%%%%%%%%%%%%%%%%%%%%%%%%%%%%%%%%%%%%%%%%%%%%%%%%%%%%%%%%%%%%%%%%%
%%%%%%%% large scale phenomena vertical ? %%%%%%%%%%%%%%
\section{Observation of large scale weather phenomena in the vertical}%\label{sec:res:verticalSWC}
\label{sec:res:large_scale_vert}
%%%%%%%%% image reflectivity %%%%%%%%%%%%%%
\begin{figure}[h!]
	\centering
	% 23/12
	\begin{subfigure}[t]{\textwidth}
		\centering
		\includegraphics[trim={4.cm 2.5cm 4.5cm 1.5cm},clip,width=0.9\textwidth]{./fig_MRR_refl/MRR_20161223}
		\caption{}\label{fig:ret:refl23}
	\end{subfigure}
	% 25/12
	\begin{subfigure}[t]{\textwidth}
		\centering
		\includegraphics[trim={4.cm 2.5cm 4.5cm 1.5cm},clip,width=0.9\textwidth]{./fig_MRR_refl/MRR_20161225}
		\caption{}\label{fig:ret:refl25}
	\end{subfigure}
	% 26/12
	\begin{subfigure}[t]{\textwidth}
		\centering
		\includegraphics[trim={4.cm 2.5cm 4.5cm 1.5cm},clip,width=0.9\textwidth]{./fig_MRR_refl/MRR_20161226}
		\caption{}\label{fig:ret:refl26}
	\end{subfigure}
	% label
	\begin{subfigure}[t]{\textwidth}
		\includegraphics[trim={6.5cm 0cm 5.3cm 15.5cm},clip,width=\textwidth]{./fig_MRR_refl/MRR_20161226}
	\end{subfigure}
	\caption{MRR reflectivity for the days when a front or an occlusion passed through at Haukeliseter. \SI{}{\decibel Z} reflectivity according to the colour bar, with weaker precipitation in blue and more intense precipitation in red. \protect\subref{fig:ret:refl23}: Friday, \SI{23}{\dec}, \protect\subref{fig:ret:refl25}: Sunday, \SI{25}{\dec}, and \protect\subref{fig:ret:refl26}: Monday, \SI{26}{\dec}.}\label{fig:ret:refl}
\end{figure}
%%%%%%%%%%%%%%%%%%%%%%%%%%%%%%%%%%%%%%%%%%%%%%%%%%%%%%%%%%%%%%%%%%%%%%%%%
\noindent
Frontal boundary passages were observed at the surface several times throughout the extreme storm in December 2016. MEPS is able to predict the large scale features and related surface changes for initialisation more than \SI{24}{\hour} before (\Cref{sec:res:large_scale_sfc}). In winter 2016 three additional instruments were installed to estimate the vertical snow water content at Haukeliseter. This unique approach gives the opportunity to compare the vertical forecasts of SWC to vertical solid precipitation observations. As far as the author knows is there no study on this particular topic about the verification of vertical ensemble member prediction models with observations.
%\\
\\
\Cref{fig:ret:refl} shows the reflectivity from the MRR at Haukeliseter for \SIlist{23;25;26}{\dec}. Passages of occluded fronts and a warm sector were observed on \SIlist{23;26;25}{\dec}, respectively. \Cref{fig:ret:refl26} presents only values until \SI{17}{\UTC}, because of the temperature change and hence a precipitation shift followed liquid drops freezing on the MRR dish and the signal got attenuated. 
\\
The transit of the boundary is shown in \Cref{fig:ret:refl} by the more consistent structure of a storm with higher reflectivity values. 
While on \SIlist{23;26}{\dec} the reflectivity did not pass values larger than \SI{28}{\decibel Z} shows \Cref{fig:ret:refl25} high reflectivity values larger than \SI{30}{\decibel Z} (compare for approximation \Cref{tab:ref_values}). These high values indicate the observation of possible liquid precipitation. Images from the MASC were able to verify observed liquid drops during \SIrange{12}{21}{\UTC} (\Cref{fig:res:obs_masc}). 
\\
On \SI{23}{\dec} allow the surface observations to assume that the occluded front passed through between \SIrange{12}{21}{\UTC} (\Cref{fig:res:sfc_pres23}, \subref{fig:res:sfc_temp23}, \subref{fig:res:sfc_wd23}). The vertical observations at Haukeliseter show intense reflectivity and therefore more intense precipitation after \SI{16}{\UTC} (\Cref{fig:ret:refl23}). Another occlusion passed through on \SI{26}{\dec} shortly before \SI{15}{\UTC} which lasted until \SI{21}{\UTC} indicated by a more consistent storm structure in \Cref{fig:ret:refl26} around \SI{15}{\UTC}. The high reflectivity on both days shows the passage of the occlusion and the associated precipitation. The wind on \SI{23}{\dec} was from the south, upslope (\Cref{fig:res:sfc_wd23}, \Cref{fig:site:kartverket}) which led to a more consistent storm structure. On \SI{25}{\dec} indicate \Cref{fig:res:sfc_wd25} and \subref{fig:res:sfc_ws25} strong wind observations from the west which led to a consistent, but shorter storm structure in \Cref{fig:ret:refl25} at \SI{15}{\UTC}. The orographic influenced wind and therefore a possible relation to the precipitation will be further assessed in \Cref{sec:res:oro_infl}. 
\\
\\
\Cref{fig:SWC} presents the reflectivity from the MRR and the snow water content retrieved from the reflectivity as well as the \SI{48}{\hour} forecast values. Minutely MRR reflectivity and retrieved snow water content can be seen in \Cref{fig:SWC:ret_22}, \subref{fig:SWC:ret_23}, \subref{fig:SWC:ret_25}, and \subref{fig:SWC:ret_26}. \Cref{fig:SWC_EM:22}, \subref{fig:SWC_EM:23}, \subref{fig:SWC_EM:25}, \subref{fig:SWC_EM:26} show in the upper panel the hourly averaged values from the retrieved SWC and in the lower panel the ensemble mean of the instantaneous forecast values every hour over all ensemble member.   
Three hourly averaged retrieved values are then presented in the upper panel of \Cref{fig:SWC3h:22}, \subref{fig:SWC3h:23}, \subref{fig:SWC3h:25}, and \subref{fig:SWC3h:26}, the lower panel are the ensemble mean forecast values every three hours.
\\
\Cref{fig:SWC_EM:23} (lower panel) shows the one hourly averaged forecast values over all ensemble members, neglecting not existing values. Initialisations less than \SI{24}{\hour} before the event predict the consistent retrieved snowfall after \SI{16}{\UTC} (\Cref{fig:SWC:ret_23}, lower panel and \Cref{fig:SWC_EM:23}, upper panel). Even the three hourly averaged forecast values show a response on the occurrence of the storm (\Cref{fig:SWC3h:23}, lower panel). 
The duration of the passage is between \SIrange{16}{23}{\UTC} because of the longer time span is the prediction system able to estimate the snow water content. Forecasts initialised \SI{48}{\hour} prior predict also the consistent storm structure and therefore the passage of the front (\Cref{fig:SWC_EM:22}, \subref{fig:SWC3h:22}). 
\\
In general, is the forecasted instantaneous snow water content amount weaker than the retrieved values for predications on \SI{23}{\dec}. Hourly averages, only using the deterministic forecast and the first ensemble member show no occurrence of the occlusion passage on either day (\Cref{fig:SWC1h:22}, \subref{fig:SWC1h:23}, \subref{fig:SWC1h:25}, \subref{fig:SWC1h:26}). The variation of each ensemble member initialised on the respective day are given in \Cref{fig:EM09}. In \Cref{fig:EM09_22} is the prediction for the occlusion passage quite weak. 
%%%%%%%%% image SWC retrieval MEPS 22 %%%%%%%%%%%%%%
\begin{figure}[H]
	\centering
	% 23/12
	\begin{subfigure}[t]{\textwidth}
		\centering
		\includegraphics[trim={0.cm 2.2cm 19.cm 0.5cm},clip,width=0.9\textwidth]{./fig_obs_ret/20161222}
		\caption{}\label{fig:SWC:ret_22}
	\end{subfigure}
	% EM
	\begin{subfigure}[t]{\textwidth}
		\centering
		\includegraphics[trim={0.cm 2.2cm 19.cm 0.5cm},clip,width=0.9\textwidth]{./fig_vert_SWC_EM/20161222}
		\caption{}\label{fig:SWC_EM:22}
	\end{subfigure}
	% 3h
	\begin{subfigure}[t]{\textwidth}
		\centering
		\includegraphics[trim={0.cm 0.8cm 19.cm 0.5cm},clip,width=0.9\textwidth]{./fig_vert_SWC_3h/20161222}
		\caption{}\label{fig:SWC3h:22}
	\end{subfigure}
	\caption{Initialisation \SIlist{22;23;25;26}{\dec} \SI{0}{\UTC}. 
		(\protect\subref{fig:SWC:ret_22},\protect\subref{fig:SWC:ret_23}, \protect\subref{fig:SWC:ret_25}, \protect\subref{fig:SWC:ret_26}) Upper panel: MRR reflectivity for \SI{48}{\hour}, lower panel minutely retrieved SWC. 
		(\protect\subref{fig:SWC_EM:22}, \protect\subref{fig:SWC_EM:23}, \protect\subref{fig:SWC_EM:25}, \protect\subref{fig:SWC_EM:26}) Upper panel: hourly averaged retrieved SWC, lower panel instantaneous hourly averaged forecast of all ensemble member SWC, neglecting missing values. 
		(\protect\subref{fig:SWC3h:22}, \protect\subref{fig:SWC3h:23}, \protect\subref{fig:SWC3h:25}, \protect\subref{fig:SWC3h:26}) Upper panel three hourly averaged retrieved SWC, lower panel instantaneous three hourly averaged forecast of all ensemble member SWC.   }\label{fig:ret:SWC}
\end{figure}
%%%%%%%%% image SWC retrieval MEPS 23 %%%%%%%%%%%%%%
\begin{figure}[H]\ContinuedFloat
	\centering
	% 23/12
	\begin{subfigure}[t]{\textwidth}
		\centering
		\includegraphics[trim={0.cm 2.2cm 19.cm 0.5cm},clip,width=0.9\textwidth]{./fig_obs_ret/20161223}
		\caption{}\label{fig:SWC:ret_23}
	\end{subfigure}
	% EM
	\begin{subfigure}[t]{\textwidth}
		\centering
		\includegraphics[trim={0.cm 2.2cm 19.cm 0.5cm},clip,width=0.9\textwidth]{./fig_vert_SWC_EM/20161223}
		\caption{}\label{fig:SWC_EM:23}
	\end{subfigure}
	% 3h
	\begin{subfigure}[t]{\textwidth}
		\centering
		\includegraphics[trim={0.cm 0.8cm 19.cm 0.5cm},clip,width=0.9\textwidth]{./fig_vert_SWC_3h/20161223}
		\caption{}\label{fig:SWC3h:23}
	\end{subfigure}
	\caption{\textit{(Continued from previous page.)} Initialisation \SI{23}{\dec}.}
\end{figure}
%%%%%%%%% image SWC retrieval MEPS 25 %%%%%%%%%%%%%%
\begin{figure}[H]\ContinuedFloat
	\centering
	% 25/12
	\begin{subfigure}[t]{\textwidth}
		\centering
		\includegraphics[trim={0.cm 2.2cm 19.cm 0.5cm},clip,width=0.9\textwidth]{./fig_obs_ret/20161225}
		\caption{}\label{fig:SWC:ret_25}
	\end{subfigure}
	% EM
	\begin{subfigure}[t]{\textwidth}
		\centering
		\includegraphics[trim={0.cm 2.2cm 19.cm 0.5cm},clip,width=0.9\textwidth]{./fig_vert_SWC_EM/20161225}
		\caption{}\label{fig:SWC_EM:25}
	\end{subfigure}
	% 3h
	\begin{subfigure}[t]{\textwidth}
		\centering
		\includegraphics[trim={0.cm 0.8cm 19.cm 0.5cm},clip,width=0.9\textwidth]{./fig_vert_SWC_3h/20161225}
		\caption{}\label{fig:SWC3h:25}
	\end{subfigure}
	\caption{\textit{(Continued from previous page.)} Initialisation \SI{25}{\dec}.}
\end{figure}
%%%%%%%%% image SWC retrieval MEPS 26 %%%%%%%%%%%%%%
\begin{figure}[H]\ContinuedFloat
	\centering
	% 25/12
	\begin{subfigure}[t]{\textwidth}
		\centering
		\includegraphics[trim={0.cm 2.2cm 19.cm 0.5cm},clip,width=0.9\textwidth]{./fig_obs_ret/20161226}
		\caption{}\label{fig:SWC:ret_26}
	\end{subfigure}
	% EM
	\begin{subfigure}[t]{\textwidth}
		\centering
		\includegraphics[trim={0.cm 2.2cm 19.cm 0.5cm},clip,width=0.9\textwidth]{./fig_vert_SWC_EM/20161226}
		\caption{}\label{fig:SWC_EM:26}
	\end{subfigure}
	% 3h
	\begin{subfigure}[t]{\textwidth}
		\centering
		\includegraphics[trim={0.cm 0.8cm 19.cm 0.5cm},clip,width=0.9\textwidth]{./fig_vert_SWC_3h/20161226}
		\caption{}\label{fig:SWC3h:26}
	\end{subfigure}
	\caption{\textit{(Continued from previous page.)} Initialisation \SI{26}{\dec}.}
\end{figure}
%%%%%%%%%%%%%%%%%%%%%%%%%%%%%%%%%%%%%%%%%%%%%%%%%%%%%%%%%%%%%%%%%%%%%%%%%
\noindent
It shows also on \SI{23}{\dec} the first perturbed ensemble member does not exist and hence little snow water content is predicted for the ensemble means. 
A comparison with \SIlist{25;26}{\dec} shows the same result. Not much more snow water content is predicted when using the instantaneous values from the deterministic and first perturbed forecast (\Cref{fig:SWC1h:25}, \subref{fig:SWC1h:26}). 
\\ 
On \SI{26}{\dec} when the passage of the occlusion is predicted, the three-hourly instantaneous SWC (\Cref{fig:SWC3h:26}) as well as the average of all ensemble members (\Cref{fig:SWC_EM:26}) predict the frontal passage. 
Already initialisations \SI{39}{\hour} prior let assume that intense precipitation over a short time will occur (\Cref{fig:SWC_EM:25}, \subref{fig:SWC3h:25}). The variation of all members in \Cref{fig:EM09_25} and \subref{fig:EM09_26} indicate that almost all perturbed members would have predicted the precipitation around \SI{16}{\UTC}, but the ensemble mean weakens the result. Higher predicted values appear for deterministic forecasts than for any other ensemble member for initialisations on \SIlist{25;26}{\dec}. This bias might have led to an overestimation at the surface on \SI{26}{\dec}, where the deterministic forecast indicates higher values than the perturbed members (\Cref{fig:sfc_acc26}).  But in \Cref{fig:SWC1h:25} and \subref{fig:SWC1h:26} is the amount of snow water content very weak. 
It shows better estimations for predicted snowfall amount when using either hourly or three hourly time resolution and all ten ensemble members to create the mean than forecasts for hourly averages with only the deterministic and first perturbed member.
Still the instantaneous average values of all ensemble members are much weaker than the retrieved SWC.
\\
%%%%%%% image liquid forecast 25 %%%%%%%%%%%%%%%%
\begin{figure}[t]
	\centering
	\begin{subfigure}[b]{\textwidth}
		\centering
		\includegraphics[trim={0.cm 11.5cm 18.5cm 0.4cm},clip,width=\textwidth]{./fig_vert_LWC_EM/20161224}
		\caption{}\label{fig:LWC:24}
	\end{subfigure}
	\begin{subfigure}[b]{\textwidth}
		\centering
		\includegraphics[trim={0.cm 10cm 18.5cm 0.4cm},clip,width=\textwidth]{./fig_vert_LWC_EM/20161225}
		\caption{}\label{fig:LWC:25}
	\end{subfigure}
	\caption{200m hourly averaged LWC forecast from MEPS with all ensemble members, neglecting missing values.
		Initialised on \SIlist{24;25}{\dec} at \SI{0}{\UTC}. Liquid water content according to the colorbar.}\label{fig:LWC:2425}
\end{figure}
%%%%%%%%%%%%%%%%%%%%%%%%%%%%%%%%%%%%%%%%%%%%%%
The \SI{25}{\dec} showed patterns of liquid precipitation (\Cref{fig:res:obs_masc}) with warm temperatures (\Cref{fig:res:sfc_temp25}) and high reflectivity (\Cref{fig:ret:refl25}) between \SIrange{12}{21}{\UTC}. High reflectivity values in \Cref{fig:ret:refl25} are present around \SI{18}{\UTC} with layer thickness up to \SI{1.2}{\km}. 
To see if liquid precipitation was predicted, the atmospheric cloud condensed water content and rainfall amount in model levels is summed. \Cref{fig:LWC:24} and \subref{fig:LWC:25} show liquid water content for initialisations at either \SI{24}{\dec} or \SI{25}{\dec}. 
Positive surface temperatures were forecasted between \SIrange{12}{21}{\UTC} (\Cref{fig:res:sfc_temp25}). Initialisations more than \SI{24}{\hour} prior show already the occurrence of the liquid layer. \Cref{fig:LWC:24} or \subref{fig:LWC:25} show also a narrow thickness up to \SI{800}{\metre}. 
In Norwegian mountainous terrain is this an important feature since precipitation change can lead to a high risk for people. The avalanche danger increases with the precipitation change especially during high wind speeds. Since MEPS forecasts the liquid layer correctly in depth and length it seems to be a good interaction between the surface model and the vertical prediction. 
This follows a high accuracy of the forecasting system and the advantage of using a high resolution convective scheme model.
\\
\\
%%% table verification %%%%%%%%%%%%%%%%%%%%%%%%%%%%%%%%%%%%%
\begin{table}[t!]
	\begin{center}
		\caption{Interpretation of the coefficient of variation for SWC.} \label{tab:verification}
		\begin{tabular}{lc|c}
			\hline\hline
			\multicolumn{2}{c|}{\textbf{Size of CV}} & {\textbf{Interpretation}} \\ 
			\multicolumn{2}{c|}{[\SI{}{\percent}]} & variability \\ \hline \hline 
			\multicolumn{2}{c|}{\numrange{0}{< 25}} & negligible  \\ \hline
			\multicolumn{2}{c|}{\numrange{25}{< 50}} & low \\ \hline
			\multicolumn{2}{c|}{\numrange{50}{< 75}} & moderate \\ \hline
			\multicolumn{2}{c|}{\numrange{75}{< 100}} & high \\ \hline
			\multicolumn{2}{c|}{\num{100} to $\infty$} & very high  \\ \hline \hline
		\end{tabular}
	\end{center}
\end{table}
%%%%%%%%%%%%%%%%%%%%%%%%%%%%%%%%%%%%%%%%%%%%%%%%%%%%%%%%%%%%%%%%%%%%%%%%%
\noindent
%A validation of how well the forecast performed is difficult to do at this state, since the time resolution of MEPS is coarse compared to the observations. 
For the first glance operates the forecast well when compared to vertical observations. One possibility is to assess the variability of all ensemble member with the coefficient of variation described in \Cref{sec:ens_mean_spread}. 
\Cref{fig:vari:EM22,fig:vari:EM25,fig:vari:EM26} show the coefficient of variation for SWC, which is the standard deviation of the ten ensemble members divided by the mean of all ensemble members. This coefficient gives the possibility to compare the SWC results for different days with different values. It follows for even a low ensemble spread of SWC (standard deviation of all ensemble members) then the different members do not need to be less variable.
\\
The grey line in \Cref{fig:ens_vari} shows the ensemble mean of the hourly predicted SWC values. The darker the colour in \Cref{fig:ens_vari} the smaller is the variation of the SWC relative to the mean. Initialisations on \SI{23}{\dec} does not exist, because it had too few ensemble members (only six) to create a reasonable verification. Therefore, the initialsation on \SI{22}{\dec} is used. The interpretation of the coefficient of variation for SWC is presented in \Cref{tab:verification}.
%%%%%%% image variability %%%%%%%%%%%%%%%%
\begin{figure}[t!]
	\centering
	\begin{subfigure}[b]{\textwidth}
		\includegraphics[trim={0cm 5cm 0cm 0cm},clip,width=\textwidth]{./fig_variation/20161222}
		\caption{}\label{fig:vari:EM22}
	\end{subfigure}
	\begin{subfigure}[b]{\textwidth}
		\includegraphics[trim={0cm 5cm 0cm 0cm},clip,width=\textwidth]{./fig_variation/20161225}
		\caption{}\label{fig:vari:EM25}
	\end{subfigure}
	\begin{subfigure}[b]{\textwidth}
		\includegraphics[trim={0cm 0cm 0cm 0cm},clip,width=\textwidth]{./fig_variation/20161226}
		\caption{}\label{fig:vari:EM26}
	\end{subfigure}
	\caption{SWC variation of the ten ensemble members of MEPS. The lighter the colour according to the colourbar the higher the variation between the perturbed ensemble members. In grey the ensemble mean of all ten members.}\label{fig:ens_vari}
\end{figure}
%%%%%%%%%%%%%%%%%%%%%%%%%%%%%%%%%%%%%%%%%%%%%%
\noindent
The CV agrees well with the prediction of the occlusion on \SI{23}{\dec} after \SI{15}{\UTC}. The variability between the members is small and show a good agreement on the occurrence of snow precipitation. \Cref{fig:EM09_22,fig:EM0923} show the same variability, where all ensemble member agree on the passage after \SI{16}{\UTC}. While comparing only six ensemble members in \Cref{fig:EM09_23}, one could assume that the uncertainty of all ensemble members during the up-slope storm is low, but not as certain as for an initialisation on \SI{22}{\dec} at \SI{0}{\UTC}.
\\
A larger difference between the ensemble members is shown for the passage of the occlusion on \SI{26}{\dec}. Initialisations on \SI{25}{\dec} present a lower variability for the passage after \SI{15}{\UTC} on \SI{26}{\dec} than initialisations less than \SI{24}{\hour} prior. Therefore, an increase of variability after shorter time and an increase in uncertainty. Again, this is not a fair comparison since hourly instantaneous values are used and there might be a time delay of half an hour about the passage which would follow it is not seen in the model forecast. 
\\
Another way to see how the MEPS forecast performed compared to the vertical observations can be done by correlating the observed SWP with the ensemble member snow water content.
% %%% fig correlation %%%%%%%%%%%%%%%%%%%%%%%%%%%%%%%%%%%%%
% \begin{figure}[t!]
% 	\centering
%     \begin{subfigure}[b]{0.49\textwidth}
%     	\includegraphics[width=\textwidth]{./fig_SWP_scat/20161222}
%         \caption{}\label{fig:scatSWP:22}
%     \end{subfigure}
%     \begin{subfigure}[b]{0.49\textwidth}
%     	\includegraphics[width=\textwidth]{./fig_SWP_scat/20161223}
%         \caption{}\label{fig:scatSWP:23}
%     \end{subfigure}
%     \begin{subfigure}[b]{0.49\textwidth}
%     	\includegraphics[width=\textwidth]{./fig_SWP_scat/20161225}
%         \caption{}\label{fig:scatSWP:25}
%     \end{subfigure}
%     \begin{subfigure}[b]{0.49\textwidth}
%     	\includegraphics[width=\textwidth]{./fig_SWP_scat/20161226}
%         \caption{}\label{fig:scatSWP:26}
%     \end{subfigure}
%     \caption{}\label{fig:scatSWP}
% \end{figure}
% %%%%%%%%%%%%%%%%%%%%%%%%%%%%%%%%%%%%%%%%%%%%%%%%%%%%%%%%%%%%%%%%%%%%%%%%%
\\
\\
One question to answer in this work is if the operational model MEPS gets large scale features correctly. As discussed here and in \Cref{sec:res:large_scale_sfc} it seems that the model is able to cover the development of large scale features and its associated precipitation. Even with the intensification of the storm seems MEPS to be able to predict extreme events such as the Christmas event, but might have some issues predicting fast transitions of frontal boundaries. 
\\
MEPS is also able to distinguish between liquid and solid precipitation in layer thickness and duration for time resolution of one hour. This can be a major advantage since a change in temperature and associated precipitation transformation can lead to high safety issues in the Norwegian mountains, especially during winter. With the knowledge more than \SI{24}{\hour} prior can risk notice be send out to the population and rescue teams can prepare in advance. Furthermore, roads and train tracks can be closed to increase the safety of people.
%\\
\textcolor{red}{I'm not sure if the above mentioned should be here?! }
%%%%%%%%%%%%%%%%%%%%%%%%%%%%%%%%%%%%%%%%%%%%%%%%%%%%%%%%%%%%%%%%%%%%%%%%%%
%%%%%%%%% surface overestimation - vertical? %%%%%%%%%%%%%%
%\section{Surface overestimation and the vertical}

%%%%%%%%% Local affects %%%%%%%%%%%%%%
\section{Orographic influence on precipitation}\label{sec:res:oro_infl}
The Haukeliseter is suspended to high wind speeds during the winter. The previous results have shown, that wind plays an important role on the precipitation. The mountain plateau is surrounded by higher mountains to the west and more open to the south east \citep{wolff_measurements_2013,wolff_derivation_2015}, this orography seems to influence the vertical precipitation pattern. The correlation between wind speed observations and forecast have shown an overestimation of predicted wind speed throughout the event (\Cref{fig:scat:ws2123} and \subref{fig:scat:ws2426}). \citet{muller_arome-metcoop:_2017} already mentioned the weakness of too strong wind prediction in AROME-MetCoOp.
\\
On \SIlist{21;23}{\dec} wind directions from the south-east and south were observed, respectively. As earlier discussed in \Cref{sec:res:large_scale_sfc} and \subref{sec:res:large_scale_vert} was the wind change associated with the occlusion passage on \SI{23}{\dec}. The wind direction on \SI{21}{\dec} change was also related to the large scale synoptic flow but was not associated with a frontal passage. A comparison with the large scale weather analysis from ECMWF shows, that the large scale surface wind is from the south-west at \SI{6}{\UTC} on \SI{21}{\dec} (\Cref{fig:GP21_06}) and has changed to west at \SI{12}{\UTC} (\Cref{fig:GP21_12}). The observations at the Haukeliseter site show between \SIrange{6}{12}{\UTC} wind from south-east, while the predicted wind direction is from south in \Cref{fig:res:sfc_wd21}. The local wind direction influenced the precipitation pattern in the vertical in the matter that a more consistent storm structure was observed and predicted between \SIrange{9}{12} {\UTC} (\Cref{fig:SWC:ret_21,fig:SWC_EM:21,fig:SWC3h:21}). 
%%%%%%%%% image SWC retrieval MEPS 21 %%%%%%%%%%%%%%
\begin{figure}[ht!]
	\centering
	% 21/12
	\begin{subfigure}[t]{\textwidth}
		\centering
		\includegraphics[trim={0.cm 2.2cm 19.cm 0.5cm},clip,width=0.9\textwidth]{./fig_obs_ret/20161221}
		\caption{}\label{fig:SWC:ret_21}
	\end{subfigure}
	% EM
	\begin{subfigure}[t]{\textwidth}
		\centering
		\includegraphics[trim={0.cm 2.2cm 19.cm 0.5cm},clip,width=0.9\textwidth]{./fig_vert_SWC_EM/20161221}
		\caption{}\label{fig:SWC_EM:21}
	\end{subfigure}
	% 3h
	\begin{subfigure}[t]{\textwidth}
		\centering
		\includegraphics[trim={0.cm 0.8cm 19.cm 0.5cm},clip,width=0.9\textwidth]{./fig_vert_SWC_3h/20161221}
		\caption{}\label{fig:SWC3h:21}
	\end{subfigure}
	\caption{Initialisation \SI{21}{\dec} \SI{0}{\UTC}. 
		(\protect\subref{fig:SWC:ret_21},\protect\subref{fig:SWC:ret_24}) Upper panel: MRR reflectivity for \SI{48}{\hour}, lower panel minutely retrieved SWC. 
		(\protect\subref{fig:SWC_EM:21}, \protect\subref{fig:SWC_EM:24}) Upper panel: hourly averaged retrieved SWC, lower panel instantaneous hourly averaged forecast of all ensemble member SWC, neglecting missing values. 
		(\protect\subref{fig:SWC3h:21}, \protect\subref{fig:SWC3h:24}) Upper panel three hourly averaged retrieved SWC, lower panel instantaneous three hourly averaged forecast of all ensemble member SWC.   }\label{fig:ret:SWC21_24}
\end{figure}
%%%%%%%%%%%%%%%%%%%%%%%%%%%%%%%%%%%%%%%%%%%%%%
\noindent
Both days show a more consistent storm structure with not as intense snow water content than for storm patterns from the west. 
\Cref{fig:site:kartverket} presents the local topography around Haukeliseter and \Cref{fig:meps:site} shows the topography resolved by MEPS. It shows that MEPS is able to cover some of the complex structure around the site, with the higher mountain to the west and the valley to the south-east. The prediction model seems to forecast the wind direction overall well, only on \SI{21}{\dec} before \SI{10}{\UTC} is a south instead of a south-east wind predicted. It displays that even if the large scale wind is from the south-west is the local wind rather from the south or south-east. The orography in \Cref{fig:site:kartverket} lets assume that large scale south-west wind is forced along the valley laying in south-east direction. As \Cref{fig:SWC_EM:21} indicates is the model obviously able to cover almost the exact timing of the up-slope storm pattern. The variability of each ensemble member is again presented in \Cref{fig:EM09_21}. It shows that almost all ensemble member agree on the occurrence of the storm pattern during \SIrange{9}{12}{\UTC}.
\\
Wind from the west and therefore over mountains followed always a pulsing with more intense precipitation in the vertical (\Cref{fig:ret:SWC} and \ref{fig:ret:SWC2}). This effect might be related to wave breaking at the mountain and result into a pulsing precipitation pattern. More precipitation events need to be studied to understand this effect around the Haukeliseter site. MEPS seems not to cover all pulses during the course of a day, which is related to the time resolution of the forecast values. Since the prediction values exist only every hour the model might miss some of the high pulses \SI{30}{\minute} before or after the occurrence. 
\\
One outcome of the presented study is that MEPS is able to resolve the local topography and predicts the wind direction in almost all the cases correctly. It did not cover the south-east wind direction on \SI{21}{\dec}, which must be related to the local topography. It seems more intuitive for the model to force large scale south-westerly flow into south direction. As seen in \Cref{fig:meps:site} must the wind go along the \SI{7.2}{\degree} longitude, since a higher elevation is to the west and a \SI{1350}{\metre} high mountain to the east. This true prediction of wind direction leads then to the correct estimation of vertical precipitation patterns. 
%%%%%%%%% image SWC retrieval MEPS 24 %%%%%%%%%%%%%%
\begin{figure}[ht!]\ContinuedFloat
	\centering
	% 24/12
	\begin{subfigure}[t]{\textwidth}
		\centering
		\includegraphics[trim={0.cm 2.2cm 19.cm 0.5cm},clip,width=0.9\textwidth]{./fig_obs_ret/20161224}
		\caption{}\label{fig:SWC:ret_24}
	\end{subfigure}
	% EM
	\begin{subfigure}[t]{\textwidth}
		\centering
		\includegraphics[trim={0.cm 2.2cm 19.cm 0.5cm},clip,width=0.9\textwidth]{./fig_vert_SWC_EM/20161224}
		\caption{}\label{fig:SWC_EM:24}
	\end{subfigure}
	% 3h
	\begin{subfigure}[t]{\textwidth}
		\centering
		\includegraphics[trim={0.cm 0.8cm 19.cm 0.5cm},clip,width=0.9\textwidth]{./fig_vert_SWC_3h/20161224}
		\caption{}\label{fig:SWC3h:24}
	\end{subfigure}
	\caption{\textit{(Continued from previous page.)} Initialisation \SI{26}{\dec}.}
\end{figure}
%%%%%%%%%%%%%%%%%%%%%%%%%%%%%%%%%%%%%%%%%%%%%%
\noindent
\Cref{sec:sfc_acc} describes the overestimation of surface snow accumulation during the intensification of the extreme storm. MEPS forecasted more ground accumulation than it was observed. One approach was to see, if the wind might have had an influence on the surface measurement of the double fence, which did not show to be true. A comparison of the hourly values of MEPS, shows that neither on \SI{24}{\dec} nor on \SI{25}{\dec} or \SI{26}{\dec} the vertical snow amount was higher than the observations (\Cref{fig:SWC_EM:24}, \ref{fig:SWC_EM:25}, and \subref{fig:SWC_EM:26}). \Cref{fig:EM09} shows very intense individual ensemble members, but no prominent sign of overestimation when the surface miscalculation was present. 
\\
During \SIrange{24}{26}{\dec} was the wind constantly from the west with higher wind speeds observed than during \SIrange{21}{23}{\dec} (\Cref{fig:res:sfc_wd23}, \subref{fig:res:sfc_wd25}, \subref{fig:res:sfc_wd26}; \Cref{fig:res:sfc_wd21}, \subref{fig:res:sfc_wd22}, \subref{fig:res:sfc_wd24}; \Cref{fig:res:sfc_ws23}, \subref{fig:res:sfc_ws25}, \subref{fig:res:sfc_ws26}, and \Cref{fig:res:sfc_ws21}, \subref{fig:res:sfc_ws22}, \subref{fig:res:sfc_ws24}). \Cref{fig:scat:wd2123} and \subref{fig:scat:wd2426} indicate a better agreement between the forecasted and observed wind directions when precipiation overestimation occurred. During \SIrange{24}{26}{\dec} were the observed wind speeds higher than on the previous days, and as \Cref{fig:scat:ws2123} and \subref{fig:scat:ws2426} present is the correlation between observation and forecasts lower during high wind speeds. The high wind speeds from the west followed a pulsing storm pattern with intense and less intense precipitation. The pulsing pattern is forecasted by MEPS for initialisations longer than \SI{24}{\hour} prior. Since the model gets the wind direction correctly and the affect of local mountains it follows that there seems to be some kind of interaction issue between the vertical snow amount and the surface accumulation. Vertical instantaneous values every hour can lead to a misinterpretation of the here presented results. 
%%%%%%% image variability %%%%%%%%%%%%%%%%
\begin{figure}[t!]
	\centering
	\begin{subfigure}[b]{\textwidth}
		\includegraphics[trim={0cm 0cm 0cm 0cm},clip,width=\textwidth]{./fig_variation/20161224}
		\caption{}\label{fig:vari:EM24}
	\end{subfigure}
	\caption{SWC variation of the ten ensemble members of MEPS. The lighter the colour according to the colourbar the higher the variation between the perturbed ensemble members. In grey the ensemble mean of all ten members.}\label{fig:ens_vari24}
\end{figure}
%%%%%%%%%%%%%%%%%%%%%%%%%%%%%%%%%%%%%%%%%%%%%%
The ensemble variability in \Cref{fig:vari:EM24}, \Cref{fig:vari:EM25}, and \subref{fig:vari:EM26} show that the ensemble members are divided about the existence of the exact pulsing. 
\\
\\
While the wind direction of MEPS has a good agreement shows the wind speed larger values over all days. Although MEPS includes ten perturbed ensemble members the insufficiency of AROME-MetCoOp too high wind prediction in extreme situations is not resolved. The regional model wind prediction is still dependent on the intensity of the storm. As \cite{muller_arome-metcoop:_2017} also mentioned are higher wind speeds in general better forecasted in AROME-MetCoOp than in ECMWF. 
%%%%%%%%%%%%%%%%%%%%%%%%%%%%%%%%%%%%%%%%%%%%%%%%%%%%%%%%%%%%%%%%%%%%%%%%%%
\section{Future work}
\textcolor{red}{Just collecting up-coming disscussion points}
Of course, more storms should be investigated to find the exact correlation between the surface observations and the estimated accumulation to see if the deviation keeps as small for different snow patterns at Haukeliseter. 
\\
\Cref{fig:res:sfc_obs_meps} indicates that MEPS is able to simulate larger scale phenomena which is probably related to the outer boundary conditions of ECMWF as described by \citet{dahlgren_comparison_2013}.
In general, surface parameters are predicted well, only wind speed and precipitation accumulation showed overestimation in MEPS. Wind speed forecasts are higher than observations, which is probably related to the presentation of the orography in MEPS.

%%%%%%%%%%%%%%%%%%%%%%%%%%%%%%%%%%%%%%%%%%%%%%%%%%%%%%%%%%%%%%%%%%%%%%%%%



%
%%%%%%%%%%%%%%%%%%%%%%%%%%%%%%%%%%%%%%%%%%%%%%%%%%%%%%%%%%%%%%%%%%%%%%%%%%%
%%%%%%%%%% SWC COMPARISON %%%%%%%%%%%%%%
%% !TeX spellcheck = en_GB
\section{SWC and SWP from MEPS and the optimal estimation retrieval}
Images for the liquid water content evaluated in MEPS can be found in \Cref{app:LWP_MEPS}. 

%%% table SWP and LWP max %%%%%%%%%%%%%%%%%%%%%%%%%%%%%%%%%%%%%
% !TeX spellcheck = en_GB
\begin{table}[h!]
	\begin{center}
		\caption{Maximum values of the snow water and liquid water content from the retrieval and MEPS}\label{tab:max_val}
		\begin{tabular}{ll|c|c|c|c|c|c} 
			\hline \hline
			& & \textbf{SWC}  & \textbf{HEIGHT}  & \textbf{TIME} & \textbf{LWC}  & \textbf{HEIGHT}  & \textbf{TIME}  \\
			& & [\SI{}{\kg\per\cubic\meter}] & [\SI{}{\meter}] &  & [\SI{}{\kg\per\cubic\meter}] & [\SI{}{\meter}] &   \\
			\hline \hline
			\multicolumn{8}{c}{\textbf{Wed, 21 Dec 2016}} \\ \hline
			\multicolumn{2}{l|}{RETRIEVAL} & 1.08 & 600.0 & \SI{16}{\UTC} & & & \\
			\multicolumn{2}{l|}{MEPS} &  &  & & & & \\
			& ensemble mean & \num{1.24} & \num{1400.0} & \SI{20}{\UTC} & & & \\
			& control & \num{2.11} & \num{1400.0} & \SI{20}{\UTC} & \num{0.15} & \num{2200.0} & \SI{23}{\UTC}\\ \hline \hline
			\multicolumn{8}{c}{\textbf{Thu, 22 Dec 2016}} \\ \hline
			\multicolumn{2}{l|}{RETRIEVAL} & \num{1.46} & \num{1200.0} & \SI{10}{\UTC} & & & \\
			\multicolumn{2}{l|}{MEPS} &  &  & & & & \\
			& ensemble mean & \num{1.54} & \num{2200.0} & \SI{14}{\UTC} & & & \\
			& control & \num{1.35} & \num{1400.0} & \SI{12}{\UTC} & \num{0.20} & \num{2000.0} & 0\SI{02}{\UTC} \\ \hline \hline
			\multicolumn{8}{c}{\textbf{Fri, 23 Dec 2016}} \\ \hline
			\multicolumn{2}{l|}{RETRIEVAL} & \num{0.91} & \num{600.0} & \SI{23}{\UTC} & & & \\
			\multicolumn{2}{l|}{MEPS} &  &  & & & & \\
			& ensembel mean & \num{0.54} & \num{400.0} & \SI{20}{\UTC} & & & \\
			& control & \num{0.54} & \num{400.0} & \SI{20}{\UTC} & \num{0.14} & \num{1000.0} & \SI{15}{\UTC} \\ \hline \hline
			\multicolumn{8}{c}{\textbf{Sat, 24 Dec 2016}} \\ \hline
			\multicolumn{2}{l|}{RETRIEVAL} & \num{1.39} & \num{1000.0} & 0\SI{06}{\UTC} & & & \\
			\multicolumn{2}{l|}{MEPS} &  &  & & & & \\
			& ensembel mean & \num{0.70} & \num{1400.0} & 0\SI{07}{\UTC} & & & \\
			& control & \num{0.73} & \num{1400.0} & \SI{17}{\UTC} & \num{0.33} & \num{1200.0} & 0\SI{09}{\UTC} \\ \hline \hline
			\multicolumn{8}{c}{\textbf{Sun, 25 Dec 2016}} \\ \hline
			\multicolumn{2}{l|}{RETRIEVAL} & \num{0.69} & \num{1400.0} & \SI{21}{\UTC} & & & \\
			& ensembel mean & \num{0.44} & \num{400.0} & \SI{20}{\UTC} & & & \\
			& control & \num{0.50} & \num{800.0} & \SI{20}{\UTC} & \num{0.34} & \num{200.0} & \SI{17}{\UTC} \\ \hline \hline
			\multicolumn{8}{c}{\textbf{Mon, 26 Dec 2016}} \\ \hline
			\multicolumn{2}{l|}{RETRIEVAL} & \num{1.25} & \num{600.0} & \SI{15}{\UTC} & & & \\
			& ensembel mean & \num{0.95} & \num{800.0} & \SI{16}{\UTC} & & & \\
			& control & \num{1.55} & \num{1000.0} & \SI{11}{\UTC} & \num{0.17} & \num{2400.0} & 0\SI{09}{\UTC} \\ \hline \hline
			\multicolumn{8}{c}{\textbf{Tue, 27 Dec 2016}} \\ \hline
			\multicolumn{2}{l|}{RETRIEVAL} & -- & -- & -- & & & \\
			& ensembel mean & \num{0.16} & \num{400.0} & 0\SI{00}{\UTC} & & & \\
			& control & \num{0.16} & \num{400.0} & 0\SI{00}{\UTC} & \num{0.25} & \num{800.0} & \SI{23}{\UTC} \\ \hline \hline
		\end{tabular}
	\end{center}
\end{table}
%%%%%%%%%%%%%%%%%%%%%%%%%%%%%%%%%%%%%%%%%%%%%%%%%%%%%%%%%%%%%%%%%%%%%%%%%%

%%% image SWC, SWP Retrieval MEPS comparison %%%%%%%%%%%%%%%%%%%%%%%%%%%%%%%%%%%%%
% !TeX spellcheck = en_GB
%\begin{landscape}
	\begin{figure}%\ContinuedFloat
		\centering
		% 21/12
		\begin{subfigure}[b]{0.8\textwidth}
			\includegraphics[trim={0.5cm 0.5cm 17.5cm .5cm},clip,width=\textwidth]{./fig_SWC/20161221}
			\caption{}\label{fig:SWC21}
		\end{subfigure}
	\end{figure}
    \begin{figure}\ContinuedFloat
   		\centering
		% 22/12
		\begin{subfigure}[b]{0.8\textwidth}
			\includegraphics[trim={0.5cm 0.5cm 17.5cm .5cm},clip,width=\textwidth]{./fig_SWC/20161222}
			\caption{}\label{fig:SWC22}
		\end{subfigure}
	\end{figure}
    \begin{figure}\ContinuedFloat
   		\centering
		% 23/12
		\begin{subfigure}[b]{0.8\textwidth}
			\includegraphics[trim={0.5cm 0.5cm 17.5cm .5cm},clip,width=\textwidth]{./fig_SWC/20161223}
			\caption{}\label{fig:SWC23}
		\end{subfigure}
	\end{figure}
    \begin{figure}\ContinuedFloat
   		\centering
		% 24/12
		\begin{subfigure}[b]{0.8\textwidth}
			\includegraphics[trim={0.5cm 0.5cm 17.5cm .5cm},clip,width=\textwidth]{./fig_SWC/20161224}
			\caption{}\label{fig:SWC24}
		\end{subfigure}
	\end{figure}
    \begin{figure}\ContinuedFloat
   		\centering
		% 25/12
		\begin{subfigure}[b]{0.8\textwidth}
			\includegraphics[trim={0.5cm 0.5cm 17.5cm .5cm},clip,width=\textwidth]{./fig_SWC/20161225}
			\caption{}\label{fig:SWC25}
		\end{subfigure}
	\end{figure}
    \begin{figure}\ContinuedFloat
   		\centering
		% 26/12
		\begin{subfigure}[b]{0.8\textwidth}
			\includegraphics[trim={0.5cm 0.5cm 17.5cm .5cm},clip,width=\textwidth]{./fig_SWC/20161226}
			\caption{}\label{fig:SWC26}
		\end{subfigure}
        \caption{Upper panel: MRR reflectivity in \SI{}{\dB Z}. 2nd panel: SWC optimal estimation retrieval output every second in \SI{}{\SWC}. 3rd panel: hourly-averaged SWC optimal estimation retrieval output. 4th panel: \SI{200}{\metre}-averaged SWC deterministic forecast from MEPS. Lowest panel: SWP from MEPS, initialised at \SI{00}{\UTC}. Black line represents the deterministic forecast and the grey lines the nine perturbed members. In blue the SWP from the averaged retrieval output.}\label{fig:SWC}
	\end{figure}
	

%%%%%%%%%%%%%%%%%%%%%%%%%%%%%%%%%%%%%%%%%%%%%%%%%%%%%%%%%%%%%%%%%%%%%%%%%%




%%%%%%%%%%%%%%%%%%%%%%%%%%%%%%%%%%%%%%%%%%%%%%%%%%%%%%%%%%%%%%%%%%%%%%%%%%%
%
%%\newpage
%%%%%%%%%%%%%%%%%%%%%%%%%%%%%%%%%%%%%%%%%%%%%%%%%%%%%%%%%%%%%%%%%%%%%%%%%%%
%%%%%%%%%% ensemble verification %%%%%%%%%%%%%%
%% !TeX spellcheck = en_GB
\subsection{Verification of MEPS ensemble members}\label{sec:variation}
%%%%%%%%% image SWC retrieved %%%%%%%%%%%%%%
% !TeX spellcheck = en_GB

%%%%%%%
\begin{figure}[t]
	\centering
	% 20/12
	\begin{subfigure}[t]{\textwidth}		\includegraphics[trim={0.cm 5.3cm 0cm 0cm},clip,width=\textwidth]{./fig_variation/20161220}
		\caption{}\label{fig:ens_vari20}
	\end{subfigure}
	%\end{figure}
	%\begin{figure}\ContinuedFloat
	% 21/12
	\begin{subfigure}[t]{\textwidth}		\includegraphics[trim={0.cm 5.3cm 0cm 0cm},clip,width=\textwidth]{./fig_variation/20161221}
		\caption{}\label{fig:ens_vari21}
	\end{subfigure}
	
	% colourbar
	\begin{subfigure}[t]{\textwidth}		\includegraphics[trim={15.cm 0cm 15cm 21cm},clip,width=\textwidth]{./fig_variation/20161224}
	\end{subfigure}
\end{figure}
\begin{figure}[t]\ContinuedFloat
	% 22/12
	\begin{subfigure}[t]{\textwidth}		\includegraphics[trim={0.cm 5.3cm 0cm 0cm},clip,width=\textwidth]{./fig_variation/20161222}
		\caption{}\label{fig:ens_vari22}
	\end{subfigure}
	% 23/12
	% 		\begin{subfigure}[t]{\textwidth}		\includegraphics[trim={0.cm 0cm 0cm 0cm},clip,width=\textwidth]{./fig_variation/20161223}
	% 			\caption{}\label{fig:ens_vari23}
	% 		\end{subfigure}
	% 24/12
	\begin{subfigure}[t]{\textwidth}		\includegraphics[trim={0.cm 5.3cm 0cm 0cm},clip,width=\textwidth]{./fig_variation/20161224}
		\caption{}\label{fig:ens_vari24}
	\end{subfigure}
	
	% colourbar
	\begin{subfigure}[t]{\textwidth}		\includegraphics[trim={15.cm 0cm 15cm 21cm},clip,width=\textwidth]{./fig_variation/20161224}
	\end{subfigure}
\end{figure}
\begin{figure}[t]\ContinuedFloat
	\centering
	% 25/12
	\begin{subfigure}[t]{\textwidth}		\includegraphics[trim={0.cm 5.3cm 0cm 0cm},clip,width=\textwidth]{./fig_variation/20161225}
		\caption{}\label{fig:ens_vari25}
	\end{subfigure}
	% 26/12
	\begin{subfigure}[t]{\textwidth}		\includegraphics[trim={0.cm 5.3cm 0cm 0cm},clip,width=\textwidth]{./fig_variation/20161226}
		\caption{}\label{fig:ens_vari26}
	\end{subfigure}
	%     % 27/12
	% 		\begin{subfigure}[t]{\textwidth}		\includegraphics[trim={0.cm 5.3cm 0cm 0cm},clip,width=\textwidth]{./fig_variation/20161227}
	% 			\caption{}\label{fig:ens_spread27}
	% 		\end{subfigure}
	
	% colourbar
	\begin{subfigure}[t]{\textwidth}		\includegraphics[trim={15.cm 0cm 15cm 21cm},clip,width=\textwidth]{./fig_variation/20161224}
	\end{subfigure}
	\caption{SWC variation of the ten ensemble members of MEPS. The lighter the colour according to the colourbar the higher the variation between the perturbed ensemble members. In grey the ensemble mean of all ten members.}\label{fig:ens_vari}
\end{figure}


%%%%%%%%%%%%%%%%%%%%%%%%%%%%%%%%%%%%%%%%%%%%%%%%%%%%%%%%%%%%%%%%%%%%%%%%%%
To verify how well the ensemble forecast system MEPS has performed, a verification is performed as described in \Cref{sec:ens_mean_spread}. \Cref{fig:ens_vari20,fig:ens_vari21,fig:ens_vari22,fig:ens_vari24,fig:ens_vari25,fig:ens_vari26} show the coefficient of variation for SWC, which is the standard deviation of the ten ensemble members divided by the mean of all ensemble members. This coefficient gives the possibility to compare the SWC results for different days with different values. It also shows if the ensemble spread (standard deviation of all ensemble members) is low the SWC is does not need to be less variable.
%\\
% %%% table verification %%%%%%%%%%%%%%%%%%%%%%%%%%%%%%%%%%%%%
\begin{table}[t!]
	\begin{center}
		\caption{Interpretation of the coefficient of variation for SWC.} \label{tab:verification}
		\begin{tabular}{lc|c|c}
			\hline\hline
			\multicolumn{2}{c|}{\textbf{Size of CV}} & \multicolumn{2}{c}{\textbf{Interpretation}} \\ 
			\multicolumn{2}{c|}{[\SI{}{\percent}]} & variability & forecast accuracy \\ \hline \hline 
			\multicolumn{2}{c|}{\numrange{0}{< 25}} & negligible & very high  \\ \hline
			\multicolumn{2}{c|}{\numrange{25}{< 50}} & low & high \\ \hline
			\multicolumn{2}{c|}{\numrange{50}{< 75}} & moderate & moderate \\ \hline
			\multicolumn{2}{c|}{\numrange{75}{< 100}} & high & low \\ \hline
			\multicolumn{2}{c|}{\num{100} to $\infty$} & very high & no  \\ \hline \hline
		\end{tabular}
	\end{center}
\end{table}
%%%%%%%%%%%%%%%%%%%%%%%%%%%%%%%%%%%%%%%%%%%%%%%%%%%%%%%%%%%%%%%%%%%%%%%%%%
% 
The grey line in \Cref{fig:ens_vari} shows the ensemble mean as a contour. The darker the colour in \Cref{fig:ens_vari} the smaller is the variation of SWC relative to the mean. The \SI{23}{\dec} does not exist, because it had too few ensemble members (only six) to create a reasonable verification and therefore is the ensemble mean in \Cref{fig:SWC23} classified as very uncertain. The interpretation of the coefficient of variation for SWC is presented in \Cref{tab:verification}.
\\
A small CV indicates a very high forecast accuracy, since the variability is negligible between the ten ensemble members (\SIrange{0}{< 25}{\percent}). Similar is a large CV associated with a low forecast accuracy and therefore a very high variability between the members (\SI{> 100}{\percent}). As expected increases the forecast uncertainty with increasing prediction time. It is still possible that in some cases the CV will be larger with a shorter prediction time than with a longer lead time. This could be the case, when strong synoptic systems with complex structure are apparent.
\\
In some cases increases the forecast accuracy with lead time. It also shows, that the CV agrees well with the prediction of the up-slope events and is more often uncertain about the pulsing part, when west wind is observed. 
\\
\\
The \SI{21}{\dec} contained an up-slope event between \SIrange{9}{13}{\UTC} and a wind change to west followed a pulsed precipitation afterwards. The coefficient of variation of the SWC in \Cref{fig:ens_vari20} shows a variation of up to \SI{100}{\percent} for the up-slope and more than \SI{100}{\percent} for the pulsed part, when initialised on \SI{20}{\dec}. An initialisation on \SI{21}{\dec} gives a better result, were the variation is less with a pronounced accuracy of up to \SI{30}{\percent} for the up-slope part around noon. 
The variation shows another good agreement around \SI{19}{\UTC}, one hour prior the maximum predicted mean SWC. For the maximum observed SWC is the CV higher than \SI{90}{\percent} and thus no forecast accuracy exists. This maximum followed from the fact, that the deterministic forecast is the strongest at time and all other ensemble members respond with a weaker SWC. The very high variability and the discrepancy between the ensemble members will be discussed further in \Cref{sec:vertEM09:2112}.
\\
While the ensemble mean produced a high SWC at \SIlist{11;14}{\UTC} on the \SI{22}{\dec}, shows the variation of SWC a very high variability at this times, when initialised on \SI{21}{\dec} (\Cref{fig:ens_vari21}). This follows a high uncertainty of the SWC peak values in \Cref{fig:SWC21}, whereas it is very certain some hours before when almost no snow water content was present. 
For an initialisation on \SI{22}{\dec} is the peak around \SI{11}{\UTC} merged together with the one at \SI{14}{\UTC}. The CV in \Cref{fig:ens_vari22} displays again a very high forecast accuracy (\SI{< 25}{\percent}) for little SWC and very high variability when the SWC peaks were observed. A moderate to low forecast accuracy can be seen around \SI{3}{\UTC} when the SWC is not higher than \SI{0.3}{\SWC}. 
A reason for this discrepancy is again due to the very high predicted SWC performed by the first ensemble member (\Cref{fig:EM09_22}). Here it shows, that six ensemble members would predict a high SWC around noon, which almost agrees with the vertical retrieved SWC. The pulsing after \SI{18}{\UTC} is forecasted by the first ensemble member and the forth, fifth, and seventh show a possibility of precipitation as well. 
According to the CV in \Cref{fig:ens_vari21} and \ref{fig:ens_vari22} exists there no forecast accuracy for the predicted peaks and the forecasts are only reliable when there is almost no SWC predicted. On \SI{22}{\dec} shows the forecast for initialisations more than \SI{24}{\hour} and less than \SI{24}{\hour} prior a pattern as for few precipitation is the forecast accuracy high to very high and for higher SWC is the forecast accuracy not existing. 
\\
All ensemble members agree well with the occurrence of the up-slope storm on \SI{23}{\dec} (\Cref{fig:EM09_22}). The verification in \Cref{fig:ens_vari22} shows little discrepancy below \SI{50}{\percent} with most having a high forecast accuracy. All ten ensemble members forecast the up-slope to occur after \SI{16}{\UTC}, compare \Cref{fig:EM09_22}. While comparing only six ensemble members in \Cref{fig:EM09_23}, one could assume that the uncertainty of all ensemble members during the up-slope storm is low, but not as certain as for an initialisation on \SI{22}{\dec} at \SI{0}{\UTC}. 
The deterministic forecast (EM0) and ensemble member one in \Cref{fig:EM09_22} indicate peaks of high SWC before \SI{8}{\UTC}. The retrieved SWC on \SI{23}{\dec} had two peaks, one at around \SI{2}{\UTC} and another at \SI{4}{\UTC}. The deterministic forecast, initialised on \SI{22}{\dec} predicted a peak at \SIlist{2;6;8}{\UTC}, where the first ensemble member (EM1) has a strong SWC at \SI{7}{\UTC}. Overall seems a combination of the deterministic and first ensemble member of the  \SI{22}{\dec} initialisation to be a good forecast when comparing to the retrieved SWC in \Cref{fig:SWC22}.
\\
The \SI{24}{\dec} was one of the days, where pulsing of the storm was observed and predicted throughout the day. 
\Cref{fig:EM09_23} can give an idea about the variation between the six existing forecasts. Three ensemble members (EM0, EM7, EM8) seem to agree on the occurrence of a SWC peak around \SI{18}{\UTC}, which would be in the range of a moderate forecast accuracy. 
For an initialisation on \SI{24}{\dec} indicates the variation coefficient of all ten ensemble members in \Cref{fig:ens_vari24} different accuracies. The ensemble mean is presented in grey and shows the pulses forecasted. Until noon on \SI{24}{\dec} is no forecast accuracy for the peaks. The peak observed at \SI{14}{\UTC} indicates a low variability between the ensemble members. The peak at \SI{17}{\UTC} has a high accuracy up to \SI{0.8}{\km} and moderate up to \SI{1.2}{\km}. After \SI{19}{\hour} forecast time is the variability between the ensemble members negligible and all agree on the existence of precipitation.
A detail inter-comparison between the surface accumulation and the vertical snow water content is presented in \Cref{sec:vertEM09:2412}.
\\
In general was the \SI{25}{\dec} a very weak snow storm with strong liquid precipitation observed between \SIlist{12;18}{\UTC}. \Cref{fig:SWC24} and \Cref{fig:SWC25} gave a low value of predicted SWC in the course of a day. As \Cref{fig:ens_vari24} indicates is the forecast accuracy very high up to \SI{1.8}{\km} until noon, this is when liquid precipitation was measured. According to \Cref{fig:LWC24} and \ref{fig:LWC25} was the depth of the liquid layer up to \SI{0.8}{\km}. The variation coefficient has a large disagreement below \SI{0.8}{\km}, but above is the variability between the members not existing or low. Initialisation on \SI{24}{\dec} show weak peak in \Cref{fig:SWC25} at \SI{18}{\UTC}, which had a moderate forecast accuracy, were it is afterwards very high. For an initialisation on \SI{25}{\dec} is the forecast accuracy high until noon (\Cref{fig:ens_vari25}). While liquid precipitation was monitored is the accuracy in the lower layer first not existing and shortly before \SI{18}{\UTC} very high. A high agreement exists for the SWC peak at \SI{20}{\UTC} up to \SI{0.8}{\km} and decreases to be moderate above. A discussion about the precipitation change and its related forecast is given in \Cref{sec:vertEM09:2512}.
\\
Again, the \SI{26}{\dec} is only comparable until \SI{17}{\UTC} even though \Cref{fig:SWC25} would suggest a continues pulsing of the storm. The two peaks around \SI{18}{\UTC} (\Cref{fig:SWC25}) are forecasted with a very high and moderate accuracy in \Cref{fig:ens_vari25}. The SWC peaks at around \SIlist{3;5}{\UTC} show a very high variability. \Cref{fig:EM09_25} shows that four out of ten ensemble members would agree with the peaked event around \SI{5}{\UTC}. Whereas the peak at \SI{3}{\UTC} is dominated by the strong predicted SWC of the deterministic forecast, which follows the high variation in \Cref{fig:ens_vari25}. 
Initialised on \SI{26}{\dec} follows that the SWC peak at \SI{1}{\UTC} is related to a moderate forecast accuracy. Low forecast accuracy is shown for the SWC between \SIrange{9}{12}{\UTC} and the one between \SIrange{15}{18}{\UTC} has a low to moderate variability between the members.  When looking at \Cref{fig:EM09_26} might this disagreement be related to the colourful variation of the vertical predicted SWC. There seems no agreement between the different members about the incidence of the SWC peaks. The high conflict for the CV before noon is most likely related to the high SWC of the deterministic SWC. 
%
% \\ \\
% Another way to verify an ensemble prediction system is to use the ensemble spread of the SWC, which is just the standard deviation of all ten ensemble members, shown in \Cref{fig:ens_spread}. Here, lighter colours of the SWC show more deviation of the SWC around the ensemble mean and darker colours indicate that the ensemble members are close to the mean. Grey contour lines indicate the ensemble mean of the SWC to see any variations.
% \\
% As the results in \Cref{fig:ens_spread} show is the spread very low for the up-slope cases (\Cref{fig:ens_spread20,fig:ens_spread21,fig:ens_spread22,fig:ens_spread23}). This means that all ensemble members perform well when the wind is from the east. 
% \\
% The ensemble spread shows more uncertainty for the pulsing events. 
% Initialisation on \SIlist{21;22;26}{\UTC} shows more spread between the different ensemble members (lighter colour in \Cref{fig:ens_spread21}, \ref{fig:ens_spread22}, and \ref{fig:ens_spread26}), especially for the ensemble mean maximum values. On these days the maximum SWC was quite high and reached the overall ensemble mean maximum of \SI{1.24}{\SWC} on \SI{21}{\dec}.
% Fewer spread between the ensemble members is shown for the initialisation \SIlist{23;24;25}{\UTC} (\Cref{fig:ens_spread23,fig:ens_spread24,fig:ens_spread25}), when the ensemble mean never reached more than \SI{0.54}{\SWC}.

%%%%%%%%%%%%%%%%%%%%%%%%%%%%%%%%%%%%%%%%%%%%%%%%%%%%%%%%%%%%%%%%%%%%%%%%%
\subsection{Wednesday, \SI{21}{\dec}}
%%%%%%%%% vertical obs %%%%%%%%%%%%%%
%\subsection{Vertical snowfall observations}
\label{sec:vertEM09:2112}
% %%% image SWP %%%%%%%%%%%%%%%%%%%%%%%%%%%%%%%%%%%%%
\begin{figure}[h]
	\centering
	\begin{subfigure}[t]{\textwidth}
		\includegraphics[trim={0.4cm .4cm 31.3cm 63.5cm},clip,width=\textwidth]{./fig_SWC/20161220}
		\caption{Initialised: Tuesday, \SI{20}{\dec}}\label{fig:SWP20}
	\end{subfigure}
	\begin{subfigure}[t]{\textwidth}
		\includegraphics[trim={0.4cm .4cm 31.3cm 63.5cm},clip,width=\textwidth]{./fig_SWC/20161221}
		\caption{Initialised: Wednesday, \SI{21}{\dec}}\label{fig:SWP21}
	\end{subfigure}
	\caption{}\label{fig:SWP2021}
\end{figure}
%%%%%%%%%%%%%%%%%%%%%%%%%%%%%%%%%%%%%%%%%%%%%%%%%%%%%%%%%%%%%%%%%%%%%%%%%%
It shows from the SWP image in \Cref{fig:SWP21}, that the deterministic forecast of MEPS (black line) dominates. Most of the other ensemble members (grey line) prognoses the daily maximum snowfall amount two hours earlier than the deterministic forecast. The blue, dashed line, indicating the ensemble mean SWP shows the weakening of the snowfall amount when taking the average of all ten ensemble members with a maximum value at \SI{20}{\UTC}. By comparing the orange line (SWP from the retrieval) and the blue, dashed line it shows, that the ensemble mean value of MEPS gets closer to the observed one, \SIlist{2833; 2162}{\SWP} respectively.
%
% %%% image ensemble member 0-9 %%%%%%%%%%%%%%%%%%%%%%%%%%%%%%%%%%%%%
\begin{figure}[t]
	\centering
	\includegraphics[trim={0cm 0cm 18.3cm 5.1cm},clip,width=0.8\textwidth]{./fig_09EM/20161221}
	\caption{SWC of all ensemble members initialised Wednesday, \SI{21}{\dec} at 0\SI{0}{\UTC} forecast for \SI{48}{\hour}.}\label{fig:EM09_21}
\end{figure}
%%%%%%%%%%%%%%%%%%%%%%%%%%%%%%%%%%%%%%%%%%%%%%%%%%%%%%%%%%%%%%%%%%%%%%%%%%
%
\textcolor{red}{DISCUSSION! Why does MEPS not catch that peak at \SI{16}{\UTC}? Maybe because it is too close to the up-slope storm. Also, Why is the control so high compared to the perturbed members? It catches the up-slope part when also a little weak. Most of the ensemble members would have caught it around \SI{9}{\UTC}. One ensemble member has predicted a high value of SWC at \SI{18}{\UTC}, compare to \Cref{fig:EM09_21}. Why is the up-slope storm more consistent compared to the pulsing? Regional effects? MEPS does well even with catching the pulses and up-slopes, at Haukeliseter is a very difficult orography. }
% %
\newline \noindent
The vertical temperature profile performed with MEPS in \Cref{fig:meps_sound_20} and \ref{fig:meps_sound_21}, shows that an initialisation \SI{36}{\hour} prior to the event would give a cloud with height up to \SI{3}{\km}, as observed in \Cref{fig:SWC21} first panel. An initialisation closer to the occurrence of the storm shows, that MEPS underestimates the intensity and height of the storm (\Cref{fig:meps_sound_21}).
%
%%% image sounding MEPS %%%%%%%%%%%%%%%%%%%%%%%%%%%%%%%%%%%%%
% !TeX spellcheck = en_GB
\begin{figure}
	\centering
	\begin{subfigure}[b]{0.49\textwidth}
		\includegraphics[width=\textwidth]{./fig_Sounding/20161220_36}
		\caption{}\label{fig:meps_sound_20}
	\end{subfigure}
	\begin{subfigure}[b]{0.49\textwidth}
		\includegraphics[width=\textwidth]{./fig_Sounding/20161221_12}
		\caption{}\label{fig:meps_sound_21}
	\end{subfigure}
	\caption{Vertical temperature profiles produced with MEPS. \protect{\subref{fig:meps_sound_20}} is initialised: Tuesday, \SI{20}{\dec} \SI{00}{\UTC}. \protect{\subref{fig:meps_sound_21}} is initialised: Wednesday, \SI{21}{\dec} \SI{00}{\UTC}.}
\end{figure}
%%%%%%%%%%%%%%%%%%%%%%%%%%%%%%%%%%%%%%%%%%%%%%%%%%%%%%%%%%%%%%%%%%%%%%%%%%
\textcolor{red}{DISCUSSION! Bring all into relation with the coefficient of variation.}



%%%%%%%%%%%%%%%%%%%%%%%%%%%%%%%%%%%%%%%%%%%%%%%%%%%%%%%%%%%%%%%%%%%%%%%%
\subsection{Saturday, \SI{24}{\dec}}
%%%%%%%%% vertical obs %%%%%%%%%%%%%%
%\subsection{Vertical snowfall observations}
\label{sec:vertEM09:2412}
% %%% image SWP %%%%%%%%%%%%%%%%%%%%%%%%%%%%%%%%%%%%%
\begin{figure}[t]
	\centering
	\begin{subfigure}[t]{\textwidth}
		\includegraphics[trim={0.4cm .4cm 31.3cm 63.5cm},clip,width=\textwidth]{./fig_SWC/20161223}
		\caption{}\label{fig:SWP23}
	\end{subfigure}
	\begin{subfigure}[t]{\textwidth}
		\includegraphics[trim={0.4cm .4cm 31.3cm 63.5cm},clip,width=\textwidth]{./fig_SWC/20161224}
		\caption{}\label{fig:SWP24}
	\end{subfigure}
	\caption{}\label{fig:SWP2324}
\end{figure}
%%%%%%%%%%%%%%%%%%%%%%%%%%%%%%%%%%%%%%%%%%%%%%%%%%%%%%%%%%%%%%%%%%%%%%%%%%
% text
%
% %%% image ensemble member 0-9 %%%%%%%%%%%%%%%%%%%%%%%%%%%%%%%%%%%%%
\begin{figure}[t]
	\centering
	\includegraphics[trim={0cm 0cm 18.3cm 5.1cm},clip,width=0.8\textwidth]{./fig_09EM/20161224}
	\caption{SWC of all ensemble members initialised Saturday, \SI{24}{\dec} at 0\SI{0}{\UTC} forecast for \SI{48}{\hour}.}\label{fig:EM09_24}
\end{figure}
%%%%%%%%%%%%%%%%%%%%%%%%%%%%%%%%%%%%%%%%%%%%%%%%%%%%%%%%%%%%%%%%%%%%%%%%%%
\textcolor{red}{DISCUSSION! Bring all into relation and include the verification plots}
\begin{itemize}
	\item Because EM3, EM4, EM7 to EM9 are only valid every three hours can precipitation peaks not crop up in such a high frequency as for example the deterministic forecast. 
\end{itemize}


%%%%%%%%%%%%%%%%%%%%%%%%%%%%%%%%%%%%%%%%%%%%%%%%%%%%%%%%%%%%%%%%%%%%%%%%
\subsection{Sunday, \SI{25}{\dec}}
%%%%%%%%% vertical obs %%%%%%%%%%%%%%
%\subsection{Vertical snowfall observations}
\label{sec:vertEM09:2512}
% %%% image SWP %%%%%%%%%%%%%%%%%%%%%%%%%%%%%%%%%%%%%
\begin{figure}[t]
	\centering
	\begin{subfigure}[t]{\textwidth}
		\includegraphics[trim={0.4cm .4cm 31.3cm 63.5cm},clip,width=\textwidth]{./fig_SWC/20161225}
	\end{subfigure}
	\caption{}\label{fig:SWP25}
\end{figure}
%%%%%%%%%%%%%%%%%%%%%%%%%%%%%%%%%%%%%%%%%%%%%%%%%%%%%%%%%%%%%%%%%%%%%%%%%%
% text
%
% %%% image ensemble member 0-9 %%%%%%%%%%%%%%%%%%%%%%%%%%%%%%%%%%%%%
\begin{figure}[t]
	\centering
	\includegraphics[trim={0cm 0cm 18.3cm 5.1cm},clip,width=0.8\textwidth]{./fig_09EM/20161225}
	\caption{SWC of all ensemble members initialised Sunday, \SI{25}{\dec} at 0\SI{0}{\UTC} forecast for \SI{48}{\hour}.}\label{fig:EM09_25}
\end{figure}
%%%%%%%%%%%%%%%%%%%%%%%%%%%%%%%%%%%%%%%%%%%%%%%%%%%%%%%%%%%%%%%%%%%%%%%%%%
\textcolor{red}{DISCUSSION! Bring all into relation and include the verification plots}
%%%%%%%%%%%%%%%%%%%%%%%%%%%%%%%%%%%%%%%%%%%%%%%%%%%%%%%%%%%%%%%%%%%%%%%%
%%%%%%%%%%%%%%%%%%%%%%%%%%%%%%%%%%%%%%%%%%%%%%%%%%%%%%%%%%%%%%%%%%%%%%%%%%

\section{Wednesday, \SI{21}{\dec}}
\textcolor{red}{Get surface accumulation and vertical into relation. Why does the surface overestimate, but the vertical is weaker. Why does the surface overestimate but the vertical seems to catch the observations better?}
\section{Saturday, \SI{24}{\dec}}
\section{Sunday, \SI{25}{\dec}}

% %\newpage
% %%%%%%%%% 21122016 24122016 25122016 %%%%%%%%%%%%%%
% % !TeX spellcheck = en_GB
\section{Wednesday, \SI{21}{\dec}}
Observed precipitation at the Haukelister measurement site started in the morning and last almost continuesly through the event. The precipitation amount was moderately and as described in \Cref{sec:largeScale} was Norway located within a cold air section. 

%%%%%%%%%%%%%%%%%%%%%%%%%%%%%%%%%%%%%%%%%%%%%%%%%%%%%%%%%%%%%%%%%%%%%%%%%%
%%%%%%%%% vertical obs %%%%%%%%%%%%%%
\subsection{Vertical snowfall observations}
% %%% image SWC SWP %%%%%%%%%%%%%%%%%%%%%%%%%%%%%%%%%%%%%
\begin{figure}[h]
	\includegraphics[trim={0.5cm 0.5cm 17.5cm .5cm},clip,width=\textwidth]{./fig_SWC/20161221}
	\caption{Wednesday, \SI{21}{\dec}}\label{fig:SWC21}
\end{figure}
%%%%%%%%%%%%%%%%%%%%%%%%%%%%%%%%%%%%%%%%%%%%%%%%%%%%%%%%%%%%%%%%%%%%%%%%%%
The vertical observations from the MRR and the results of the snowfall retrieval are presented in the three upper panels of \Cref{fig:SWC21}. The first panel shows the transformed radar reflectivity in [\SI{}{\decibel Z}]. Between \SIlist{9;13}{\UTC} was a convective storm observed which afterwards turned into a 'pulsing' with a change of higher and lower reflectivities. 
\\
After applying the forward model (\Cref{sec:forward_model}) the snow water content of the \SI{21}{\dec} is shown. The structure of the convective storm can still be seen, with high values between \SIlist{10;13}{\UTC}. After the pass of the convective storm high reflectivity values up to \SI{25}{\decibel Z} are on and off observed. That follows high snow water content higher than \SI{1.5}{\SWC}. \\
Taking the third panel in \Cref{fig:SWC21} into account, which shows the hourly averaged SWC from the optimal estimation retrieval, one maximum snow water content occurs at \SI{11}{\UTC}, with values up to \SI{0.9}{\SWC}. Associated with the pulsing afterwards is the days SWC maximum of \SI{1.08}{\SWC} monitored at \SI{16}{\UTC} (compare \Cref{tab:max_val}). 
\\
The ensemble mean of all ten ensemble members of MEPS, averaged over \SI{200}{\metre} layers is shown in the third panel. It shows, that the numerical forecast model captures the the convective part of the storm, much weaker but around the same time as observed from the MRR. The high water content around \SI{16}{\UTC} is not observable at all. But MEPS forecasts the pulse at \SI{20}{\UTC} with a maximum of \SI{1.24}{\SWC}.
\\
It shows from the SWP image in \Cref{fig:SWC21}, lower panel that the control run of MEPS (black line) dominates. Most of the other ensemble members (grey line) prognoses the snowfall amount one hour earlier. The blue, dashed line, indicating the ensemble mean SWP shows the weakening of the snowfall amount when taking the average. By comparing the orange line (SWP from the retrieval) and the blue, dashed line it shows, that mean value of MEPS gets closer to the observed one, \SIlist{2833; 2162}{\SWP} respectively.
\\
\textcolor{red}{DISCUSSION! Why does MEPS not catch that peak at \SI{16}{\UTC}? Maybe because it is too close to the convective storm. Also, Why is the control so high compared to the perturbed members?}
%
\newline \noindent
The vertical temperature profile performed with MEPS in \Cref{fig:meps_sound_20} and \ref{fig:meps_sound_21}, shows that an initialisation \SI{36}{\hour} prior to the event would give a cloud with height up to \SI{3}{\km}, as observed in \Cref{fig:SWC21} first panel. An initialisation closer to the occurrence of the storm shows, that MEPS underestimates the intensity and height of the storm (\Cref{fig:meps_sound_21}).
%
%%% image sounding MEPS %%%%%%%%%%%%%%%%%%%%%%%%%%%%%%%%%%%%%
% !TeX spellcheck = en_GB
\begin{figure}
	\centering
	\begin{subfigure}[b]{0.49\textwidth}
		\includegraphics[width=\textwidth]{./fig_Sounding/20161220_36}
		\caption{}\label{fig:meps_sound_20}
	\end{subfigure}
	\begin{subfigure}[b]{0.49\textwidth}
		\includegraphics[width=\textwidth]{./fig_Sounding/20161221_12}
		\caption{}\label{fig:meps_sound_21}
	\end{subfigure}
	\caption{Vertical temperature profiles produced with MEPS. \protect{\subref{fig:meps_sound_20}} is initialised: Tuesday, \SI{20}{\dec} \SI{00}{\UTC}. \protect{\subref{fig:meps_sound_21}} is initialised: Wednesday, \SI{21}{\dec} \SI{00}{\UTC}.}
\end{figure}
%%%%%%%%%%%%%%%%%%%%%%%%%%%%%%%%%%%%%%%%%%%%%%%%%%%%%%%%%%%%%%%%%%%%%%%%%%
%
\newline \noindent
The ensemble spread in \Cref{fig:spread21} shows that almost all the members agree, that snowfall is likely to occur around \SIlist{9;12}{\UTC} the spread of the ensemble members is small. The spread increases with forecast time and peaks at \SI{20}{\UTC} with a maximum spread of \SIrange{1.2}{1.3}{\SWC}. 
% 
%%% image ensemble spread %%%%%%%%%%%%%%%%%%%%%%%%%%%%%%%%%%%%%
\begin{figure}[h]
	\includegraphics[width=\textwidth]{./fig_ensemble_spread/20161221}
	\caption{Ensemble spread indicated by the colour bar. Lighter colours show higher spread and darker colour very small spread of the ensemble members. Grey lines show the ensemble mean of the ten forecast members.}\label{fig:spread21}
\end{figure}
%%%%%%%%%%%%%%%%%%%%%%%%%%%%%%%%%%%%%%%%%%%%%%%%%%%%%%%%%%%%%%%%%%%%%%%%%%
%
%%%%%%%%%%%%%%%%%%%%%%%%%%%%%%%%%%%%%%%%%%%%%%%%%%%%%%%%%%%%%%%%%%%%%%%%%%
%%%%%%%%% surface obs %%%%%%%%%%%%%%
\subsection{Surface accumulation}
% %%% image surface accumulation %%%%%%%%%%%%%%%%%%%%%%%%%%%%%%%%%%%%%
% \begin{figure}[h]
% 			\includegraphics[width=\textwidth]{./fig_sfc_acc/acc_wind_20161221_00}
% 			\caption{}\label{fig:sfc_acc21}
% \end{figure}
% %%%%%%%%%%%%%%%%%%%%%%%%%%%%%%%%%%%%%%%%%%%%%%%%%%%%%%%%%%%%%%%%%%%%%%%%%%
The surface accumulation at the ground showed a good agreement between retrieved snowfall amount, MEPS precipitation amount, and the reference frame of the double fence gauge. Since MEPS had some outlier ensemble members a box-whisker-plot is been provided.
%%% image surface MEPS boxplot %%%%%%%%%%%%%%%%%%%%%%%%%%%%%%%%%%%%%
\begin{figure}[h]
	\includegraphics[width=\textwidth]{./fig_boxplot_sfc/20161221_0}
	\caption{Box-whisker-plot of the ten ensemble members of MEPS. Red line indicating the ensemble mean, lower and upper whisker the 25th and 75th percentile, respectively. Light green shows the median of all members and the box represents the middle \SI{50}{\percent} of scores of the precipitation.}\label{fig:boxplt21}
\end{figure}
%%%%%%%%%%%%%%%%%%%%%%%%%%%%%%%%%%%%%%%%%%%%%%%%%%%%%%%%%%%%%%%%%%%%%%%%%%
The box-whisker-plot in \Cref{fig:boxplt21} shows the distribution of the ten ensemble members. In the first \SI{15}{\hour} of the forecast time agree all members well, since the box and whiskers are small. With increasing forecast time, increases the uncertainty. After \SI{30}{\hour} is the ensemble mean slightly higher than the median of the data. In general can the surface forecast be trusted since the values of the ensemble members are well distributed around the mean.
%%% image surface MEPS boxplot %%%%%%%%%%%%%%%%%%%%%%%%%%%%%%%%%%%%%
\begin{figure}[h]
	%	\centering
	\begin{subfigure}[b]{0.84\textwidth}
		\includegraphics[trim={2.3cm 19.5cm 2.cm .7cm},clip,width=\textwidth]{./fig_windrose/20161221}
	\end{subfigure}
	\begin{subfigure}[b]{0.15\textwidth}
		\includegraphics[trim={50.cm 0.cm 8.3cm 28.4cm},clip,width=\textwidth]{./fig_windrose/20161221}
	\end{subfigure}
	\caption{}\label{fig:wind21}
\end{figure}
%%%%%%%%%%%%%%%%%%%%%%%%%%%%%%%%%%%%%%%%%%%%%%%%%%%%%%%%%%%%%%%%%%%%%%%%%%
A relation between the SWP and the surface wind can be shown in \Cref{fig:wind21}. The scatter plot on the left shows, that  the values obtained by the optimal estimation retrieval are stronger than the deterministic prognoses from MEPS, black dots and black line. The nine perturbed ensemble members in red indicate even weaker forecast than actually measured. 
\\
Comparison between the ensemble mean wind and the weather mast show a good agreement, were MEPS observed more often higher values. 
\\
%%%%%%%%%%%%%%%%%%%%%%%%%%%%%%%%%%%%%%%%%%%%%%%%%%%%%%%%%%%%%%%%%%%%%%%%%%
\textcolor{red}{DISCUSSION! The observations from \Cref{fig:SWC21} have shown, that south easterly wind is associated with convective storms. If west wind was observed the MRR reflectivity showed patterns of convection. West wind is one of the main wind directions at Haukeliseter (\Cref{sec:int:dec_obs}) and the event showed always a 'pulsing' pattern when west wind occurred.  }

























% %%%%%%%%%%%%%%%%%%%%%%%%%%%%%%%%%%%%%%%%%%%%%%%%%%%%%%%%%%%%%%%%%%%%%%%%%
% %\newpage
% %%%%%%%%% 24122016 %%%%%%%%%%%%%%
% % !TeX spellcheck = en_GB
\section{Saturday, \SI{24}{\dec}}
%%% image SWC, SWP Retrieval MEPS comparison %%%%%%%%%%%%%%%%%%%%%%%%%%%%%%%%%%%%%
%% !TeX spellcheck = en_GB
\begin{figure}
	\centering
	\begin{subfigure}[b]{0.49\textwidth}
		\includegraphics[width=\textwidth]{./fig_Sounding/20161220_36}
		\caption{}\label{fig:meps_sound_20}
	\end{subfigure}
	\begin{subfigure}[b]{0.49\textwidth}
		\includegraphics[width=\textwidth]{./fig_Sounding/20161221_12}
		\caption{}\label{fig:meps_sound_21}
	\end{subfigure}
	\caption{Vertical temperature profiles produced with MEPS. \protect{\subref{fig:meps_sound_20}} is initialised: Tuesday, \SI{20}{\dec} \SI{00}{\UTC}. \protect{\subref{fig:meps_sound_21}} is initialised: Wednesday, \SI{21}{\dec} \SI{00}{\UTC}.}
\end{figure}
%%%%%%%%%%%%%%%%%%%%%%%%%%%%%%%%%%%%%%%%%%%%%%%%%%%%%%%%%%%%%%%%%%%%%%%%%%

% %%%%%%%%%%%%%%%%%%%%%%%%%%%%%%%%%%%%%%%%%%%%%%%%%%%%%%%%%%%%%%%%%%%%%%%%
% %%%%%%%%% 25122016 %%%%%%%%%%%%%%
% % !TeX spellcheck = en_GB
\section{Saturday, \SI{25}{\dec}}

% %%%%%%%%%%%%%%%%%%%%%%%%%%%%%%%%%%%%%%%%%%%%%%%%%%%%%%%%%%%%%%%%%%%%%%%%%%


%%%%%%%%% Summary, Conclusion %%%%%%%%%%%%%%
\chapter{Summary and Conclusion}
\textcolor{red}{SUMMARIZE! What did you do? Why did you do it? What did you use? What were your findings? What could be done in the future?}
% Even though the model might have performed well for some days it might be interesting to investigate the same results with an hourly time resolution of all ten ensemble members. Another interesting approach could also be to perturb the ensemble members and initialise them in a different way to see if the actual forecast performed best.

% % % %%%%%%%%%%%%%%%%%%%%%%%%%%%%%%%%%%%%%%%%%%%%%%%%%%%%%%%%%%%%%%%%%%%%%%%%%%
