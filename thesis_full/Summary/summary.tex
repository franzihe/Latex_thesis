% !TeX spellcheck = en_GB
%%%%%%%%% Summary, Conclusion %%%%%%%%%%%%%%

\chapter{Summary and Outlook}
In this thesis, a case study of an extreme event occurring on \num{21} to \SI{26}{\dec} was studied. The Christmas storm in 2016, affected large parts of Eastern, Southern, and Western Norway. The Meteorological Cooperation on Operational Numerical Weather Prediction (MetCoOp) Ensemble Prediction system (MEPS) became operational from November 2016 when it substituted the Mèteo-France Applications of Research to Operations at MEsoscale (AROME)-MetCoOp system at the Norwegian Meteorological Institute. 
The meteorological measurement site Haukeliseter (\SI{991}{\metre} above sea level) was equipped additionally with instruments such as a Micro Rain Radar (MRR), Particle Imaging Package (PIP), and Multi-Angular Snowfall Camera (MASC), during winter 2016/2017. 
%The modified CloudSat optimal estimation retrieval was applied to estimate surface snow amount. 
Since MEPS has just become operational, a unique opportunity was given comparing observations from the World Meteorological Organization station Haukeliseter, and the Norwegian weather forecast model.
%The change from a deterministic forecast to an ensemble prediction system will help to take into account measurement uncertainties. / predictability
\\
%%%% weather analysis
The 2016 Christmas storm was analysed with the help of ECMWF analysis from the surface to the dynamic tropopause level. Meteorological parameters were evaluated to prove if consistent large-scale phenomena were observed and predicted by MEPS.
%%%% sensitivity study + agreement dofe
A sensitivity study of retrieved surface snow accumulation for different a-priori assumptions was implemented. 
%%%% overestimation sfc snowfall
Snow comparison between double fence gauge observations, retrieved profiles of snow water content, and MEPS forecasts were carried out for the World Meteorological Organization measurement site Haukeliseter. 
%%%% vertical analysis
Furthermore, a comparison between  and MEPS forecast was carried out. %Wind related 
\par\medskip
\noindent
%%%%%%%% weather analysis
%A low-pressure east of Iceland and a cold tropopause led to observed mixed-phase precipitation at the Haukeliseter site. A second cyclone developed in the western Atlantic on \SI{22}{\dec} and moved to Norway. 
During \num{21} to \SI{26}{\dec} a low-pressure system developed east of Iceland propagating poleward, followed by a second low-pressure system, which evolved in the western Atlantic, moving eastward.
Temperature changes related to a low and high tropopause, occlusions, and warm sector passages led to precipitation changes %from frozen to liquid 
at Haukeliseter. 
Within the warm sector passage liquid precipitation was observed at Haukeliseter, followed by a landfall of the Christmas storm on \SI{26}{\dec} and dissipation afterwards. 
%%%%%%%%%%%%%%%%%%%%%%%%%%%%%%
\par\medskip
\noindent
%%%%%%%% met analysis
%The closest model grid point to the measurement site Haukeliseter was chosen to answer the research question if MEPS is able to predict large-scale phenomena such as occlusions related to an extreme event.
%It turned out that t
The regional forecast model MEPS is capable of predicting sea level pressure, \SI{2}{\metre} air temperature, and \SI{10}{\metre} wind changes associated to frontal passages and occlusions during the Christmas 2016 storm at Haukeliseter. 
Transitions of the occlusions and the warm sector were predicted \SIlist{24;48}{\hour} in advance for this particular case.
However, MEPS simulated too high wind speed (mean absolute error up to \SI{10}{\mPs}) and surface precipitation amount (mean absolute error up to \SI{15}{\mm}).
\par\medskip
\noindent
%%%%%%%% sensitivity study
%The results for surface snow accumulation agreed better with the double fence gauge observations for rimed particle assumptions (\SI{-5}{\percent}) than the use of a small pristine particles (+\SI{70}{\percent}).
%Double fence gauge measurements are considered as 
According to literature the double fence gauge instrument is  one of the best surface measurements for snow and taken as reference norm for precipitation observations. 
A state of the art optimal estimation snowfall retrieval using a-priori guess from MRR, PIP, and MASC, allowed to compare surface observations to vertically retrieved snow amounts. 
\\
In the sensitivity study assumptions of a particle model for rimed aggregates ('B6'), climatological particle size distributions (PSD) and fall speeds followed the best estimate for surface snowfall accumulation compared to the use of a less reflective CloudSat aggregate model ('B8') 
%the CloudSat PSD estimate %for pristine habits 
at Haukeliseter. 
The results revealed a \SI{70}{\percent} bias for CloudSat aggregate particle model and particle size distribution compared to double fence gauge measurements on \SI{22}{\dec}.
\par\medskip
\noindent
%%%%%%%% agreement estimated acc with dofe
Using particle model assumptions for rimed aggregates led to small differences between observed and estimated snow ($\le$ \SI{-5}{\percent}) and show the importance to choose a priori assumptions correctly to achieve reasonable surface estimates of precipitation amount at the ground. The small deviation gives confidence to trust the vertical estimated snow water content.
\par\medskip
\noindent
%%%%%%%% overestimation of sfc precip
The average difference between observations and forecasted precipitation amount at the surface decreased with increasing lead time (\SI{12}{\hour}: +\SI{135}{\percent}, \SI{24}{\hour}: +\SI{33}{\percent}) for \num{21} to \SI{26}{\dec}. However, inspecting the temporal evolution in detail revealed that the average difference for \SIlist{12;24}{\hour} accumulation   was less than +\SI{11}{\percent} for \num{21} to \SI{23}{\dec} and +\SI{60}{\percent} for \num{24} to \SI{26}{\dec}. Additionally, taking the \SI{10}{\percent} negative bias of the double fence measurements due to wind effects into account, reduces the forecasted surface snow error to +\SI{40}{\percent}.
\par\medskip
\noindent
%%%%%%% vertical analysis
On \SI{25}{\dec}, MASC images allowed to verify the presence of liquid precipitation during the passage of a warm sector at Haukeliseter. MEPS was able to simulate the timing of liquid precipitation \SIlist{24;48}{\hour} in advance. %Vertical 
Comparison between MRR reflectivity profiles and modelled liquid water content showed the accurate simulation of liquid precipitation, both in time and thickness layer, in the lower most atmosphere. 
\\
More variability ($\ge$ \SI{50}{\percent}) between the ensemble members for the short, pulsed than for continuous precipitation patterns. Less variability ($\le$ \SI{25}{\percent}) of snow water content between the ensemble members was simulated for the continues precipitation associated with south-easterlies for initialisations \SIlist{24;48}{\hour} prior, in contrast MEPS ensemble members agreed about the appearance of pulsing snowfall patterns related to westerlies, but not the timing. 
The coefficient of variation showed more variability ($\ge$ \SI{50}{\percent}) between the ensemble members for the short, pulsed than for continuous precipitation pattern. The larger variability of the ensemble members for the pulsing storm patterns is likely related to the temporal resolution of MEPS ensemble forecast data and the short appearance of the pulses of around \SI{30}{\min}.
\\
The hourly averaged estimated snow water content from MRR profiles is larger ($\le$ \SI{1.5}{\SWC}) than the ensemble mean of instantaneous MEPS forecasts ($\le$ \SI{1.2}{\SWC}). In contrast the deterministic and first ensemble member with \SI{1}{\hour} resolution predicted higher snow water content ($\ge$\SI{2.0}{\SWC}) compared to the retrieved values ($\le$\SI{1.5}{\SWC}) %on some days 
during the 2016 Christmas storm, showing the importance of high temporal resolution.
\par\medskip
\noindent
%%% orography
Finally, topographic influence on wind and precipitation next to the impact of horizontal resolution of MEPS were discussed. 
Although MEPS has a high horizontal resolution of \SI{2.5}{\km}, the representation of the %mountainous terrain 
topography in Norway might still be an issue. MEPS %is able to 
resolves some of the major orographic patterns at the Haukeliseter station, such as high mountains to the west and the south-east. %Westerlies and their associated pulsing precipitation pattern were correctly.% predicted by MEPS but the ensemble members showed a large variability about the existence of this storm patterns. 
The one and three hourly ensemble mean forecast of snow water content displayed the ability to predict more continuous, up-slope snow storm patterns related to occlusion passages as well as pulsing precipitation associated to strong westerlies. Additionally the forecasts of westerlies during \num{24} and \SI{26}{\dec} showed a good correlation with observations. In contrast, observed south-easterly winds were predicted as south-westerly wind on \num{21} and \SI{23}{\dec}. %Nevertheless, MEPS is able to relate southerly winds 
%%%% topography and wind
\\
\\
%This study demonstrates %the complex interaction of analysing snowfall extreme events with observations, and regional model prediction.
%how difficult it is to merge different observations to verify a regional model during an extreme event.
The here presented results are a first case study for one winter storm at a Norwegian station in the mountain. Further studies have to be carried out to verify the results.
%Similar approaches to compare a mesoscale ensemble member forecast system with vertical in-situ measurements for snowfall are barely done.  


\section{Outlook}
Only a few studies have addressed similar approaches like this study here by comparing snow observations with weather forecast models. 
\par\medskip
\noindent
First and foremost, it is important to investigate more extreme storm events during winter at Haukeliseter, whether the deviations between observed and retrieved surface snow accumulation stay small. Furthermore, these results should be verified different stations in Norway with similar polar tundra climate, to investigate if the a-priori assumptions can be generalised for as similar local climate. 
\par\medskip
\noindent
It is important to have correct measurements such as the double fence gauge or the radar-particle size distribution retrieval approach. The double fence gauge observations should be investigated further to understand the wind related under-catchment of surface precipitation. Correct measurements will help to improve initial conditions for weather forecast models. %so initialisations can start at a truer state. 
Furthermore, accurate observations will help to get a better understanding of snowfall.
\par\medskip
\noindent
Even though MEPS performed well in the vertical by simulating the wind to the storm structure correctly, it will be interesting to investigate the presented results with a higher time resolution to resolve the short pulses. %Since MEPS overestimates the surface accumulation and the vertical SWC of ensemble member show to be in general smaller than the estimated snow water content (\Cref{fig:SWP}). 
This will help to investigate the simulation of high snow amount at the surface and the relationship between the surface and vertical forecast column.
\par\medskip
\noindent
Sensitivity studies for the outer boundary model could help to understand the influence of European Centre for Medium-Range Weather Forecasts (ECMWF) analysis input data on the MEPS predictions. An initialisation with and without all available observations inside ECMWF will help to investigate the boundary and initial state influence on the wind and precipitation predictions in MEPS.
\\
\\
Finally, more case studies will also help to get a better estimate about the performance of MEPS for snow prediction during winter storm events. The mean absolute error for the \SI{12}{\hour} accumulation of precipitation revealed large variability depending on the initialisation time of MEPS and the state of the cyclone.
