% !TeX spellcheck = en_GB
%%%%%%%%% Summary, Conclusion %%%%%%%%%%%%%%

\chapter{Summary and Outlook}
% \textcolor{red}{SUMMARIZE! What did you do? Why did you do it? What did you use? What were your findings? What could be done in the future?} 
% \\
In this thesis, a case study of an extreme event occurring on \SI{21} to \SI{26}{\dec} has been explored for snowfall. The extreme weather event 'Urd' affected large parts of Eastern, Southern, and Western Norway. During this period, the first low-pressure system developed east of Iceland, and the second cyclone evolved in the western Atlantic. Temperature changes related to the cold and warm fronts, and warm sectors followed observation of snow and liquid precipitation at Haukeliseter.
\\
A sensitivity study of retrieved snow accumulation for different a-prior assumptions was carried out. 
Snowfall comparisons between double fence gauge observations and MEPS forecast of snowfall amount have been investigated at the WMO measurement site Haukeliseter. Meteorological parameters have been evaluated to see if large scale phenomena such as occlusion passages were observed and predicted by MEPS.
Furthermore, a vertical comparison between retrieved profiles of snow water content is compared to \SI{48}{\hour} forecasts of MEPS. Wind related topographical influence on precipitation and the resolution of MEPS were examined.
\par\medskip
\noindent
Double fence gauge measurements are considered as one of the best surface measurements for snowfall. Since additional instruments such as a MRR, MASC and PIP were installed at Haukeliseter during winter 2016/2017, a state of the art optimal estimation snowfall retrieval allowed to compare surface observations to vertically retrieved snowfall amounts. 
Assumptions of a particle model for rimed aggregates, climatological particle size distributions and fall speeds followed the best estimate for surface snowfall accumulation compared to the use of the CloudSat PSD estimate at Haukeliseter. The difference between retrieved snowfall amount and double fence was not larger than \SI{-5}{\percent} for \SIlist{12;24}{\hour} snowfall accumulation. The small difference between observation and estimated snowfall shows that it is important to choose a priori assumptions correctly to achieve reasonable surface estimates precipitation amount at the surface.
\par\medskip
\noindent
AROME-MetCoOp was operational until November 2016 when it got substituted by the ensemble prediction weather forecast model at the Norwegian Meteorological Institute. The change from a deterministic forecast to a ensemble prediction system will help to take into account measurement uncertainties. 
Since MEPS has just become operational, an unique opportunity is given to do first comparisons between the WMO station Haukeliseter, the additional installed instruments, and the weather forecast model.
The closest model grid point to the measurement site Haukeliseter was chosen to answer the research question if MEPS is able to predict large-scale phenomena such as occlusions related to an extreme event.
It turns out that the regional forecast model MEPS is capable of predicting changes associated to frontal passages and occlusions. 
Pressure, temperature, and wind changes associated with passages of fronts were predicted \SIlist{24;48}{\hour} in advance. The correlation between surface observations at the site and the weather forecast model were best for sea level pressure prediction. The mean absolute error for \SI{2}{\metre} temperature was never larger than \SI{1}{\kelvin}, but a warm temperature bias occurred on \SI{24}{\dec}. \textcolor{red}{Include in results? Or double check}
During the extreme Christmas storm in 2016, overestimations of wind and precipitation accumulation at the surface by MEPS were seen. 
\par\medskip
\noindent
In the meteorological analysis an overestimation of surface precipitation accumulation up to \SI{60}{\percent} shows during the intensification of the Christmas 2016 extreme event and hence in detail analysed. The average difference between precipitation amount at the surface decreases with increasing lead time for \num{21} to \SI{26}{\dec}. The \SI{12}{\hour} precipitation accumulation had a mean absolute error up to \SI{12}{\mm} which is approximately eight times larger than the Norwegian mean of December 2014.
\\
A comparison of the wind influenced under-catchment by the double fence of \SI{10}{\percent} showed a small difference error. The surface accumulation was nevertheless overestimated by more than \SI{40}{\percent}. Reasons for the prediction of too much precipitation accumulation at the ground could be initialisation errors as well as the use of one grid point instead of using a variable average of surrounding grid points. The results show that MEPS is not able to predict the precipitation correctly during the Christmas 2016 extreme event. 
\par\medskip
\noindent
The vertical snowfall measurements can be trusted after it was shown that the retrieved estimated snowfall at the surface is in good agreement with those observed at the double fence gauge.
\\
The one and three hourly ensemble means of MEPS displayed the ability to predict more consistent, up-slope storm patterns. For this type of storm the ensemble variability for initialisations \SIlist{24;48}{\hour} prior occurrence was low. MEPS ensemble members prediction were uncertain about the appearance of pulsing patterns related to westerlies. Greater variability between the ensemble members for pulsing storm patterns is probably related to the temporal resolution of MEPS ensemble forecast data and the short appearance of the pulses, of around \SI{30}{\min}.
\\
In general, the estimated snow water content from MRR profiles is greater than the instantaneous average prediction of snowfall. The deterministic and first ensemble member (\SI{1}{\hour} resolution) predicted high snow water contents on some days during the 2016 Christmas storm.
\\
On \SI{25}{\dec}, MASC images could verify the presence of liquid precipitation during the passage of a warm sector at Haukeliseter. MEPS was able to estimate the occurrence of liquid precipitation \SIlist{24;48}{\hour} in advance. Vertical comparison between reflectivity profiles and modelled liquid water content have shown the prediction of liquid precipitation, both in time and layer thickness. This is an important advantage of a regional weather forecast model, as the change from snowfall to liquid precipitation poses a great risk in Norwegian mountains. 
\\
Although MEPS has a horizontal resolution of 2.5km, the representation of the mountainous terrain in Norway might still be an issue. MEPS is able to resolve some of the major orographic patterns, such as high mountains to the west and the south-east of Haukeliseter. Westerlies and their associated pulsing precipitation pattern were correct predicted by MEPS. Forecasts for westerlies during \num{24} and \SI{26}{\dec} showed a good correlation. In contrast, observed south-east winds were often predicted to be south-westerly wind (not along the mountain valley, SE) during \num{21} and \SI{23}{\dec}. Nevertheless is MEPS able to relate southerly winds with up-slope patterns of precipitation.
% \par\medskip
% \noindent
\\
\\
This study demonstrates the complex interaction of analysing snowfall extreme events with observations and regional model prediction.
The here presented results are a first look for only one extreme storm at one station and further studies have to be done. 
%Similar approaches to compare a mesoscale ensemble member forecast system with vertical in-situ measurements for snowfall are barely done.  


\section{Outlook}
Only a few studies have addressed similar approaches like the comparison of vertical snowfall prediction to observations, and thus comparison with other work is difficult. Hence, the here presented results motivate to deal more with the subject of vertical snowfall observation and its comparison to regional weather forecast models. 
\\
First and foremost it is important to investigate more extreme snowfall events during winter at Haukeliseter. To see if the deviations between surface accumulation and estimated precipitation amount from vertical observations keep as small. Furthermore these results should be compared to different stations in Norway with similar polar tundra climate, to find out if a-priori assumptions can be generalised for the same local climate. 
\\
More case studies will also help to get a better estimate about the performance of MEPS during extreme winter events. 
The mean absolute error for \SI{12}{\hour} accumulation has shown a great variability, depending on the initialisation time and the intensification of the low-pressure system. This increase in mean absolute error with intensification might be low when compared to other cases. 
\\
\\
The previous Chapters have indicated, the regional model MEPS is able to predict larger scale phenomena. This might be related to the outer boundary condition ECMWF or the Christmas 2016 extreme event was more predictable. In general, surface parameters such as wind pressure and temperature were predicted well, only wind speed and precipitation accumulation showed overestimation in MEPS predictions. Wind speed forecasts were higher than observations, this might be related to the representation of the orography in MEPS, or the general overestimation of wind speed already apparent in AROME-MetCoOp.
\\
Sensitivity studies for the outer boundary could help to understand how much ECMWF forecast influence the MEPS prediction for local meteorological effects. ECMWF as boundary condition might not have reached its stabilised state when MEPS was initiated during the event, and could also have led to the overestimation of wind speed and surface accumulation. A comparison between ECMWF forecast and MEPS will verify if that might be the case.
Re-analysis data can help to show the uncertainty in the initial and boundary conditions. A re-run with analysis data from ECMWF could possibly improve the original forecast. It will be interesting to initiate MEPS with all available observations, to see if this has an influence on the overestimation of wind and precipitation accumulation.
\\
Another approach could be to perturb the deterministic forecast in other way. Different perturbations might lead to a better correlation between observation and forecast at the ground. Furthermore, the choice of using the closest grid point to Haukeliseter might not have been the best approach. Using four close by grid points instead of only one could help to verify overestimations of MEPS forecasts compared to observations. 
\\
Even though MEPS performed well in the vertical by relating the wind to the storm structure correctly, it will be interesting to investigate the here presented results with a higher time resolution to resolve for the short pulses. Since MEPS overestimates the surface accumulation and the ensemble means show to be in general smaller than the estimated snow water content, the afore mentioned solution help to investigate the overestimation of snowfall at the surface and the relationship between the vertical forecast and surface prediction model.
\\
The local affect of pulsing patterns related to westerlies should be examined. To understand if e.g. wave breaking occurs at the mountain to the west, or if it is an effect of local surface fronts. 
\\
\\
It is important to have correct measurements such as the double fence gauge or the MRR-PSD retrieval approach. Correct measurements will help to improve initial conditions for weather forecast models, so that initialisations can start at the true state. Furthermore accurate observations will help to get a greater understanding of vertical snowfall structure.
Investigating in more detail microphysical processes within mesoscale storms with vertical measurements of snowfall for extreme events can be a grand improvement for weather and climate prediction. 

