% !TeX spellcheck = en_GB
\section{Extreme Weather}
'Extreme weather' is a meteorological term, associated with the extent of a weather type. The Norwegian Meteorological Institute declares an extreme event, if strong winds, large amounts of precipitation and large temperature changes are expected before the event occurs. As well as a large avalanche risk is present and coastal areas are influenced by extremely high-water levels. 
Generally, an event is divided into four phases by Met-Norway to be called extreme event \citep{pedersen_hva_2013}. 
\begin{itemize}
	\setlength\itemsep{-.85em}
	\item[\textbf{Phase A:}] \textit{Increased monitoring before the possible extreme weather.} The meteorologists give special attention to the weather situation. At this point it is not certain, that there will be an extreme weather event.
	\item[\textbf{Phase B:}] \textit{Short-term forecasts.} It is decided, that there will be an extreme event. The forecasts are more detailed, and updates will be published at least every six hours. The event will get a name.
	\item[\textbf{Phase C:}] \textit{The extreme weather is in progress.} The meteorologists send out weather announcements at least every six hours.
	\item[\textbf{Phase D:}] \textit{The extreme weather event is over. Clean-up and repairs are in progress.} When the extreme weather is over the public is notified and information about the upcoming weather and clearing work is given.
\end{itemize}
The Christmas storm was considered an extreme event by the Met-Norway, named 'Urd'
%All this occurred during the Christmas storm
\citep{olsen_ekstremvaerrapport._2017}. 
The average wind along the coast of Western Norway reached hurricane strength (observed: \SIrange{40}{55}{\mPs}). In South and Eastern Norway, west to north-west winds between \SIrange{25}{40}{\mPs} were measured.
At Haukeliseter, \SI{136.4}{\milli\metre} of precipitation were measured from \num{21} to \SI{27}{\dec}.
The 2016 Christmas storm was just above the limit of been called an extreme weather event.
\\
\\
To understand which damage a storm can have, \cite{faeraas_urd_2016} released a table to associate wind strength with damage (see \Cref{tab:wind}).
%%% Damage related to wind speed %%%%%%%%%%%%%%%%%%%%%%%%%%%%%%%%%%%%%
% !TeX spellcheck = en_GB
\begin{table}[h!]
	\begin{center}
		\caption{Damage related to wind speed, from \cite{faeraas_urd_2016}. }\label{tab:wind}
		\begin{tabular}{l|c|l}
			\hline \hline
			slight storm& \SI{20.8}{\mPs} -- \SI{24.4}{\mPs}&  Large trees sway and hiver. \\
			&  &  Roofs can blow down. \\ \hline
			full storm & \SI{24.5}{\mPs} -- \SI{28.4}{\mPs}& Trees are pulled up with clutter. \\
			&  & Big damages to houses.\\ \hline
			strong storm & \SI{28.5}{\mPs} -- \SI{32.6}{\mPs}& Extensive damage.\\
			&  & \\ \hline
			hurricane & \textgreater \SI{32.6}{\mPs}& Unusually large destruction.\\
			\hline \hline
		\end{tabular}
	\end{center}
\end{table}
%%%%%%%%%%%%%%%%%%%%%%%%%%%%%%%%%%%%%%%%%%%%%%%%%%%%%%%%%%%%%%%%%%%%%%%%%%



