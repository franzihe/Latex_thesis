% !TeX spellcheck = en_GB
\section{Extreme weather}
'Extreme weather' is a meteorological term, associated with the extent of a weather type. The Norwegian Meteorological Institute declares an extreme event, if strong winds, large amounts of precipitation and large temperature changes are expected before the event occurs. As well as a large avalanche risk is present and coastal areas are influenced by extremely high-water levels. 
Generally, an event is divided into four phases that it can be called extreme \citep{pedersen_hva_2013}. 
\begin{itemize}
	\setlength\itemsep{-.85em}
	\item[\textbf{Phase A:}] \textit{Increased monitoring before the possible extreme weather.} The meteorologists give special attention to the weather situation. At this point it is not certain, that there will be an extreme weather event.
	\item[\textbf{Phase B:}] \textit{Short-term forecasts.} It is decided, that there will be an extreme event. The forecasts are more detailed, and updates will be published at least every six hours. The event will get a name.
	\item[\textbf{Phase C:}] \textit{The extreme weather is in progress.} The meteorologists send out weather announcements at least every six hours.
	\item[\textbf{Phase D:}] \textit{The extreme weather event is over. Clean-up and repairs are in progress.} When the extreme weather is over the public is notified and information about the upcoming weather and clearing work is given.
\end{itemize}
The Christmas storm was deemed an extrme event by the Met-Norway, named Urd
%All this occurred during the Christmas storm
\citep{olsen_ekstremvaerrapport._2017}. 
The average wind along the coast of Western Norway reached hurricane strength (observed: \SIrange{40}{55}{\mPs}). In South and Eastern Norway, west to north-west winds between \SIrange{25}{40}{\mPs} were measured.
At Haukeliseter, \SI{136.4}{\milli\metre} of precipitation were measured from \num{21} to \SI{27}{\dec}.
The event was just above the limit of been called an extreme weather event.



