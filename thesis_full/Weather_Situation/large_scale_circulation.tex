% !TeX spellcheck = en_GB
\section{Large Scale Circulation} \label{sec:largeScale}
It has long been known that precipitation and temperature fluctuations occur during European winter in connection with NAO (North Atlantic Oscillation). The NAO is a redistribution of atmospheric mass between the North Atlantic high (Azores high), and the polar low (Iceland low).
The NAO index is defined as the gradient between the sea level pressure of the Azores high and the Icelandic low during the winter months (December - March) \citep{hurrell_decadal_1995}. Positive index shows the deepening of the Icelandic low and a strengthening of the Azores high and negative indexes respectively \citep{uvo_analysis_2003}. 
\\
Positive NAO is associated with stronger westerlies than usual across the middle latitudes of the Atlantic \citep{uvo_analysis_2003}. In addition, more moisture is transported to Scandinavia \citep{hurrell_decadal_1995}. A positive NAO during winter is often associated with higher temperatures than normal and with an increase in precipitation in northern Europe and low temperatures in southern Europe \citep{uvo_analysis_2003}.
\\
In December 2016, the NAO was positive, with + \num{0.4}, and negative sea level pressure for the Iceland low \citep{shi_climate_2016}, which might explain the incident of the extreme Christmas storm in 2016.

\subsection*{\num{19} to \SI{20}{\dec}}
A precursor to extreme storm Urd was a high-pressure system over Scandinavia  and an occluded low to the proximate east of Iceland (\Cref{fig:DT19_00}, \subref{fig:GP19_00}). Over the coming \SI{36}{\hour} a confluence of events resulted in a suppressed tropopause and cold air over Haukeliseter: a filament of a cut off low to the south and cold air advection from the north-east (\Cref{fig:DT19_00}, \subref{fig:GP19_00}).
%%%%%%%%%%%%%%%%%%%%%%%%%%%%%%%%%%%%%%%%%%%%%%%%%%%%%%%%%%%%%%
\begin{figure}[ht!]
	\centering
	%%% dyntropo %%%%
	\begin{subfigure}[b]{0.49\textwidth}
		\includegraphics[trim={4.2cm 0cm 4.3cm 5.1cm},clip,
		width=\textwidth]{./fig_DynTropo/20161219_00}
		\caption{} \label{fig:DT19_00}
	\end{subfigure}
	%%% geopot %%%%
	\begin{subfigure}[b]{0.49\textwidth}
		\includegraphics[trim={4.2cm 0cm 4.3cm 5.1cm},clip,
		width=\textwidth]{./fig_Geopot_Jet/20161219_00}
		\caption{} \label{fig:GP19_00}
	\end{subfigure}
	%%% local obs %%%%
	\begin{subfigure}[b]{0.49\textwidth}
		\includegraphics[trim={4.9cm 1.cm 1.5cm 1cm},clip,
		width=\textwidth]{./fig_weathermast/T_P_U_20161219}
		\caption{} \label{fig:TPU19}
	\end{subfigure}
	\caption{ECMWF analysis for dynamic tropopause (\protect\subref{fig:DT19_00}) as described in \Cref{sec:DT}, thickness map (\protect\subref{fig:GP19_00}) as evaluated in \Cref{sec:Geop}, and local observations at Haukeliseter (\protect\subref{fig:TPU19}; \Cref{sec:loc_obs}). Analysis is shown for \SI{19}{\dec} at 0\SI{0}{\UTC}. %\textit{Continued on next page.} 
    }\label{fig:weather:19}
\end{figure}
%%%%%%%%%%%%%%%%%%%%%%%%%%%%%%%%%%%%%%%%%%%%%%%%%%%%%%%%%%%%%%



%%% 21/12
% 21/00
% Westerly flow, perfect for orographic lifting
% → lots of moisture (check for correct values)
% Cold air goes right over Norway 
% → probably snow
% 21/18
% Low forming right at the main baroclinic zone (~45°N, south of Greenland)
% Formation @ the right entrance region of the jet, which helps for lifting
\subsection*{\SI{21}{\dec}}
Associated with the previously mentioned occluded cyclone, cold air impinges on Scandinavia on \SI{21}{\dec} (\Cref{fig:DT21}, \subref{fig:GP21}). Moisture is transported from the low latitudes to high latitudes, influencing Norways west coast. The westerly flow in \Cref{fig:GP21} is additionally conducive to orographic lifting. This is coincident with the onset of precipitation at Haukeliseter at \SI{10}{\UTC} on \SI{21}{\dec} (\Cref{fig:TPU21}). Furthermore, given the low temperatures, solid phase precipitation is observed.
\\
Additionally, cyclogenesis is observed off the east coast of the United States (as evidenced by a trough to the northwest of low-level averaged relative vorticity.) 
%%%%%%%%%%%%%%%%%%%%%%%%%%%%%%%%%%%%%%%%%%%%%%%%%%%%%%%%%%%%%%
\begin{figure}[ht!]
	\centering
	%%% dyntropo %%%%
	\begin{subfigure}[b]{0.49\textwidth}
		\includegraphics[trim={4.2cm 0cm 4.3cm 5.1cm},clip,
		width=\textwidth]{./fig_DynTropo/20161221_06}
		\caption{} \label{fig:DT21_06}
	\end{subfigure}
	\begin{subfigure}[b]{0.49\textwidth}
		\includegraphics[trim={4.2cm 0cm 4.3cm 5.1cm},clip,
		width=\textwidth]{./fig_DynTropo/20161221_12}
		\caption{} \label{fig:DT21}
	\end{subfigure}
	%%% geopot %%%%
	\begin{subfigure}[b]{0.49\textwidth}
		\includegraphics[trim={4.2cm 0cm 4.3cm 5.1cm},clip,
		width=\textwidth]{./fig_Geopot_Jet/20161221_06}
		\caption{} \label{fig:GP21_06}
	\end{subfigure}
	\begin{subfigure}[b]{0.49\textwidth}
		\includegraphics[trim={4.2cm 0cm 4.3cm 5.1cm},clip,
		width=\textwidth]{./fig_Geopot_Jet/20161221_12}
		\caption{} \label{fig:GP21}
	\end{subfigure}
	%%% local obs %%%%
	\begin{subfigure}[b]{0.49\textwidth}
		\includegraphics[trim={4.9cm 1.cm 1.5cm 1cm},clip,
		width=\textwidth]{./fig_weathermast/T_P_U_20161221}
		\caption{} \label{fig:TPU21}
	\end{subfigure}
	\caption{\textit{(As \Cref{fig:weather:19}.)} For \SI{21}{\dec} at \SI{06}{\UTC} (\protect\subref{fig:DT21_06}, \protect\subref{fig:GP21_06}) and at \SI{12}{\UTC} (\protect\subref{fig:DT21}, \protect\subref{fig:GP21}).}\label{fig:weather:21}
\end{figure}
%%%%%%%%%%%%%%%%%%%%%%%%%%%%%%%%%%%%%%%%%%%%%%%%%%%%%%%%%%%%%%


\subsection*{\SI{22}{\dec}}
%%% 22/12
% 22/12
% @ DT, phasing of the vorticity
% → low center is in perfect condition for synoptic scale lifting
% \noindent Twenty-four hours later the analysis shows from \SI{22}{\dec} phasing between the surface relative vorticity and the baroclinic zone at \ang{50}{\,N} in the DT. The centre of the surface low is directly located below the temperature gradient at the \SI{2}{PVU} surface, hence this is good for synoptic lifting. Furthermore, the strongest baroclinicity is observed on the south west side of the surface low.
% The synoptic map of the geopotential thickness and the surface pressure show the beginning of the frontal boundaries in \Cref{fig:GP22}. 
% At the same time shows the AR map, \Cref{fig:AR22}, large values just at the baroclinic zone, where the low pressure is beginning to form. \textcolor{red}{Help?! Does that lead to even more lifting in this area? Or does it just mean that the cyclone gets a good amount of moisture?!}.
% Norway is located in a cold area. The continues precipitation observed at Haukeliseter (\Cref{fig:TPU22}) is associated with the westerly flow which is conducive to orographic lifting, and therefore moisture release.  
Twenty-four hours later the analysis from \SI{22}{\dec} shows cold air remaining over Scandinavia (\Cref{fig:DT22}, \subref{fig:GP22}). Continues frozen precipitation is observed at Haukeliseter (\Cref{fig:TPU22}).
\\
The previously mentioned cyclone is observed to intensify during this time period along the baroclinic zone. The upper-level trough to the north-east of the low-level cyclone centre represents an optimal configuration for cyclone development, since the surface low is located belwo the temperature gradient at the \SI{2}{PVU} surface (\Cref{fig:DT22}, \subref{fig:GP22}). This is apparent in the intensification of the surface system (low-level averaged vorticity and mean sea level pressure minima) at \ang{50}{\,N}.  Also, in evidence is the diabatic ridging associated with the ascending warm conveyor belt air-stream (see warm colours in \Cref{fig:DT22}).
%%%%%%%%%%%%%%%%%%%%%%%%%%%%%%%%%%%%%%%%%%%%%%%%%%%%%%%%%%%%%%
\begin{figure}[ht!]%\ContinuedFloat
	\centering
	%%% dyntropo %%%%
	\begin{subfigure}[b]{0.49\textwidth}
		\includegraphics[trim={4.2cm 0cm 4.3cm 5.1cm},clip,
		width=\textwidth]{./fig_DynTropo/20161222_12}
		\caption{} \label{fig:DT22}
	\end{subfigure}
	%%% geopot %%%%
	\begin{subfigure}[b]{0.49\textwidth}
		\includegraphics[trim={4.2cm 0cm 4.3cm 5.1cm},clip,
		width=\textwidth]{./fig_Geopot_Jet/20161222_12}
		\caption{} \label{fig:GP22}
	\end{subfigure}
	%%% local obs %%%%
	\begin{subfigure}[b]{0.49\textwidth}
		\includegraphics[trim={4.9cm 1.cm 1.5cm 1cm},clip,
		width=\textwidth]{./fig_weathermast/T_P_U_20161222}
		\caption{} \label{fig:TPU22}
	\end{subfigure}
	\caption{\textit{(As \Cref{fig:weather:19}.)} For \SI{22}{\dec} at \SI{12}{\UTC}.}\label{fig:weather:22}
\end{figure}
%%%%%%%%%%%%%%%%%%%%%%%%%%%%%%%%%%%%%%%%%%%%%%%%%%%%%%%%%%%%%%
\newpage
\subsection*{\SI{23}{\dec}}
%%% 23/12
% 23/06
% Still conducive westerly flow → orographic lifting
% Norway in cold area
% Jet not very strong
% Probably frozen precipitation
% Baroclinicity strongest South West of low center
% Low has well-formed frontal boundaries (low level vorticity, → lifting at the cold frontal boundary?)
% 23/12
% Westerlies change
% That is because of the ridging @ DT
% → pushes away the cold air
% First frontal boundary (warm front/Occluded front?) goes through
% 23/18
% Frontal boundary went through, warms up → probably mixed phase/liquid precip.
% Lifting associates with the warm front
% Some forcing due to low level flow / weak warm front
% Southern Norway is briefly into warm air mass, before the cold front comes along
The west Atlantic cyclone translates along the jet stream (\Cref{fig:GP23}, \subref{fig:GP23_18}). The occluded front of the storm arrives roughly between \SIlist{6;12}{\UTC}. By \SI{12}{\UTC} an elevated tropopause is observed in \Cref{fig:DT23}. The surface temperature rises (\Cref{fig:TPU23}), and mixed phase precipitation is related to the increased temperature and moisture transport from the low latitudes \Cref{fig:GP23}.  
\\
A cyclogenesis event occurs in the west Atlantic with a similar disposition of an upper-level trough and a low-level baroclinic zone at \ang{40}{\,N} (\Cref{fig:DT23,fig:DT23_18,fig:GP23,fig:GP23_18}).
%%%%%%%%%%%%%%%%%%%%%%%%%%%%%%%%%%%%%%%%%%%%%%%%%%%%%%%%%%%%%%
\begin{figure}[ht!]%\ContinuedFloat
	\centering
	%%% dyntropo %%%%
	\begin{subfigure}[b]{0.49\textwidth}
		\includegraphics[trim={4.2cm 0cm 4.3cm 5.1cm},clip,
		width=\textwidth]{./fig_DynTropo/20161223_12}
		\caption{} \label{fig:DT23}
	\end{subfigure}
	\begin{subfigure}[b]{0.49\textwidth}
		\includegraphics[trim={4.2cm 0cm 4.3cm 5.1cm},clip,
		width=\textwidth]{./fig_DynTropo/20161223_18}
		\caption{} \label{fig:DT23_18}
	\end{subfigure}
	%%% geopot %%%%
	\begin{subfigure}[b]{0.49\textwidth}
		\includegraphics[trim={4.2cm 0cm 4.3cm 5.1cm},clip,
		width=\textwidth]{./fig_Geopot_Jet/20161223_12}
		\caption{} \label{fig:GP23}
	\end{subfigure}
	\begin{subfigure}[b]{0.49\textwidth}
		\includegraphics[trim={4.2cm 0cm 4.3cm 5.1cm},clip,
		width=\textwidth]{./fig_Geopot_Jet/20161223_18}
		\caption{} \label{fig:GP23_18}
	\end{subfigure}
	%%% local obs %%%%
	\begin{subfigure}[b]{0.49\textwidth}
		\includegraphics[trim={4.9cm 1.cm 1.5cm 1cm},clip,
		width=\textwidth]{./fig_weathermast/T_P_U_20161223}
		\caption{} \label{fig:TPU23}
	\end{subfigure}
	\caption{\textit{(As \Cref{fig:weather:19}.)} For \SI{23}{\dec} at \SI{12}{\UTC} (\protect\subref{fig:DT23}, \protect\subref{fig:GP23}) and at \SI{18}{\UTC} (\protect\subref{fig:DT23_18}, \protect\subref{fig:GP23_18}).}\label{fig:weather:23}
\end{figure}
%%%%%%%%%%%%%%%%%%%%%%%%%%%%%%%%%%%%%%%%%%%%%%%%%%%%%%%%%%%%%%
\newpage
\subsection*{\SI{24}{\dec}}
%%% 24/12
% 24/00
% Norway goes right back into colder air @ DT, @ sfc (thickness lines)
% @ DT lifting associated with the low-level vorticity 
% 24/12
% Cold frontal boundary pushes through
% Norway is back into colder air → reduction in temp.
% Again, westerly flow → good for orographic lifting
After the passage of the occluded front over Norway, passes cold air into Scandinavia (\Cref{fig:DT24}, \subref{fig:GP24}).
The temperature drops in \Cref{fig:TPU24}), and solid phase precipitation resumes. The importance of moisture transport is emphasized by the integrated water vapour plot (\Cref{fig:AR24}). This represents a crucial component to the high precipitation amounts that were measured at the observational site. This represents a quantitative confirmation of previous climatological studies of extreme, cold-season precipitation \citep[][unpublished]{azad_extreme_2017,moore_large_2018}. 
%%%%%%%%%%%%%%%%%%%%%%%%%%%%%%%%%%%%%%%%%%%%%%%%%%%%%%%%%%%%%%
\begin{figure}[ht!]%\ContinuedFloat
	\centering
	%%% dyntropo %%%%
	\begin{subfigure}[b]{0.49\textwidth}
		\includegraphics[trim={4.2cm 0cm 4.3cm 5.1cm},clip,
		width=\textwidth]{./fig_DynTropo/20161224_12}
		\caption{} \label{fig:DT24}
	\end{subfigure}
	%%% geopot %%%%
	\begin{subfigure}[b]{0.49\textwidth}
		\includegraphics[trim={4.2cm 0cm 4.3cm 5.1cm},clip,
		width=\textwidth]{./fig_Geopot_Jet/20161224_12}
		\caption{} \label{fig:GP24}
	\end{subfigure}
	%%% local obs %%%%
	\begin{subfigure}[b]{0.49\textwidth}
		\includegraphics[trim={4.9cm 1.cm 1.5cm 1cm},clip,
		width=\textwidth]{./fig_weathermast/T_P_U_20161224}
		\caption{} \label{fig:TPU24}
	\end{subfigure}
	\caption{\textit{(As \Cref{fig:weather:19}.)} For \SI{24}{\dec} at \SI{12}{\UTC}.}\label{fig:weather:24}
\end{figure}
%%%%%%%%%%%%%%%%%%%%%%%%%%%%%%%%%%%%%%%%%%%%%%%%%%%%%%%%%%%%%%

\subsection*{\SI{25}{\dec}}
%%% 25/12
% 25/00
% @ DT ridging coming in
% @ sfc, low with frontal boundaries
% Jet exit region is directly over Norway → Haukeliseter on right exit region? → sinking?
% Some orographic lifting
% 25/12
% Warm front comes through
% → low-level vorticity → orographic lifting
% 25/18
% Ridging bring more warm air to Norway (should be moist too → check total water vapor)
% Norway in warm sector
% Conducive westerly flow → orographic lifting
Twenty-four hours later the upper level ridge is more pronounced and covers large parts of Norway (\Cref{fig:DT25}. The cyclone south-east of Iceland has built its frontal boundaries (\Cref{fig:GP25}). Between \SIlist{12;18}{\UTC} the warm sector passes through Haukeliseter (\Cref{fig:GP25}, \subref{fig:GP25_18}, \subref{fig:TPU25}). Connected to the warm sector the temperature rises in \Cref{fig:TPU25} and the precipitation becomes liquid. 
%%%%%%%%%%%%%%%%%%%%%%%%%%%%%%%%%%%%%%%%%%%%%%%%%%%%%%%%%%%%%%
\begin{figure}[ht!]%\ContinuedFloat
	\centering
	%%% dyntropo %%%%
	\begin{subfigure}[b]{0.49\textwidth}
		\includegraphics[trim={4.2cm 0cm 4.3cm 5.1cm},clip,
		width=\textwidth]{./fig_DynTropo/20161225_12}
		\caption{} \label{fig:DT25}
	\end{subfigure}
	\begin{subfigure}[b]{0.49\textwidth}
		\includegraphics[trim={4.2cm 0cm 4.3cm 5.1cm},clip,
		width=\textwidth]{./fig_DynTropo/20161225_18}
		\caption{} \label{fig:DT25_18}
	\end{subfigure}
	%%% geopot %%%%
	\begin{subfigure}[b]{0.49\textwidth}
		\includegraphics[trim={4.2cm 0cm 4.3cm 5.1cm},clip,
		width=\textwidth]{./fig_Geopot_Jet/20161225_12}
		\caption{} \label{fig:GP25}
	\end{subfigure}
	\begin{subfigure}[b]{0.49\textwidth}
		\includegraphics[trim={4.2cm 0cm 4.3cm 5.1cm},clip,
		width=\textwidth]{./fig_Geopot_Jet/20161225_18}
		\caption{} \label{fig:GP25_18}
	\end{subfigure}
	%%% local obs %%%%
	\begin{subfigure}[b]{0.49\textwidth}
		\includegraphics[trim={4.9cm 1.cm 1.5cm 1cm},clip,
		width=\textwidth]{./fig_weathermast/T_P_U_20161225}
		\caption{} \label{fig:TPU25}
	\end{subfigure}
	\caption{\textit{(As \Cref{fig:weather:19}.)} For \SI{25}{\dec} at \SI{12}{\UTC} (\protect\subref{fig:DT25}, \protect\subref{fig:GP25}) and at \SI{18}{\UTC} (\protect\subref{fig:DT25_18}, \protect\subref{fig:GP25_18}).}\label{fig:weather:25}
\end{figure}
%%%%%%%%%%%%%%%%%%%%%%%%%%%%%%%%%%%%%%%%%%%%%%%%%%%%%%%%%%%%%%

\newpage
\subsection*{\SI{26}{\dec}}
%%% 26/12
% 26/06
% Cold front went through
% Norway lies in cold area (@ sfc: blue thickness lines, @ DT: cold anomaly)
% 26/18
% North westerly flow at west coast of Norway
% Conducive for orographic lifting, @ DT: strong low-level vorticity gradient
Within the next twenty-four hours the cold sector comes through Haukeliseter and Norway is covered in cold air (\Cref{fig:DT26}, \subref{fig:GP26_18}). The surface pressure indicates the occlusion of the cyclone and therefore a weakening and dissipation by \SI{18}{\UTC}. A drop in temperature and a change in precipitation phase is observed at Haukeliseter (\Cref{fig:TPU26}).
%%%%%%%%%%%%%%%%%%%%%%%%%%%%%%%%%%%%%%%%%%%%%%%%%%%%%%%%%%%%
\begin{figure}[ht!]%\ContinuedFloat
	\centering
	%%% dyntropo %%%%
	\begin{subfigure}[b]{0.49\textwidth}
		\includegraphics[trim={4.2cm 0cm 4.3cm 5.1cm},clip,
		width=\textwidth]{./fig_DynTropo/20161226_12}
		\caption{} \label{fig:DT26}
	\end{subfigure}
	\begin{subfigure}[b]{0.49\textwidth}
		\includegraphics[trim={4.2cm 0cm 4.3cm 5.1cm},clip,
		width=\textwidth]{./fig_DynTropo/20161226_18}
		\caption{} \label{fig:DT26_18}
	\end{subfigure}
	%%% geopot %%%%
	\begin{subfigure}[b]{0.49\textwidth}
		\includegraphics[trim={4.2cm 0cm 4.3cm 5.1cm},clip,
		width=\textwidth]{./fig_Geopot_Jet/20161226_12}
		\caption{} \label{fig:GP26}
	\end{subfigure}
	\begin{subfigure}[b]{0.49\textwidth}
		\includegraphics[trim={4.2cm 0cm 4.3cm 5.1cm},clip,
		width=\textwidth]{./fig_Geopot_Jet/20161226_18}
		\caption{} \label{fig:GP26_18}
	\end{subfigure}
	%%% local obs %%%%
	\begin{subfigure}[b]{0.49\textwidth}
		\includegraphics[trim={4.9cm 1.cm 1.5cm 1cm},clip,
		width=\textwidth]{./fig_weathermast/T_P_U_20161226}
		\caption{} \label{fig:TPU26}
	\end{subfigure}
\caption{\textit{(As \Cref{fig:weather:19}.)} For \SI{26}{\dec} at \SI{12}{\UTC} (\protect\subref{fig:DT26}, \protect\subref{fig:GP26}) and at \SI{18}{\UTC} (\protect\subref{fig:DT26_18}, \protect\subref{fig:GP26_18}).}\label{fig:weather:26}
\end{figure}
%%%%%%%%%%%%%%%%%%%%%%%%%%%%%%%%%%%%%%%%%%%%%%%%%%%%%%%%%%%%%

% %%%%%%%%%%%%%%%%%%%%%%%%%%%%%%%%%%%%%%%%%%%%%%%%%%%%%%%%%%%%%%
% % \subsection*{\SI{27}{\dec}}
% % %%% 27/12
% % \noindent
% % The images of \SI{27}{\dec} show that the storm passed and disappeared. Southern Norway lies in cold air (\Cref{fig:DT27}), but on the right exit region of the jet ($\rightarrow$ sinking motion of cold air), compare \ref{fig:GP27}. A small amount of moisture is present (\Cref{fig:AR27}). Because of the wind change from west to north-west follows that orographic lifting is not present and the precipitation amount decreases at the end of the storm. 



