% !TeX spellcheck = en_GB
\section{Observations}
\label{sec:loc_obs}
%The large scale synoptic analysis will be related to the local weather  observations at Haukeliseter. 
It is a primary goal of this work to relate the local weather observations from the WMO site at Haukeliseter to the synoptic scale structure and the additional measurements taken at Haukeliseter during the winter of 2016/2017.
\\
Examples of the \SI{60}{\minute} precipitation accumulation from the double fence rain gauge, \SI{2}{\metre} air temperature, and wind observations are presented in \Cref{fig:TPU24_pres} to document the continuous precipitation at Haukeliseter during the extreme event. The temperature evolution will be used to investigate possible changes in the type of precipitation. 
\\
Snowfall is likely for temperatures up to \SI{2}{\celsius}. The intensity of the storm can be classified by the hourly averaged wind speed and direction as wind barbs in \SI{}{\mPs}.
To understand which damage a storm can have, \cite{faeraas_urd_2016} released a table to associate wind strength with damage (see \Cref{tab:wind}).
% \\
%%%%%%%%%%%%%%%%%%%%%%%%%
%%% local observations @ Haukeliseter %%%%%%%%%%%%%%%%%%%%%%%%%%%%%%%%%%%%%
\begin{figure}[H]
	\centering
	%%%%%% 24/12
	%	\begin{subfigure}[b]{\textwidth}
	\includegraphics[trim={4.9cm 1.cm 1.5cm 1cm},clip,
	width=0.95\textwidth]{./fig_weathermast/T_P_U_20161224}
	% 		\caption{}\label{fig:TPU24_pres}
	% 	\end{subfigure}
	\caption{Surface precipitation, \SI{2}{\metre} temperature and \SI{10}{\metre} wind observation from the weather mast at Haukeliseter on \SI{24}{\dec}. \SI{60}{\minute} total accumulation [\SI{}{\mm}] in light blue as bar, \SI{2}{\metre} temperature (red, [\SI{}{\celsius}]), and \SI{10}{\minute} mean wind observations at \SI{10}{\metre}, as barbs [\SI{}{\mPs}]. Grey dashed line indicates the freezing temperature and the green dashed line the 30-year climate mean temperature of \SI{-6}{\celsius}. Hourly processed data taken from \cite{eklima_norwegian_2016}.} \label{fig:TPU24_pres}
\end{figure}
%%%%%%%%%%%%%%%%%%%%%%%%%%%%%%%%%%%%%%%%%%%%%%